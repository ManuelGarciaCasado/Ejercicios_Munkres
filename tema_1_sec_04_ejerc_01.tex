\documentclass{article}
% Uncomment the following line to allow the usage of graphics (.png, .jpg)
%\usepackage[pdftex]{graphicx}
% Comment the following line to NOT allow the usage of umlauts



% Start the document
\begin{document}

% Create a new 1st level heading
\section{Tema 1 Sección 4 Ejercicio 01}
% Create a new 1st level heading
\begin{itemize}
\item \bf (a) \rm 
\end{itemize}
Veamos que si \(x+y=x\) entonces \(y=0\). Por la propiedad (3), se existe un \(0\) tal que \(x+0=x\). Como dicho elemento es único, \(0=y\).
\begin{itemize}
\item \bf (b) \rm 
\end{itemize}
Veamos que \(0\cdot x=0\). Por (5) se tiene que \(\left(x+0\right)\cdot x = x\cdot x +0\cdot x\) y por (3) se tiene que \(x\cdot x= \left(x+0\right)\cdot x\). Por tanto \(\left(x+0\right)\cdot x = \left(x+0\right)\cdot x+ 0\cdot x\). Si \(y=\left(x+0\right)\cdot x\) entonces el uso de (3) otra vez implica que \(x\cdot 0\) es \(0\) y es único. Luego \(x\cdot 0 = 0\).
\begin{itemize}
\item \bf (c) \rm 
\end{itemize}
Veamos que \(-0=0\). Por (4), existe un único \(y\), denotado \(-x\), tal que \(x+y=0\) para cada \(x\). Luego también \(0+y=0\) cuando \(x\) es \(0\). Entonces, por (3) se tiene que \(y=0\). Entonces \(y\equiv -0=0\).
\begin{itemize}
\item \bf (d) \rm 
\end{itemize}
Veamos que \(-\left(-x\right)=x\). Por (4), existe un único \(y\), denotado \(-x\), tal que \(x+y=0\) para cada \(x\). En particular, cuando \(x\) es \(-x\), se tiene que hay un único \(y\) tal que \(-x+y=0\) denotado por \(-\left(-x\right)\). Luego \(-x-\left(-x\right)=0\). Por (3) tenemos que \(x+\left[-x-\left(-x\right)\right]=x\). Por (1) se tiene que \(\left(x-x\right)-\left(-x\right)=x\). En el primer parentesis por (4) se tiene que es cero. Luego \(0-\left(-x\right)=x\). Por (2) se tiene \(-\left(-x\right)+0=x\) y por (3), \(-\left(-x\right)=x\).
\begin{itemize}
\item \bf (e) \rm 
\end{itemize}
Veamos que \(x\left(-y\right)=-\left(xy\right)=\left(-x\right)y\). Por (4) se tiene que \(\left(-y\right)+y=0\). De ejercicio (b) se tiene \(x\left[\left(-y\right)+y\right]=0\). Luego, por (5) se tiene \(x\left(-y\right)+xy=0\). Luego por (2), \(xy+x\left(-y\right)=0\). Por (4) el opuesto de \(xy\) es \(-\left(xy\right)\). Luego \(x\left(-y\right)=-\left(xy\right)\) por ser dicho opuesto único. Aplicando (2) se tiene que \(\left(-y\right)x=-\left(yx\right)\). Renombrando \(x\) por \(y\) e \(y\) por \(x\), se tiene que \(\left(-x\right)y=-\left(xy\right)\)
\begin{itemize}
\item \bf (f) \rm 
\end{itemize}
Veamos que \(\left(-1\right)x=-x\). Por (4), se tiene \(1+\left(-1\right)=0\) y por ejercicio (a), \(x\left(1+\left(-1\right)\right)=0\). Entonces (5) implica \(x\cdot1+x\cdot\left(-1\right)=0\). Por (3), \(x+x\cdot\left(-1\right)=0\). Luego por (4), \(x\cdot\left(-1\right)\) es el inverso de \(x\). Es decir \(x\cdot\left(-1\right)=-x\) y, por (2), \(\left(-1\right)x=-x\)
\begin{itemize}
\item \bf (g) \rm 
\end{itemize}
Veamos que \(x\left(y-z\right)=xy-xz\). Por ejercicio (f), se tiene \(x\left(y-z\right)=x\left(y+\left(-1\right)z\right)\). Por (5), \(x\left(y-z\right) = xy+x \left(\left(-1\right)z\right)\).
Por (1), \(x\left(y-z\right) = xy+\left(x\left(-1\right)\right)z\). Por (2) \(x\left(y-z\right) = xy+\left(\left(-1\right)x\right)z\).
Por (f) otra vez, \(x\left(y-z\right)=xy+\left(-xz\right)\), es decir  \(x\left(y-z\right)=xy-xz\).
\begin{itemize}
\item \bf (h) \rm 
\end{itemize}
Veamos que \(-\left(x+y\right)=-x-y,-\left(x-y\right)=-x+y\). Por ejercicio (f) \(-\left(x+y\right)=\left(-1\right)\left(x+y\right)\). Por (5), \(-\left(x+y\right)=\left(-1\right)x + \left(-1\right)y \). Por ejercici (f), \(-\left(x+y\right)=-x-y\).
 Así mismo, por ejercicio (f), \(-\left(x-y\right)=\left(-1\right)\left(x+\left(-1\right)y\right)\).Por (5), \(-\left(x-y\right)=\left(-1\right)x+\left(-1\right)\left(\left(-1\right)y\right)\). Por (1), \(-\left(x-y\right) = \left(-1\right)x+\left(\left(-1\right)\left(-1\right)\right)y\). Por ejercicio (f), \(-\left(x-y\right) = -x+\left(1y\right)\). Es decir,\(-\left(x-y\right)=-x+y\)
\begin{itemize}
\item \bf (i) \rm 
\end{itemize}
Veamos que si \(x\neq0\) entonces \(x\cdot y=x\Longrightarrow y=1\).
Por (3), para cada \(x\) existe un único elemento distinto de cero representado por 1 tal que\(x\cdot 1=x\). Como ha de ser único, si \(x\cdot y=x\) entonces \(y=1\).
\begin{itemize}
\item \bf (j) \rm 
\end{itemize}
Veamos que \(x/x=1\) si \(x\neq 0\). Se tiene que \(x/x=x\left(1/x\right)\) donde \(\left(1/x\right)\) es el \(y\) tal que \(y\cdot x=1\). Por (2) se tiene que \(x\cdot y=1\). Por (4) este \(y\) es único. Luego \(y=1/x\). Luego \(x/1=x\left(1/1\right)= x\cdot 1\). Por (3), esto es \(x/1=x\left(1/1\right)= x\cdot 1=x\).
\begin{itemize}
\item \bf (k) \rm 
\end{itemize}
Veamos que \(x/1=x\). Se tiene que \(x/1=x\left(1/1\right)\) donde \(\left(1/1\right)\) es el \(y\) tal que \(y\cdot 1=1\). Por (3) se tiene que \(y\cdot 1=y\). Luego \(y=1\). Luego \(x/1=x\left(1/1\right)= x\cdot 1\). Por (3), esto es \(x/1=x\left(1/1\right)= x\cdot 1=x\).
\begin{itemize}
\item \bf (l) \rm 
\end{itemize}
Veamos que si \(x\neq0\) e \(y\neq 0\) entonces \(xy\neq0\). Supongamos que \(y\neq0\) y \(xy=0\), entonces, por ejercicio (b) se tiene que \(x=0\). Por lo tanto decir \(\left(xy=0\text{ y } y\neq0\right) \Longrightarrow x=0\) es lo mismo que decir \(x\neq0 \Longrightarrow \left(xy \neq0 \text{ o } y=0\right)\). Del mismo modo
decir \(\left(xy=0\text{ y }x\neq0\right) \Longrightarrow y=0\) es lo mismo que decir \(x\neq0\Longrightarrow\left(xy\neq0\text{ o }x= 0\right)\). Por tanto \( \left( x\neq0 \text{ y }y \neq 0 \right)\Longrightarrow \left(\left(xy\neq0\text{ o }x=0\right)\right. \) y 
\(\left.\left(xy\neq0\text{ o }y=0\right)\right)\). Por tanto \( \left( x\neq 0 \text{ y }y \neq 0 \right)\Longrightarrow \left( xy \neq 0 \right)\)
\begin{itemize}
\item \bf (m) \rm 
\end{itemize}
Veamos que \(\left(1/y\right)\left(1/z\right)=1/\left(yz\right)\) si\(y,z\neq0\). Por (2), \(\left[\left(1/y\right)\left(1/z\right)\right]\left(yz\right)=\left[\left(1/y\right)\left(1/z\right)\right]\left(zy\right)\). Por (1), \(\left[\left(1/y\right)\left(1/z\right)\right]\left(yz\right)=\left\{\left(1/y\right)\left[\left(1/z\right)z\right]y\right\}\). Por ejercicio (j), \(\left[\left(1/y\right)\left(1/z\right)\right]\left(yz\right)=\left\{\left(1/y\right)\left[1\cdot y\right]\right\}=1 \cdot 1=1\). Por (4), se denomina \(1/yz\) al numero \(\left[\left(1/y\right)\left(1/z\right)\right]\), inverso de \(yz\). Luego \(\left(1/y\right)\left(1/z\right)=1/\left(yz\right)\)
\begin{itemize}
\item \bf (n) \rm 
\end{itemize}
Veamos que \(\left(x/y\right)\left(w/z\right)=\left(xw\right)/\left(zy\right)\). Por definicion, se tiene que \(\left(x/y\right)\left(w/z\right)=\left[x\cdot\left(1/y\right)\right]\cdot\left[w\cdot\left(1/z\right)\right]\). Por (1) otra vez, \(\left(x/y\right)\left(w/z\right)=x\cdot\left\{\left[\left(1/y\right)w\right]\cdot\left(1/z\right)\right\}\). Por (2), \(\left(x/y\right)\left(w/z\right)=\left\{x\cdot\left[w\left(1/y\right)\right]\right\}\cdot\left(1/z\right)\). Por (1) otra vez, \(\left(x/y\right)\left(w/z\right)=\left[x\cdot w\right]\cdot \left[\left(1/y\right)\cdot\left(1/z\right)\right]\). Por ejercicio (m),  \(\left(x/y\right)\left(w/z\right)=\left(xw\right)\cdot \left(1/yz\right)\). Por definicion, \(\left(x/y\right)\left(w/z\right)=xw/yz\).
\begin{itemize}
\item \bf (o) \rm 
\end{itemize}
Veamos que \(\left(x/y\right)+\left(w/z\right)=\left(xz+wy\right)/\left(zy\right)\). Por ejercicio (3), se tiene \(\left(x/y\right)+\left(w/z\right)=\left[\left(x/y\right)\cdot 1\right]+\left[\left(w/z\right)\cdot 1\right]\). Por ejercicio (j), se tiene \(\left(x/y\right)+\left(w/z\right)=\left[\left(x/y\right)\cdot \left( z/z\right)\right]+\left[\left(w/z\right)\cdot \left(y/y\right)\right]\). por ejercicio (n), se tiene \(\left(x/y\right)+\left(w/z\right)=\left(xz/yz\right)+\left(wy/zy\right)\). Por (2), se tiene \(\left(x/y\right)+\left(w/z\right)=\left(xz/yz\right)+\left(wy/yz\right)\). Por definicion, se tiene \(\left(x/y\right)+\left(w/z\right)=xz\cdot\left(1/yz\right)+wy\cdot\left(1/yz\right)\). Por (2), \(\left(x/y\right)+\left(w/z\right)=\left(1/yz\right)\cdot xz+\left(1/yz\right)\cdot wy\). Por (5), \(\left(x/y\right)+\left(w/z\right)=\left(1/yz\right)\cdot \left(zx+wy\right)\). Por (2), \(\left(x/y\right)+\left(w/z\right)=\left(zx+wy\right)\cdot\left(1/yz\right)\). Y por, definicion \(\left(x/y\right)+\left(w/z\right)=\left(zx+wy\right)/yz\).
\begin{itemize}
\item \bf (p) \rm 
\end{itemize}
Veamos que \(x\neq0\Longrightarrow\left(1/x \right)\neq0\). Supongamos que \(x\neq0\Longrightarrow\left(1/x \right)=0\). Por (2) y ejercicio (b), \(x\cdot\left(1/x \right)=\left(1/x\right)\cdot x=0\). Pero por definicion de \(1/x\), si \(x\neq0\) entonces\(x\cdot\left(1/x\right)=1\). Luego \(1=0\), lo que es imposible. por tanto, \(x\neq0\Longrightarrow\left(1/x \right)=0\) es falso y \(x\neq0\Longrightarrow\left(1/x \right)\neq0\) es verdadero.
\begin{itemize}
\item \bf (q) \rm 
\end{itemize}
Veamos que \(1/\left(w/z\right)=z/w\) si \(w,z\neq0\). Por definicion \(\left(w/z\right)\cdot\left[1/\left(w/z\right)\right]=1\). Por ejercicio (n), \(\left(w/z\right)\cdot\left(z/w\right)=\left(wz\right)/\left(zw\right)\). Por (2), \(\left(w/z\right)\cdot\left(z/w\right)=\left(zw\right)/\left(zw\right)\). Por ejercicio (n), \(\left(w/z\right)\cdot\left(z/w\right)=\left(z\right)/\left(z\right)\cdot\left(w\right)/\left(w\right)\). Por ejercicio (j),  \(\left(w/z\right)\cdot\left(z/w\right)=1\cdot1\) Por (3), \(\left(w/z\right)\cdot\left(z/w\right)=1\). Por ser el inverso de \(\left(w/z\right)\) único, se tiene que \(1/\left(w/z\right)=z/w\).
\begin{itemize}
\item \bf (r) \rm 
\end{itemize}
Veamos que \(\left(x/y\right)/\left(w/z\right)=\left(xz\right)/\left(yw\right)\). Por definicion, \(\left(x/y\right)/\left(w/z\right)=\left(x/y\right)\cdot 1/\left(w/z\right)\). Por ejercicio (q),  \(\left(x/y\right)/\left(w/z\right)=\left(x/y\right)\cdot \left(z/w\right)\). Por ejercicio (n), \(\left(x/y\right)/\left(w/z\right)=\left(xz/yw\right)\)
\begin{itemize}
\item \bf (s) \rm 
\end{itemize}
Veamos que \(\left(ax\right)/y=a\left(x/y\right)\). Por definicion, \(\left(ax\right)/y=\left(ax\right)\cdot\left(1/y\right)\). Por (1), \(\left(ax\right)/y=a\left(x\cdot\left(1/y\right)\right)\). Por definicion,  \(\left(ax\right)/y=a\left(x/y\right)\).
\begin{itemize}
\item \bf (t) \rm 
\end{itemize}
Veamos que \(\left(-x\right)/y=x/\left(-y\right)=-\left(x/y\right)\). Por ejercicio (f), \(\left(-x\right)/y=\left(\left(-1\right)x\right)/y=(-1)\left(x\cdot\left(1/y\right)\right)\). Por (1),  \(\left(-x\right)/y=(-1)\left(x/y\right)\). Luego,   \(\left(-x\right)/y=-\left(x/y\right)\). Por otro lado, ejercicio (o) implica \(x/y+x/(-y)=\left(x\cdot\left(-1\right)y+yx\right)/\left(y\cdot\left(-1\right)y\right)\). Por (2), \(x/y+x/(-y)=\left(-xy+yx\right)/\left(-yy\right)\). Por (1), \(x/y+x/(-y)=\left(-xy+xy\right)/\left(-yy\right)\). Por (4), \(x/y+x/(-y)=0\cdot\left(\left(-yy\right)\right)\). Por ejercicio (b), \(x/y+x/(-y)=0\). Por (4), el opuesto de \(x/y\) es único, por tanto, \(-x/y=x/(-y)\).


Si \(C\) relación en \(A\)
\newline
Si \(C\) entonces \(A\)
% Uncomment the following two lines if you want to have a bibliography
%\bibliographystyle{alpha}
%\bibliography{document}
\end{document}
