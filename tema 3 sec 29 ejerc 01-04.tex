\documentclass{article}
% Uncomment the following line to allow the usage of graphics (.png, .jpg)
%\usepackage[pdftex]{graphicx9990}o0y
% Comment the following line to NOT allow the usage ofp0 umlauts

%\usepackage[utf8]{inputenc}
%\usepackage{amsmath}
%\usepackage{amssymb}

\newcommand{\vect}[1]{\boldsymbol{#1}}
% Start the document00
\begin{document}
\section{Tema 3 Sección 29 Ejercicio 1}
Veamos que el conjunto de números racionales $\mathbb{Q}$ no es localmente compacto.
Si $x\in \mathbb{Q}$ y $x\in (p,q)$ con $p,q\in\mathbb{R}-\mathbb{Q}$, el subespacio $[p,q]\cap \mathbb{Q}$ de $\mathbb{Q}$ no es compacto. Se puede cubrir por el cubrimiento infinito $\mathcal{A}=\{(p+\epsilon/i,q-\epsilon/i)\}_{i\in \mathbb{N},\epsilon\in \mathbb{Q}}$. Pero no hay subrecubrimiento finito de elementos de $\mathcal{A}$ que cubra $[p,q]\cap \mathbb{Q}$, porque para algún $N\in \mathbb{N}$, el conjunto $\{q-\epsilon/i)\}_{N\leq i, i\in \mathbb{N},\epsilon\in \mathbb{Q}}$ no está contenido ninguna unión finita de los elementos de $\mathcal{A}$.
\section{Tema 3 Sección 29 Ejercicio 2}
Sea $X_\alpha$ una familia indexada de espacios no vacíos.
\begin{itemize}
\item \bf (a) \rm Veamos que si $ \prod X_\alpha$ es localmente compacto, cada $X_\alpha$ es localmente compacto y $X_\alpha$ son compactos para un número finito de $\alpha$
\end{itemize}
En la topología producto los abiertos de $\prod X_\alpha$ son del tipo $U =\prod U_\alpha$ donde $U_\alpha$ son abiertos de $X_\alpha$ para un número finito de $\alpha$ y $U_\alpha=X_\alpha$ para el resto. Por definición, existe un subespacio $C=\prod C_\alpha$ compacto, donde $C_\alpha=\pi_\alpha(C)$, tal que para cada $(x_\alpha)\in \prod X_\alpha$ existe un entorno $V$ de $(x_\alpha)$ contenido en $C$. Por definición de compacto, cada cubrimiento infinito $\mathcal{A}=\{U_\beta\}$ de $C$ por abiertos de $\prod X_\alpha$ admite un subrecubrimiento finito $ \{U_{\beta_1},U_{\beta_2},...,U_{\beta_n}\}$. Por tanto, $(x_\alpha)\in U_{\beta_i}=\prod U_{\alpha,\beta_i}$ para algún $i\leq n$. Como la proyección es una aplicación continua, el teorema 26.5 asegura que $C_\alpha$ es un subespacio compacto.   Por tanto cada cubrimiento de abiertos $\mathcal{A}_\alpha=\{U_{\alpha,\beta}\}$ de $C_\alpha$ admite un subrecubrimiento finito $\{U_{\alpha,\beta_1},U_{\alpha,\beta_2},...,U_{\alpha,\beta_n}\}$. Como $(x_\alpha)\in C$, se tiene que   $x_\alpha\in C_\alpha$. Por tanto,  $x_\alpha$ pertenece a alguno de los $U_{\alpha,\beta_i}$ contenido en $C_\alpha$. Esto se da para todo $x_\alpha\in X_\alpha$, luego $X_\alpha$ es localmente compacto para cada $\alpha$.

Además, como hay un subespacio compacto en $X_\alpha$, cada recubrimiento $\mathcal{A}_\alpha =\{U_{\alpha,\beta}\}$ de $C_\alpha$ tiene un subrecubrimiento finito que cubre a $C_\alpha$, se tiene que $\mathcal{A}_\gamma=\pi_\gamma\left(\pi_\alpha^{-1}\left(\mathcal{A}_\alpha\right)\right)$ es un recubrimiento infinito de $X_\gamma$, para el número infinito de los $\gamma$ tales que la proyección sobre los abiertos de $X$ es $\pi_\gamma\left(\prod U_\alpha\right)=X_\gamma$. Y además los $\pi_\gamma\left(\pi_\alpha^{-1}\left(C_{\alpha,\beta_i}\right)\right)$ son los elementos del subrecubrimiento finito $\mathcal{A}_\gamma$ para $i\leq n$
\begin{itemize}
\item \bf (b) \rm Demostremos el recíproco de (a) suponiendo cierto el teorema de Tychonoff.
\end{itemize}
Suponiendo que el producto infinito de espacios compactos es compacto, veamos que si cada $X_\alpha$ es localmente compacto y los $X_\alpha$ no son compactos salvo para un número finito de $\alpha$, entonces $\prod_\alpha X_\alpha$ es localmente compacto.
Sean $C_\alpha$ los subespacios compactos de $X_\alpha$ con $C_\alpha \neq X_\alpha$ para un número finito de $\alpha$ y $C_\alpha = X_\alpha$ para el resto. Sean $U_\alpha$ los entornos de $x_\alpha$ contenidos en $C_\alpha$ donde $U_\alpha=X_\alpha$ para los $\alpha$ tales que $C_\alpha = X_\alpha$. Entonces, por teorema de Tychonoff, $C=\prod_\alpha C_\alpha$ es subespacio compacto. Por otro lado $U_\alpha\subset C_\alpha$ implica $\prod_\alpha U_\alpha \subset \prod_\alpha C_\alpha$ (ver ejercicio 5.3(a)). Por tanto, para cada $(x_\alpha)\in \prod_\alpha X_\alpha$ existe un subespacio compacto $C$ que contienen un entorno $\prod_\alpha U_\alpha$ de  él.
\section{Tema 3 Sección 29 Ejercicio 3}
Sea $X$ un subespacio localmente compacto. Si $f:X\rightarrow Y$ es continua, veamos si $f(X)$ es localmente compacto o no. Sea $C$ el subespacio compacto que contiene un entorno $U$, del $x\in X$ dado. Por el teorema 26.5, la imagen de un espacio compacto por una función continua es compacta. Por tanto $f(C)$ es subespacio compacto por abiertos de $Y$. Como $x\in U\subset C$ implica $f(x)\in f(U)\subset f(C)$, para cada $f(x)\in f(X)$. Pero en general, $f(U)$ no es abierto de $Y$.
En el caso de una aplicación abierta,  $f(U)$ es un abierto del espacio $Y$ si $U$ es abierto del espacio $X$. Por tanto, $f(U)$ es un abierto de $f(X)$ como subespacio de $Y$ y es un entorno de $f(x)$ si $x\in U\subset C$. Luego, $f(X)$ también es localmente compacto si $f$ es aplicación abierta.
\section{Tema 3 Sección 29 Ejercicio 4}
Veamos que $[0,1]^\omega$ no es localmente compacto con la topología uniforme. Como por teorema 20.4 la topología producto sobre $\mathbb{R}^\omega$ es mas fina  que la topología uniforme sobre $\mathbb{R}^\omega$, la topología producto sobre $[0,1]^\omega$ es mas fina que la topología uniforme sobre $[0,1]^\omega$. La topología uniforme sobre $[0,1]^\omega$ es Hausdorff, por teorema 19.3. Entonces los subespacios compactos de $[0,1]^\omega$ son cerrados por teorema 26.3. Recordemos que los abiertos de $\mathbb{R}^\omega$ en la topología uniforme son las bolas centradas en un punto $\vect{x}$ del tipo $B_{\overline{\rho}}(\vect{x},\epsilon)$, con $\overline{\rho}(\vect{y},\vect{x})=\sup\{\overline{d}(y_i,x_i)|i \in \mathbb{Z}_+\}$, y con $\overline{d}(y_i,x_i)=\min\{d(x_i,y_i),1\}$. Supongamos que para cada $\vect{x}\in[0,1]^\omega$ existe un subespacio compacto $C$ tal que contiene a $B_{\overline{\rho}}(\vect{x},\epsilon)\cap[0,1]^\omega$ para algún $\epsilon>0$. Entonces $C$ es del tipo $\overline{B}_{\overline{\rho}}(\vect{x},\delta)\cap [0,1]^\omega$ con $\delta\geq \epsilon$. Entonces se tiene el cubrimiento de $C=\overline{B}_{\overline{\rho}}(\vect{x},\delta)\cap [0,1]^\omega$ dado por $\mathcal{A}=\{B_{\overline{\rho}}(\vect{a}_i,2\delta))\cap[0,1]^\omega\}_{i\in \mathbb{Z}_+}\cup\{U\}$ donde $\vect{a}_i=(a_{ji})$ son los puntos tales que $a_{ji}=x_j$ para todo $i\neq j$ y $a_{jj}=x_j+1$ y $U=C-\{a_i\}_{i\in \mathbb{Z}_+}$. Pero $\mathcal{A}$ no tiene un subrecubrimiento finito que cubra $C$, luego los subespacios del tipo $C=\overline{B}_{\overline{\rho}}(\vect{x},\delta)\cap [0,1]^\omega$ no pueden ser compactos. Por tanto, $C$ es del tipo $[ x_1-\delta, x_1+\delta]\times [ x_2-\delta, x_2+\delta]\times ... [ x_{n}-\delta, x_n+\delta]\times \{x_{n+1}\}\times \{x_{n+2}\}\times...$. Pero en este caso no existe entorno de $\vect{x}$ tal que $B_{\overline{\rho}}(\vect{x},\epsilon)\cap[0,1]^\omega\subset C$. Además por el teorema 28.2, $C$ no puede ser compacto porque no es compacto por punto límite (siempre hay un conjunto infinito de puntos que no convergen). Luego, $[0,1]^{\omega}$ no es localmente compacto.
\end{document}
