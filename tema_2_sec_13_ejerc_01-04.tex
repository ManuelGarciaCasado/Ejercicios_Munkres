\documentclass{article}
% Uncomment the following line to allow the usage of graphics (.png, .jpg)
%\usepackage[pdftex]{graphicx9990}o0y
% Comment the following line to NOT allow the usage ofp0 umlauts


\newcommand{\vect}[1]{\boldsymbol{#1}}
% Start the document00
\begin{document}
\section{Tema 2 Sección 13 Ejercicio 1}
Sea $X$ un espacio topológico y $A$ un subconjunto de $X$. Supongamos que para cada $x\in A$ existe un abierto $U$ tal que $x\in U\subset A$. Veamos que $A$ es abierto en $X$. Sea $U_x$ el abieto que contiene a cierto $x$ de $A$ y tal que $U_x\subset A$. Como $U_x\subset A$ para todo $x\in A$ se tiene que $ \cup_{x\in A} U_x\subset A$. Pero  $A=\cup_{x\in A}\{x\}\subset \cup_{x\in A}U_x$. Por tanto $A=\cup_{x\in A}U_{x}$. Es decir, $A$ es la unión de abiertos, es un abierto por definición de topología.
\section{Tema 2 Sección 13 Ejercicio 2}
Sean las topología definidas por $X=\{a,b,c\}$ como $\mathcal{T}_1=\{\varnothing,X\}$, $\mathcal{T}_2=\{\varnothing,X,\{a\},\{a,b\}\}$, $\mathcal{T}_3=\{\varnothing,X,\{b\},\{a,b\},\{b,c\}\}$, $\mathcal{T}_4=\{\varnothing,X,\{b\}\}$, $\mathcal{T}_5=\{\varnothing,X,\{a\},\{b,c\}\}$, $\mathcal{T}_6=\{\varnothing,X,\{b\},\{a,b\},\{a,b\},\{c\}\}$, $\mathcal{T}_7=\{\varnothing,X,\{a,b\}\}$, $\mathcal{T}_8=\{\varnothing,X,\{a,b\},\{a\},\{b\}\}$, $\mathcal{T}_9=\{\varnothing,X,\{a,b\},\{a,c\},\{c,b\},\{a\},\{b\},\{c\}\}$. Entonces es obvio que $\mathcal{T}_1\subset \mathcal{T}_i$ para $i=\{2,3,4,5,6,7,8,9\}$ y $\mathcal{T}_j\subset \mathcal{T}_9$ para $j=\{1,2,3,4,5,6,7,8\}$. También $\mathcal{T}_1\subset \mathcal{T}_2\subset \mathcal{T}_3\subset \mathcal{T}_6 \subset \mathcal{T}_9$, $\mathcal{T}_7\subset \mathcal{T}_8 \subset \mathcal{T}_9$, y $\mathcal{T}_2\subset \mathcal{T}_8 $, $\mathcal{T}_4\subset \mathcal{T}_8 $, $\mathcal{T}_4\subset \mathcal{T}_3 $ y $\mathcal{T}_7\subset \mathcal{T}_2$. De aquí de deduce que $\mathcal{T}_{9}$ es la topología mas fina.
\section{Tema 2 Sección 13 Ejercicio 3}
Sea $X$ un conjunto y $\mathcal{T}_{c}$ un conjunto de subconjuntos $U$ de $X$ tales que $X-U$ es numerable o es todo $X$. Veamos que es una topología. Si $U=X$, $X-X=\varnothing$ es finito, por tanto, numerable. Luego $X\in \mathcal{T}_{c}$. Si $U=\varnothing$ entonces $X-\varnothing=X$. Luego $\varnothing\in \mathcal{T}_{c}$. Sea $\{U_\alpha\}$ una familia indexada de subconjuntos de $X$ tales que $U_\alpha\in \mathcal{T}_{c}$. Ahora queramos ver que $\cup U_\alpha\in \mathcal{T}_c$. Entonces $X-\cup U_\alpha=\cap (X-U_\alpha)$. Como los $X-U_\alpha$ son numerables o son todo $X$, la intersección de conjuntos numerables es numerables, por ser subconjunto de conjunto numerable (corolario 7.3). Por tanto $X-\cup U_\alpha$ es numerable y $\cup U_\alpha\in \mathcal{T}_c$. Ahora queremos ver que $\cap_{i=1}^{n} U_i\in \mathcal{T}_c$. Entonces $X-\cap_{i=1}^{n} U_i=\cup_{i=1}^{n}(X-U_i)$. Como los $X-U_i$ para $i=\{1,2...,n\}$ son numerables o son todo $X$, la unión numerable de conjuntos numerables es numerables (corolario 7.5). Por tanto $X-\cup_{i=1}^{n} U_i$ es numerable y $\cup_{i=1}^{n} U_i\in \mathcal{T}_c$. Por tanto, $\mathcal{T}_c$ es una topología.
Sea la colección $\mathcal{T}_{\infty}=\{U|X-U\text{ es infinita o vacía o todo} X\}$. Si $U=X$, $X-X=\varnothing$ y, por tanto, $X\in \mathcal{T}_{\infty}$. Si $U=\varnothing$, $X-\varnothing=X$ y, por tanto, $\varnothing\in \mathcal{T}_{\infty}$. Sean $X-U_\alpha$ conjuntos infinitos para todo $\alpha \in J$ donde $J$ es cierto conjunto indexante. Entonces existen funciones inyectivas $f_\alpha: \mathbb{Z}_{+}\rightarrow U_\alpha$. Entonces $X-\cup_{\alpha\in J} U_\alpha=\cap_{\alpha\in J} (X-U_\alpha)$. Puesto que la intersección de conjuntos infinitos no es necesariamente infinita (p.e., la intersección del conjunto de números pares con el conjunto de numeros impares es vacío y, en consecuencia, finito), no se puede afirmar que $X-\cup_{\alpha\in J} U_\alpha$ sea infinito y, por tanto, no se puede afirmar que $\mathcal{T}_{\infty}$ sea una topología.

\section{Tema 2 Sección 13 Ejercicio 4}
\begin{itemize}
\item \bf (a) \rm
\end{itemize}
Sea $\{\mathcal{T}_{\alpha}\}$ una familia de topologías sobre $X$. Veamos que $\cap \mathcal{T}_{\alpha}$ es una topología sobre $X$. Como $X\in \mathcal{T}_{\alpha}$ para todo $\alpha$, $X\in \cap\mathcal{T}_{\alpha}$. Como $\varnothing\in \mathcal{T}_{\alpha}$ para todo $\alpha$, $\varnothing\in \cap\mathcal{T}_{\alpha}$. Se tiene que $\cap\mathcal{T}_{\alpha}=\{U| U\in {T}_{\alpha} \text{ para todo }\alpha\}$. Si los $U_\beta$ cumplen que $U_\beta \in \mathcal{T}_\alpha$ para todo $\alpha$, el conjunto formado por cualquier unión de los $\beta$ también partenece a $\cap\mathcal{T}_{\alpha}$, puesto que si $U_\beta$ pertenece a cada una de las topologías $\mathcal{T}_{\alpha}$, por la definición de topología, las uniones $\cup_\beta U_\beta$ también pertenecerán a cada una de las topologías $\mathcal{T}_{\alpha}$. Del mismo modo, cualquier intersección finita de los $\beta$ también partenece a $\cap\mathcal{T}_{\alpha}$, puesto que si $U_\beta$ pertenece a cada una de las topologías $\mathcal{T}_{\alpha}$, por la definición de topología, las intersecciones $\cap_{\beta=1}^n U_\beta$ también pertenecerán a cada una de las topologías $\mathcal{T}_{\alpha}$. Por tanto, $\cup_\beta U_\beta\in \cap \mathcal{T}_{\alpha}$ y $\cap_{\beta=1}^n U_\beta\in \cap \mathcal{T}_{\alpha}$. Lo cual significa que $\cap \mathcal{T}_{\alpha}$ es una topología. El conjunto de subconjuntos $\cup \mathcal{T}_\alpha=\{U| U\in \mathcal{T}_\alpha\text{ para algún }\alpha\}$ no es una topología puesto que si $U\in \mathcal{T}_\alpha$ para algún $\alpha$ y $V\in \mathcal{T}_\alpha'$ para algún $\alpha'\neq\alpha$, el conjunto $U\cup V$ no tiene por qué pertenecer a $\mathcal{T}_\alpha$ y no tiene por qué pertenecer a $\mathcal{T}_\alpha'$.
\begin{itemize}
\item \bf (b) \rm
\end{itemize}
Sea $\{\mathcal{T}_{\alpha}\}$ una familia de topologías sobre $X$. Veamos que hay una única topología que contiene a cada una de las otras topologías $\mathcal{T}_{\alpha}$ y que es menor que cualquier otra topología que contiene a cada una de las otras topologías $\mathcal{T}_{\alpha}$; y que hay otra única topología que está contenida en cada una las topologías $\mathcal{T}_{\alpha}$ y que es mayor que cualquier otra topología que está contenida en cada una las topologías $\mathcal{T}_{\alpha}$. Primero veamos que hay una topología $\mathcal{T}$ tal que $\mathcal{T}_{\alpha}\subset \mathcal{T}$ para todo $\alpha$. Sea la colección $\mathcal{T}$ de subconjuntos $U$ de $X$ tales que $U$ es la unión de elementos de $\cup \mathcal{T}_\alpha$ o es la intersección finita de elementos de $\cup \mathcal{T}_\alpha$; entonces $\mathcal{T}_{\alpha}\subset\mathcal{T}$ para todo $\alpha$. Veamos que es una topología. Dado que $X\in \mathcal{T}_{\alpha}$ para todo $\alpha$ se tiene que $X\in \mathcal{T}$. Dado que $\varnothing\in \mathcal{T}_{\alpha}$ para todo $\alpha$ se tiene que $\varnothing\in \mathcal{T}$. La intersección finita de elementos de $\mathcal{T}$ pertenece a $\mathcal{T}$ ya que dicha intersección o es $\varnothing$ o  es un elemento de $\mathcal{T}_{\alpha}$ para algún $\alpha$ y como $\mathcal{T}_{\alpha}\subset \cup \mathcal{T}_{\alpha}$, dicha intersección pertenece a $\cup \mathcal{T}_{\alpha}$. Unión de elementos de $\mathcal{T}$ pertenece a $\mathcal{T}$ ya que  dicha unión es la unión de elementos de $ \cup \mathcal{T}_{\alpha}$ y, por definición, son elementos de $\mathcal{T}$. Supongamos que hubiera otra topología $\mathcal{T}'$ tal que $\mathcal{T}_{\alpha}\subset \mathcal{T}'$ para todo $\alpha$. Entonces $X\in \mathcal{T}'$, $\varnothing \in \mathcal{T}'$, la intersección finita de elementos de $\mathcal{T}'$ estaría en $\mathcal{T}'$ y la unión de elementos de $\mathcal{T}'$ estaría en $\mathcal{T}'$. por tanto, la unión finita de elementos de $\mathcal{T}_{\alpha}$ para cualquier $\alpha$ estaría en $\mathcal{T}'$ y la unión finita de elementos de $\cup\mathcal{T}_{\alpha}$ estaría en $\mathcal{T}'$. Pero esto es la definición de $\mathcal{T}$. Luego $\mathcal{T}=\mathcal{T}'$. Ahora veamos que hay una única topología que está contenida en cada una de las $\mathcal{T}_{\alpha}$ y que es mayor que cualquier otra topología que cumple esa condición. Se tiene que la topología $\cap \mathcal{T}_{\alpha}$ cumple $\cap \mathcal{T}_{\alpha}\subset \mathcal{T}_{\alpha}$ para todo $\alpha$. Si hubiera una topología $\mathcal{T}''$ mayor que $\cap \mathcal{T}_{\alpha}$ se tendría que $\mathcal{T}_{\alpha}\subset \mathcal{T}''$ para algún $\alpha$ por tanto, $\cap \mathcal{T}_{\alpha}$ es única.
\begin{itemize}
\item \bf (c) \rm
\end{itemize}
Sea $X=\{a,b,c\}$ y sean las topologías
 $\mathcal{T}_1=\{\varnothing,X,\{a\},\{a,b\}\}$ y  $\mathcal{T}_2=\{\varnothing,X,\{a\},\{b,c\}\}$. Por tanto, si se define $\mathcal{T}=\{\varnothing,X,\{a\}\}$ entonces $\mathcal{T}\subset\mathcal{T}_1$ y $\mathcal{T}\subset\mathcal{T}_2$. Y si se define $\mathcal{T}'=\{\varnothing,X,\{a\},\{a,b\},\{b,c\},\{b\}\}$ entonces $\mathcal{T}_1\subset\mathcal{T}'$ y $\mathcal{T}_2\subset\mathcal{T}'$.



\end{document}
