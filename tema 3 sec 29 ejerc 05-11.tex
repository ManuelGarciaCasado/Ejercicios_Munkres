\documentclass{article}
% Uncomment the following line to allow the usage of graphics (.png, .jpg)
%\usepackage[pdftex]{graphicx9990}o0y
% Comment the following line to NOT allow the usage ofp0 umlauts

%\usepackage[utf8]{inputenc}
%\usepackage{amsmath}
%\usepackage{amssymb}

\newcommand{\vect}[1]{\boldsymbol{#1}}
% Start the document00
\begin{document}
\section{Tema 3 Sección 29 Ejercicio 5}
Sea $f:X_1\rightarrow X_2$ un homeomorfismo entre espacios de Hausdorff localmente compactos. Veamos cómo se extiende $f$ a un homeomorfismo entre las respectivas compactificaciones por punto. Sean $Y_1$ e $Y_2$ las respectivas compactificaciones por un punto de $X_1$ y $X_2$. Entonces se tiene $Y_1-X_1$ e $Y_2-X_2$ constan de un único punto. Sea $g:Y_1\rightarrow Y_2$ el homeomofismo. Entonces $g(Y_1-X_1)=Y_2-X_2$, ya que por el teorema 29.1 $Y_1-X_1$  e $Y_2-X_2$ son conjuntos con un único elemento, y $g(X_1)=f(X_1)=X_2$
\section{Tema 3 Sección 29 Ejercicio 6}
Veamos que la compactificación por un punto de $\mathbb{R}$ es homeomorfa a la circunferencia $S^1$. Se tiene que la función $f:(0,1)\rightarrow \mathbb{R}$ dada por $f(x)=(x-1)/[1-(x-1)^2]$ es un homeomorfismo. Y la función $g:[0,1)\rightarrow S^1$ definida por $g(t)=(\cos(2\pi t),\sin(2\pi t))$ también es un homeomorfismo. Por tanto, $g \circ f^{-1} :\mathbb{R}\rightarrow S^1-\{\vect{a}\}$ con $\vect{a}=(1,0)\in S^1$, también es un homeomorfismo. Como $\mathbb{R}$ es Hausforff (por ser metrizable) y es localmente compacto, el teorema 29.1 garantiza que existe un espacio $\mathbb{R}\cup \{\infty\}$ compacto y de Hausdorff. Por tanto, $S^1$ es compacto y Hausdorff. Por tanto existe un homeomorfismo $h:\mathbb{R}\cup\{\infty\}\rightarrow S^1$ definido por $h(\infty)=\vect{a}$ y $h(\mathbb{R})=g(f(\mathbb{R}))=S^1-\{\vect{a}\}$.
\section{Tema 3 Sección 29 Ejercicio 7}
Veamos que la compactificación por un punto de $S_\Omega$ es homeomorfa a $\overline{S}_\Omega$. Se tiene que $S_\Omega$ es el conjunto bien ordenado minimal. Además, $\Omega$ es  punto límite de $S_\Omega$ en la topología del orden y $\overline{S}_\Omega=S_\Omega\cup \{\Omega\}$. El espacio $S_\Omega$ no es compacto (ver Ejemplo 2 de la Sección 28), ni el espacio $\overline{S}_\Omega$ es metrizable (ver Ejemplo 3 de la Sección 28). Hay que demostrar que $\overline{S}_\Omega$ es Hausdorff y compacto. Se vió en ejercicio 17.10 que toda topología del orden es Hausdorff. Por tanto, $\overline{S}_\Omega$ es Hausdorff. Como $\overline{S}_\Omega$ tiene la propiedad del supremo y es cerrado (contiene a todos sus puntos límite), por el teorema 27.1, también es compacto. Luego, por definición, $\overline{S}_\Omega$ es la compactificación por un punto de $S_\Omega$, que es homeomorfo a si mismo.
\section{Tema 3 Sección 29 Ejercicio 8}
Veamos que la compactificación por un punto de $\mathbb{Z}_+$ es homeomorfa al subespacio $\{0\}\cup\{1/n|n\in\mathbb{Z}_+\}$ de $\mathbb{R}$. Se tiene que la función $f:(0,\infty)\rightarrow (0,\infty)$ dada por $f(x)=1/x$ es biyectiva. Por tanto, $g:\mathbb{Z}_+\rightarrow \{1/n|n\in\mathbb{Z}_+\}$ con $g(x)=f(x)$ es un homeomorfismo en la topología del orden. Se tiene que $0$ es punto límite de $\{1/n|n\in\mathbb{Z}_+\}$ en el espacio $\{0\}\cup\{1/n|n\in\mathbb{Z}_+\}$ con la topología del orden, ya que todo entorno de $0$ interseca a $\{1/n|n\in\mathbb{Z}_+\}$. Por ejercio 17.10, $\{0\}\cup\{1/n|n\in\mathbb{Z}_+\}$ es Hausdorff por ser tener la topología del orden. Además el teorema 27.1 asegúra que $\{0\}\cup\{1/n|n\in\mathbb{Z}_+\}$ es compacto, ya que contiene a todos sus puntos límite. Por tanto, como $f$ es un homeomorfismo, $\{0\}\cup\{1/n|n\in\mathbb{Z}_+\}$ es la compactificación por un punto de $\{1/n|n\in\mathbb{Z}_+\}$ si, y sólo si, $\mathbb{Z}_+\cup\{\infty\}$ es la compactificación por un punto de $\mathbb{Z}_+$.
\section{Tema 3 Sección 29 Ejercicio 9}
Veamos que si $G$ es un grupo topológico localmente compacto y $H$ es un subgrupo, entonces $G/H$ es localmento compacto. En los ejercicio s 7 y 5 de los ejercicios complementarios de Tema 2 vimos que $G$ es Hausdorff y que $p:G\rightarrow G/H$ es una aplicación cociente abierta. En el Ejercio 3 se ha visto que la imagen de un espacio localmente compacto por una aplicación abierta es localmente compacto. Por tanto, $p(G)=G/H$ es localmente compacto.
\section{Tema 3 Sección 29 Ejercicio 10}
Veamos que si $X$ es un espacio de Hausdorff que es localmente compacto en el punto $x$, para cada entorno $U$ de $x$ existe un entorno $V$ de $x$ tal que $\overline{V}$ es compacto y $\overline{V}\subset U$. Por definición existe un subespacio compacto $C$ que contiene a $x$. Por teorema 26.2, cada subespacio cerrado de un espacio compacto es compacto y por teorema 26.3, cada subespacio compacto de un espacio de Hausdorrf es cerrado. Por tanto, si $C$ es el espacio compacto que contiene a $x$, cualquier $\overline{V}$ tal que $\overline{V}\subset C$ es compacto. Entonces sea $V$ entorno de $x$ tal que $V\subsetneq U$. En este caso $\overline{V}\subset U$.
\section{Tema 3 Sección 29 Ejercicio 11}
\begin{itemize}
\item \bf (a) \rm Veamos que si $p:X\rightarrow Y$ es una aplicación cociente y $Z$ es un espacio localmente compacto y Hausdoff, entonces la aplicación $\pi= p\times i_Z: X\times Z\rightarrow Y\times Z$ es una aplicación cociente.
\end{itemize}
Hay que probar que $A$ es abierto de $Y\times Z$ si, y sólo si, $\pi^{-1}(A)$ es abierto en $X\times Z$. Supongamos que $\pi^{-1}(A)$ es un abierto que contiene a $x\times y$. Además, sean $U_1$ y $V$ abiertos tales que $\overline{V}$ es compacto, $x\times y\in U_1\times V$ y $U_1\times \overline{V}\subset \pi^{-1}(A)$. Suponiendo $U_i\times \overline{V}\subset \pi^{-1}(A)$ utilicemos el lema del tubo para hayar un $U_{i+1}$ tal que $p^{-1}(p(U_i))\subset U_{i+1}$ y tal que $U_{i+1}\times \overline{V}\subset \pi^{-1}(A)$. Sea entonces $U=\bigcup  U_i$, veamos que $U\times V$ es un entorno de $x\times y$ que es saturado y está contenido en $\pi^{-1}(A)$. Primero hay que demostrar que $\pi^{-1}(\{a\times b\})\subset U\times V$ si $\pi^{-1}(\{a\times b\})\cap U\times V \neq \varnothing$, es decir $\pi^{-1}(\pi(U\times V))=U\times V$. Por inducción tenemos que $U_i\subset p^{-1}(p(U_i))\subset U_{i+1}$ para todo $i\in \mathbb{Z}_+$. Por tanto, $\left(\bigcup_{i\in \mathbb{Z}_+}U_i\right)\times V\subset \left(\bigcup_{i\in \mathbb{Z}_+}p^{-1}(p(U_i ))\right)\times V\subset \left(\bigcup_{i\in \mathbb{Z}_+}U_{i+1}\right)\times V\subset \pi^{-1}(A)$. Como $\bigcup_{i\in \mathbb{Z}_+}p^{-1}(p(U_i ))=p^{-1}(p(\bigcup_{i\in \mathbb{Z}_+}U_i ))$ resulta que $U\times V\subset p^{-1}(p(U ))\times V\subset U\times V$. Luego $U\times V =p^{-1}(p(U ))\times V$. Como $V=i^{-1}_Z (i_Z(V))$ entonces se tiene que $U\times V \subset \pi^{-1}(\pi(U\times V))\subset U\times V\subset \pi^{-1}(A)$ y que $U\times V =\pi^{-1}(\pi(U\times V))$. Luego, $U\times V$ es un conjunto saturado tal que $U\times V \subset \pi^{-1}(A)$. Por tanto, como $\pi^{-1}(A)$ es la unión de los entornos de sus elementos, y estos entornos son saturados, $\pi^{-1}(A)$ también es saturado. Como $\pi$ asocia conjuntos abiertos saturados $\pi^{-1}(A)$ con conjuntos abiertos $A$, es aplicación cociente.
\begin{itemize}
\item \bf (b) \rm Sean $p:A\rightarrow B$ y $q:C\rightarrow D$ aplicaciones cocientes. Si $B$ y $C$  son espacios localente compactos y Hausdoff, veamos que la aplicación $p\times q: A\times C\rightarrow B\times D$ es una aplicación cociente.
\end{itemize}
Aplicando el lema del apartado (a), tenemos que las aplicaciones $p\times i_C:A\times C\rightarrow B\times C$ y $i_B\times q: B\times C\rightarrow B\times D$ son aplicaciones cociente. Entonces $p\times q = (i_B\times q) \circ (p\times i_C)$ es aplicación cociente por ser la composición de dos alpicaciones cociente. Esto resulta de la ecuación.
\begin{eqnarray}
(p\times q)^{-1}(U)= (p\times i_C)^{-1}((i_B\times q)^{-1}(U)) \nonumber
\end{eqnarray}
Ya que $U$ es abierto de $B\times D$ si, y sólo si, $(i_B\times q)^{-1}(U)$ es abierto de $B\times C$ si, y sólo si,$(p\times i_C)^{-1}((i_B\times q)^{-1}(U))$ es abierto de $A\times C$.
\end{document}
