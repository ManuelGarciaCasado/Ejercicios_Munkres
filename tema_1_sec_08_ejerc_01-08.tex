\documentclass{article}
% Uncomment the following line to allow the usage of graphics (.png, .jpg)
%\usepackage[pdftex]{graphicx}
% Comment the following line to NOT allow the usage of umlauts


\newcommand{\vect}[1]{\boldsymbol{#1}}
% Start the document
\begin{document}

% Create a new 1st level heading
\section{Tema 1 Sección 8 Ejercicio 1}
Sea la $(b_1,b_2,b_3,...)$ una sucesión infinita de números reales. La suma $\sum^n_{k=1}b_k$ se define como
\begin{eqnarray}
\sum^n_{k=1}b_k=b_1 & \text{ si }n=1 \nonumber\\
\sum^n_{k=1}b_k= (\sum^{n-1}_{k=1}b_k)+ b_n & \text{ si }n>1 \nonumber
\end{eqnarray}
Si $A=\mathbb{R}$, veamos cómo definir $\rho$ para que aplicar el teorema 8.4. Sea $a_0=b_1$ y si $\rho(f)=b_{m+1} +f(m)$ donde $f:\mathbb{Z}_{+} \rightarrow \mathbb{R}$ Sea 
\begin{eqnarray}
h(1)=b_1 & \text{ si }n=1 \nonumber\\
h(n)=\rho(h|\{1,2,...,n-1\}) & \text{ si }n>1 \nonumber
\end{eqnarray}
Entonces, si se define  $h:\mathbb{Z}_{+} \rightarrow \mathbb{R}$ como $h(n)=\sum^n_{k=1}b_k$, se tiene para $n>1$ que 
\begin{eqnarray}
\sum^n_{k=1}b_k=\rho(h|\{1,2,...,n-1\}) & \nonumber\\
=b_n+h(n-1) & \nonumber\\
=b_n+\sum^n_{k=1}b_k & \nonumber
\end{eqnarray}
Por tanto, esta definición cumple con los requerimientos del teorema 8.4.
\section{Tema 1 Sección 8 Ejercicio 2}
Sea la $(b_1,b_2,b_3,...)$ una sucesión infinita de números reales. El producto $\Pi^n_{k=1}b_k$ se define como
\begin{eqnarray}
\Pi^n_{k=1}b_k=b_1 & \text{ si }n=1 \nonumber\\
\Pi^n_{k=1}b_k= (\Pi^{n-1}_{k=1}b_k)\cdot b_n & \text{ si }n>1 \nonumber
\end{eqnarray}
Si $A=\mathbb{R}$, veamos cómo definir $\rho$ para que aplicar el teorema 8.4. Sea $a_0=b_1$ y si $\rho(f)=f(m)\cdot b_{m+1}$ donde $f:\mathbb{Z}_{+} \rightarrow \mathbb{R}$ Sea 
\begin{eqnarray}
h(1)=b_1 & \text{ si }n=1 \nonumber\\
h(n)=\rho(h|\{1,2,...,n-1\}) & \text{ si }n>1 \nonumber
\end{eqnarray}
Entonces, si se define  $h:\mathbb{Z}_{+} \rightarrow \mathbb{R}$ como $h(n)=\Pi^n_{k=1}b_k$, se tiene para $n>1$ que 
\begin{eqnarray}
\sum^n_{k=1}b_k=\rho(h|\{1,2,...,n-1\}) & \nonumber\\
=h(n-1)\cdot b_n & \nonumber\\
=(\Pi^n_{k=1}b_k)\cdot b_n & \nonumber
\end{eqnarray}
Por tanto, esta definición cumple con los requerimientos del teorema 8.4.
\section{Tema 1 Sección 8 Ejercicio 3}
Si $b_n=a$ para todo elemento de la sucesión infinita, $(b_1,b_2,...)$ se tiene que $h(n)=\Pi^n_{k=1}b_k=\Pi^n_{k=1}a$ por tanto, $\rho(f)= f(m)\cdot a$. 
\begin{eqnarray}
h(1)=a & \text{ si }n=1 \nonumber\\
h(n)=\rho(h|\{1,2,...,n-1\}) & \text{ si }n>1 \nonumber
\end{eqnarray}
Entonces 
\begin{eqnarray}
\Pi^n_{k=1}a=a & \text{ si }n=1 \nonumber\\
\Pi^n_{k=1}a=(\Pi^{n-1}_{k=1}a)\cdot a & \text{ si }n>1 \nonumber
\end{eqnarray}
Entonces, si se renombra a $\Pi^{n}_{k=1}a$ como $a^n$, se tiene que 
\begin{eqnarray}
a^1=a & \text{ si }n=1 \nonumber\\
a^n=a^{n-1}\cdot a  & \text{ si }n>1 \nonumber
\end{eqnarray}
Ahora, si $b_n=n$ para todo elemento de la sucesión infinita, $(b_1,b_2,...)$ se tiene que $h(n)=\Pi^n_{k=1}b_k=\Pi^n_{k=1}k$ por tanto, $\rho(f)= f(m)\cdot (m+1)$. 
\begin{eqnarray}
h(1)=1 & \text{ si }n=1 \nonumber\\
h(n)=\rho(h|\{1,2,...,n-1\}) & \text{ si }n>1 \nonumber
\end{eqnarray}
Entonces 
\begin{eqnarray}
\Pi^n_{k=1}k=1 & \text{ si }n=1 \nonumber\\
\Pi^n_{k=1}k=(\Pi^{n-1}_{k=1}k)\cdot n & \text{ si }n>1 \nonumber
\end{eqnarray}
Entonces, si se renombra a $\Pi^{n}_{k=1}a$ como $n!$, se tiene que 
\begin{eqnarray}
1!=1 & \text{ si }n=1 \nonumber\\
n!=(n-1)!\cdot n  & \text{ si }n>1 \nonumber
\end{eqnarray}
\section{Tema 1 Sección 8 Ejercicio 4}
Sean los números de Fibonacci definidos como
\begin{eqnarray}
\lambda_1=1 & \nonumber\\
\lambda_2=1 & \nonumber\\
\lambda_{n}=\lambda_{n-1}+\lambda_{n-2} & \text{ para }n>2\nonumber
\end{eqnarray}

Veamos que cumple las condiciones del teorema 8.4 si \begin{eqnarray}
\rho(f)=
\begin{cases}
f(m)+f(m-1) & \text{ si }m>1\nonumber\\
1 & \text{ si }m=1\nonumber
\end{cases}
\end{eqnarray}
 y $h:\mathbb{Z}_{+}\rightarrow \mathbb{Z}_{+}$ se tiene que 
\begin{eqnarray}
h(1)=1 & \text{ si }n=1\nonumber\\
h(n)=\rho(h|\{1,2,...,n-1\}) &
\text{ si }n>1 \nonumber
\end{eqnarray}
y por tanto
\begin{eqnarray}
&h(1)=1 \text{ si }n=1& \nonumber\\
&h(n)=\begin{cases}
h(n-1)+h(n-2) & \text{ si }n-1>1\nonumber\\
1 & \text{ si }n-1=1\nonumber
\end{cases}
\end{eqnarray}
Entonces se cumplen las condiciones del teorema 8.4.
\section{Tema 1 Sección 8 Ejercicio 5}
Sea la función $h:\mathbb{Z}_{+}\rightarrow \mathbb{R}_{+}$
\begin{eqnarray}
h(1)=3 & \text{ si }i=1\nonumber\\
h(i)=\sqrt{h(i-1)+1} &
\text{ si }i>1 \nonumber
\end{eqnarray}
Supongamos que hay una $g:\mathbb{Z}_{+}\rightarrow \mathbb{R}_{+}$ definido por
\begin{eqnarray}
g(1)=3 & \text{ si }i=1\nonumber\\
g(i)=\sqrt{g(i-1)+1} &
\text{ si }i>1 \nonumber
\end{eqnarray}
tal que existe un mínimo elemento $i\in \mathbb{Z}_{+}$ para el cual $g(i)\neq f(i)$. Ese $i$ no es 1, ya que $g(1)=3=f(1)\Rightarrow g(1)=f(1)$. Entonces $i>1$ lo cual implica que $\sqrt{g(i-1)+1}\neq \sqrt{f(i-1)+1} \Rightarrow  g(i-1)+1\neq f(i-1)+1$ por tanto, $g(i-1)\neq f(i-1)$. Pero $i-1<i$ contradice la suposición de $i$ es el elemento mínimo tal que $g(i)\neq f(i)$. Luego $f$ es única.
\section{Tema 1 Sección 8 Ejercicio 6}
\begin{itemize}
\item (a)
\end{itemize}
Sea la función $h:\mathbb{Z}_{+}\rightarrow \mathbb{R}_{+}$
\begin{eqnarray}
h(1)=3 & \text{ si }i=1\nonumber\\
h(i)=\sqrt{h(i-1)-1} &
\text{ si }i>1 \nonumber
\end{eqnarray}
Ésta definición se puede escribir como $h(i)=\rho(h|\{1,2,...,i-1\})$ para $i>1$ donde $\rho(f)=\sqrt{f(m)-1}$ es una función de $\rho:\mathbb{R}_{+}\rightarrow \mathbb{R}_{+}$ que asigna un valor $\rho(f)\in\mathbb{R}_{+}$ a cada $f\in\mathbb{R}_{+}$. Además, la formula se aplica para una sección de $\mathbb{Z}_{+}$.A parte, $h(1)\in \mathbb{R}_{+}$. Por tanto, cumple las condiciones del teorema.
\begin{itemize}
\item (b)
\end{itemize}
Sea la función $h:\mathbb{Z}_{+}\rightarrow \mathbb{R}_{+}$
\begin{eqnarray}
&h(1)=3  \text{ si }i=1&\nonumber\\
&h(i)=\begin{cases}\sqrt{h(i-1)-1} &
\text{ si }h(i-1)> 1\text{ y } i>1 \nonumber\\
5 &
\text{ si }h(i-1)\leq 1 \text{ y } i>1 \nonumber
\end{cases}
\end{eqnarray}
Esta definición cumple las condiciones del teorema, pues hay una $\rho:\mathbb{R}_{+}\rightarrow \mathbb{R}_{+}$ dada por 
\begin{eqnarray}
\rho(f)=\begin{cases}\sqrt{f(m)-1} &
\text{ si }f(m)> 1\nonumber\\
5 &
\text{ si }f(m)\leq 1 \nonumber
\end{cases}
\end{eqnarray}
Veamos que la función $h$ es única. Supongamos que hay otra función $g:\mathbb{Z}_{+}\rightarrow \mathbb{R}_{+}$
\begin{eqnarray}
&g(1)=3  \text{ si }i=1&\nonumber\\
&g(i)=\begin{cases}\sqrt{g(i-1)-1} &
\text{ si }g(i-1)> 1\text{ y } i>1 \nonumber\\
5 &
\text{ si }g(i-1)\leq 1 \text{ y } i>1 \nonumber
\end{cases}
\end{eqnarray}
Supongamos también que existe un mínimo elemento $i\in \mathbb{Z}_{+}$ para el cual $g(i)\neq h(i)$. Ese $i$ no es 1, ya que  $g(1)=3=h(1)\Rightarrow g(1)=h(1)$. Entonces cuando es $i>1$ y $h(i)\neq g(i)$ se tiene que: a) $h(i)\neq g(i)$ y $g(i)\neq 5 = h(i)$; b) $h(i)\neq g(i)$ y $g(i)= 5 \neq h(i)$ y c) $5\neq f(i)\neq g(i)\neq 5$. En caso (a), $g(i)=\sqrt{g(i-1)-1}\neq 5\Rightarrow 1<g(i-1)\neq 26$ y $h(i)=5\Rightarrow h(i-1)\leq 1$, por tanto $h(i-1)\neq g(i-1)$. En caso  (b) se tiene el mismo resultado que en (a) pero cambiando $g$ por $h$. En caso (c), $\sqrt{g(i-1)-1}\neq \sqrt{h(i-1)-1}\Rightarrow g(i-1)\neq h(i-1)$. Por lo tanto, en todos los casos, $\sqrt{g(i)}\neq \sqrt{h(i)}\Rightarrow g(i-1)\neq h(i-1)$  lo cual  contradice la suposición de $i$ es el elemento menor. Luego $h$ es única.
\section{Tema 1 Sección 8 Ejercicio 7}
Veamos como se demuestra el teorema 8.4. Entonces veamos que hay una unica función $h:\mathbb{Z}_{+}\rightarrow A$ tal que
\begin{eqnarray}
h(1)=a & \text{ si }n=1 \nonumber\\
h(n)=\rho(h|\{1,2,...,n-1\}) & \text{ si }n>1 \nonumber
\end{eqnarray}
para cada $i$. Supongamos que hay otra función $g:\mathbb{Z}_{+}\rightarrow A$ que cumple
\begin{eqnarray}
g(1)=a & \text{ si }n=1 \nonumber\\
g(n)=\rho(g|\{1,2,...,n-1\}) & \text{ si }n>1 \nonumber
\end{eqnarray}
y sea $i$ el menor elemento tal que $h(i)\neq g(i)$. Entonces $i$ no es 1, ya que $h(1)=a=g(1)\Rightarrow h(1)=g(1)$. Por tanto, $i>1$ y $h(i)\neq g(i)\Rightarrow\rho(h|\{1,2,...,i-1\})\neq\rho(g|\{1,2,...,i-1\})$. Por la definición de funcion como regla de asignación, $a=b\Rightarrow f(a)=f(b)$ y por tanto $f(a)\neq f(b)\Rightarrow a\neq b$. Entonces $\rho(h|\{1,2,...,i-1\})\neq\rho(g|\{1,2,...,i-1\})\Rightarrow h|\{1,2,...,i-1\}\neq g|\{1,2,...,i-1\}$. Pero afirmar: $h(j)=g(j)$ para todo  entero $j<i$ implica $h|\{1,2,...,i-1\}=g|\{1,2,...,i-1\}$. Esto es lo mismo que decir que si $h|\{1,2,...,i-1\}\neq g|\{1,2,...,i-1\}$ entonces existe algún entero $j<i$ tal que $h(j)\neq g(j)$. Lo cual es una contradición. Por tanto, solo hay una $h$ que cumpla esa condición.
\section{Tema 1 Sección 8 Ejercicio 8}
Sea $A$ un conjunto. Y $\rho$ una función que asigna, a cada función $f$ que aplica una sección $S_n$ de $\mathbb{Z}_{+}$ en $A$, un elemento $\rho(f)$ de $A$. Entonces existe una única función $h:\mathbb{Z}_{+}\rightarrow A$ tal que $h(n)=\rho(h|S_n)$ para cada $n\in \mathbb{Z}_{+}$. Cuando $n=1$ se tiene que $S_1=\varnothing$ por tanto, $h|S_1=\varnothing$ y para que extista $\rho(h|S_1)$ es necesario que $\rho:\mathcal{P}(A)\rightarrow A$ ya que $\varnothing \in \mathcal{P}(A)$. Veamos por el principio de inducción que se verifica esa afirmación. Sea $C$ el conjunto de todos los $n$ para los cuales se verifica que $h$ es única. Veamos que para todo $n\in C$ $h$ es única. Sea $g:\mathbb{Z}_{+}\rightarrow A$ también definida por $g(n)=\rho(g|S_n)$. Sea $i$ el mínimo elemento de $C$ para el cual se verifica que $h(i)\neq g(i)$. Ese $i$ no es 1, ya que $h(1)=\rho(\varnothing)=g(1)$ y por tanto $h(1)=g(1)$. Por tanto, $i>1$ y $h(i)\neq g(i)\Rightarrow\rho(h|S_i)\neq\rho(g|S_i)$. Por la definición de funcion como regla de asignación, $a=b\Rightarrow f(a)=f(b)$ y por tanto $f(a)\neq f(b)\Rightarrow a\neq b$. Entonces $\rho(h|S_i)\neq\rho(g|S_i)\Rightarrow h|S_i\neq g|S_i$. Pero afirmar: $h(j)=g(j)$ para todo  entero $j<i$ implica que $h|S_i=g|S_i$. Esto es lo mismo que afirmar: $h|S_i\neq g|S_i$ implica que existe algún entero $j<i$ tal que $h(j)\neq g(j)$. Lo cual es una contradición. Por tanto, solo hay una $h$ que cumpla esa condición. Ahora veamos por inducción que $C=\mathbb{Z}_{+}$. Si $n=1$ se tiene $S_1=\varnothing$ y $h(1)=\rho(h|S_1)=\rho(\varnothing)$ es único. Por tanto, la definición se cumple para $n=1$. Supongamos que $h'$ definida por $h'(n)=\rho(h'|S_n)$ es única. Entonces $h'(n)\in A$. Definamos $h$ como $h(i)=h'(i)$ si $i<n+1$ y $h(i)=\rho(h|S_i)$ si $i=n+1$ dado que no existe función sobreyectiva de $\mathcal{P}(A)$ en $A$ (por teorema 7.8), y como $\{h'(n)\}\in\mathcal{P}(A)$; y como $\{h'(n)\}$ y $h'|S_n$ son disjuntos, el conjunto $A-h'|S_n$ es no vacío y $h$ está bien definida. Por tanto $h(n+1)=\rho(h|S_{n+1})$ y por tanto $n+1\in C$. Entonces se tiene que $C=\mathbb{Z}_{+}$.
% Create a new 1st level heading
% Uncom\Rightarrow  a^n \cdot a^{0} =a^{n}ment the following two lines if you want to have a bibliography
%\bibliographystyle{alpha}
%\bibliography{document}

\end{document}
