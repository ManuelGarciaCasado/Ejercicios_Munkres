\documentclass{article}
% Uncomment the following line to allow the usage of graphics (.png, .jpg)
%\usepackage[pdftex]{graphicx9990}o0y
% Comment the following line to NOT allow the usage ofp0 umlauts

%\usepackage[utf8]{inputenc}
%\usepackage{amsmath}
%\usepackage{amssymb}

\newcommand{\vect}[1]{\boldsymbol{#1}}
% Start the document00
\begin{document}

\section{Tema 3 Sección 29 Ejercicio Complementario 7}
Sea $f:X\rightarrow Y$. Veamos que $f$ es continua si, y sólo si, para cada red  $(x_\alpha)$ convergente en $X$ que converge a $x$, entonces la red $(f(x_\alpha))$ converge a $f(x)$. Supongamos que $f$ es continua y que $(x_\alpha)$ es una red convergente en $X$ que converge a $x$. Si $f(U)$ es entorno de $f(x)$ entonces $U$ es entorno de $x$, por continuidad. Y si existe un $\alpha$ tal que $x_\alpha \in U$ y implica que $x_\beta \in U$ para todo $\beta$ tal que $\alpha\preceq \beta$. Entonces $f(x_\beta)\in f(U)$ para todo $\beta$ tal que $\alpha\preceq \beta$. Por tanto, $(f(x_\alpha))$ es una red en $Y$ que converge a $f(x)$. Recíprocamente, supongamos que para cada red $(x_\alpha)$ convergente en $X$ que converge a $x$ implica que la red $(f(x_\alpha))$ converge a $f(x)$, veamos que $f$ es continua. Entonces, como $(x_\alpha)\subset A$, resulta que $x\in \overline{A}$ y, por tanto, como $(f(x_\alpha))\subset f(A)$, resulta que $f(x)\in \overline{f(A)}$. Entonces si $y\in f(A)\subset f(\overline{A})$, cada red de elementos $(y_\alpha)$ que convergen a $y$ interseca a $f(A)$, esto es $y\in \overline{f(A)}$. Por tanto $f(\overline{A})\subset\overline{f(A)}$ y, por teorema 18.1, $f$ es continua.
\section{Tema 3 Sección 29 Ejercicio Complementario 8}
Sea $f:J\rightarrow X$ una red en $X$  y $f(\alpha)=x_\alpha$. Y sea $K$ el conjunto dirigido de tal manera que $g:K\rightarrow J$ cumple (i) $i\preceq j\Rightarrow g(i)\preceq g(j)$ y (ii) $g(K)$ es cofinal en $J$. Entonces se define la subred de $(x_\alpha)$ como $f\circ g:K\rightarrow X$. Veamos que si la red $(x_\alpha)$ converge a $x$ entonces cualquier subred suya converge a $x$. Como $g(K)$ es cofinal en $J$, para cada $\alpha\in J$ existe un $\beta \in g(K)$ tal que $\alpha\preceq \beta$. Por tanto, dado que para cada entorno $U$ de $x$ existe un $\alpha \in J$ tal que $\alpha \preceq \beta \Rightarrow x_\beta\in U$, para todo $\beta \in J$. En particular, para todo $\alpha \in J$ existe un $\beta\in g(K)$ y un $\gamma\in K$ tales que  $\alpha\preceq \beta \preceq g(\gamma) \Rightarrow x_{g(\gamma)}\in U$, para todo $\beta \in g(K)$, esto es para todo $\gamma\in K$. Por tanto para cada entorno $U$ de $x$ existe un $\alpha\in J$ tal que $\alpha \preceq g(\gamma) \Rightarrow x_{g(\gamma)}\in U$, para todo $\gamma\in K$. Por tanto, si los elementos de la red $(x_{\alpha})$ converge a $x$, los elementos de la subred $(x_{g(\alpha)})$ convergen a $x$ .
\section{Tema 3 Sección 29 Ejercicio Complementario 9}
Se dice que un punto $x$ es de acumulación de la red $(x_\alpha)$ en $X$ cuando para cada entorno de $U$ de $x$, el conjunto de indices $\alpha \in J$ tales que $x_\alpha \in U$ es cofinal en $J$. Veamos que el punto $x$ es punto de acumulación de la red $(x_\alpha)$ si, y sólo si, alguna subred de $(x_\alpha)$ converge a $x$.
Supongamos que  alguna subred de $(x_\alpha)$ converge a $x$, veamos que el punto $x$ es punto de acumulación de la red $(x_\alpha)$. Por definición, existe un conjuto $K$ tal que $g:K\rightarrow J$ cumple (i) y (ii) del ejercicio 8. Ahora llamemos $g(K)=\{\alpha| x_\alpha \in U \text{ y } \alpha\in J\}$, donde $U$ es entorno de $x$. Como existe $g(K)$ que es cofinal en $J$ para cada entorno $U$ de $x$, $x$ es punto de acumulación de la red $(x_\alpha)$.

Reciprocamente, supongamos que $x$ es punto de acumulación de la red $(x_\alpha)$, veamos que alguna subred suya converge a $x$. Procediendo como en la ayuda, sea $K$ el conjunto definido por los pares $(\alpha,U)$ donde $\alpha \in J$ y $U$ es un entorno de $x$ conteniendo a $x_\alpha$. Se define el orden $(\alpha,U)\preceq (\beta,V)$ si $\alpha\preceq \beta$ y $V\subset U$. Veamos que $K$ es un conjunto dirigido. Si $(\alpha, U), (\beta,V)\in K$ y $(\alpha, U)\preceq (\beta,V)$, entonces $x_\alpha,x\in U$, $x_\beta,x\in V$, $\alpha\preceq \beta$ y $V\subset U$. Por definición de punto de acumulación de la red, como el conjunto de indices $\alpha\in J$ tales que $x_\alpha\in U$ (dónde $U$ es cualquier entorno de $x$) es cofinal en $J$, existe un $\gamma\in J$ y un entorno $W$ de $x$ tales que $\alpha\preceq \beta\preceq \gamma$, $x_\gamma\in W$ y $W\subset V\subset U$. Por tanto existe $(\gamma, W)\in K$ tal que $(\alpha,U)\preceq (\beta, V)\preceq (\gamma, W)$. Si fuera intercambiando $\alpha$ por $\beta$ y $U$ por $V$, se llega a que existe un $(\gamma, W)\in K$ tal que $(\beta,V)\preceq (\alpha, U)\preceq (\gamma, W)$. Luego $K$ es un conjunto dirigido. Como existe un conjunto de indices $\alpha\in J$ y un entorno $U$ de $x$ tales que $J$ es cofinal y $x_\alpha\in U$, sea $g((\alpha,U)$ un elemento de tal conjunto, con $g:K\rightarrow J$ tal que $(\alpha,U)\preceq (\beta, V)\Rightarrow g((\alpha,U))\preceq g((\beta, V))$. Entonces $(x_{g((\alpha, U))})$ define una subred.
\section{Tema 3 Sección 29 Ejercicio Complementario 10}
Veamos que $X$ es compacto si, y solo si, cada red en $X$ tiene una subred convergente.
Supongamos que $X$ es compacto. Veamos que toda red tiene una subred convergente. Para ello, sea $B_\alpha=\{x_\beta|\alpha\preceq \beta\}$, primero veamos que $\{B_\alpha\}$ tiene la propiedad de la intersección finita. En efecto, sean $B_{\alpha_1},B_{\alpha_2},...,B_{\alpha_n}\in\{B_\alpha\}$ con $\alpha_1\preceq \alpha_2\preceq...\preceq\alpha_n$. Entonces $\bigcap_{i\leq n}B_{\alpha_i}=B_{\alpha_n}$. Luego cada subred de de la red $(x_\alpha)$, donde $x_\alpha\in B_\alpha$, se puede construir con los elementos de algún $B_{\alpha_n}$. Por el teorema 28.1, todo espacio compacto es compacto por punto límite. Por tanto, cada subconjunto infinito de $X$ tiene punto límite. Entonces los elementos de la red $B_\alpha$ convergen. Por ejercicio complementario 8, cada subred también converge.

Reciprocamente, supongamos que toda red tiene una subred convergente. Veamos que $X$ es compacto. Para ello, sea $\mathcal{A}$ una colección de conjuntos cerrados con la propiedad de la intersección 
finita, y sea $\mathcal{B}$ la colección de todas las intersecciones finitas de elementos de 
$\mathcal{A}$, parcialmente ordenadas por la inclusión inversa. Si $B_n,B_m\in \mathcal{B}$ con $B_n=\bigcap_{i\leq n}A_{\alpha_i}$ y $B_n\subset B_m$ entonces $B_m\preceq B_n$. Por tanto, $X-B_m\subset X-B_n$ implica $B_m\preceq B_n$. Ahora sean $x_m\in B_m$ entonces $(x_m)_{m\in \mathbb{Z}_+}$ es una red, y por hipótesis tiene una subred convergente, digamos convergente a $x$. Sea la subred convergente $(x_{g(\alpha)})_{g(\alpha)\in \mathbb{Z}_+}$ tal que $x_{g(\alpha)}\in B_{g(\alpha)}$ y $\alpha\in K$ donde $K$ es dirigido. Por el ejercicio 9, $x$ es punto de acumulación de la red $(x_m)_{m\in \mathbb{Z}_+}$. Entonces, para el entorno $U$ de $x$, existe una subred $(x_{g(\alpha)})_{g(\alpha)\in g(K)}$ donde  $x_{g(\beta)} \in U$  con $g(\alpha) \preceq g(\beta)$ para cada $g(\beta)\in g(K)$. Luego $U\cap B_{g( a)}\neq \varnothing$.  Entonces $x\in \bigcap_{g(\alpha)\in g(K)}B_{g(\alpha)}\neq \varnothing$ ya que la intersección de conjuntos cerrados es cerrado y un cerrado contiene a todos sus puntos de acumulación. Por teorema 26.9 el espacio $X$ es compacto.
\end{document}

