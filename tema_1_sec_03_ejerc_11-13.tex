\documentclass{article}
% Uncomment the following line to allow the usage of graphics (.png, .jpg)
%\usepackage[pdftex]{graphicx}
% Comment the following line to NOT allow the usage of umlauts


% Start the document
\begin{document}

% Create a new 1st level heading
\section{Tema 1 Sección 3 Ejercicio 11}
% Start the document
Sea \(A\) un conjunto ordenado con la relación \(<\) y \(A_0\subset A\). Ahora, veamos que \(x\in X\subset A\) tiene un único sucesor y un único predecesor. Si \(X=\{x| a<x<b\}\) es conjunto vacío, se define \(a\) como inmediato predecesor de \(b\) y \(b\) como inmediato sucesor de \(a\). Sea \(a'\) otro inmediato predecesor de \(b\). Entonces, \(X'=\{x| a'<x<b\}\) es cunjunto vacío. Pero el conjunto vacío es único, luego \(X'=X\Rightarrow \{x| a'<x<b\}=X'=X=\{x| a<x<b\}\Rightarrow a'=a\). Luego \(b\) tiene un único predecesor. La demostración de que \(a\) tiene un único sucesor es igual. Veamos que \(a\) tal que \(x\leq a\) para todo \(x\in A_0\) es el único maximo de \(A_0\). Sunpongamos que \(a'\) es otro máximo de \(A_0\), por definición \(x\leq a'\) para todo \(x\in A_0\). Por tanto \(x=a\leq a'\). Pero el enunciado de \(a\) máximo implica que  \(a'\leq a\). Por tanto \(a=a'\). Veamos que \(a\) tal que \(a\leq x\) para todo \(x\in A_0\) es el único minimo de \(A_0\). Sunpongamos que \(a'\) es otro minimo de \(A_0\), por definición \(a'\leq x\) para todo \(x\in A_0\). Por tanto \(a'\leq a=x\). Pero el enunciado de \(a\) mínimo implica que  \(a\leq a'\). Por tanto \(a=a'\). 

\section{Tema 1 Sección 3 Ejercicio 12}
\bf (i) \rm  Sea \(\mathbb{Z}_{+}\times \mathbb{Z}_{+}\) con el orden del diccionario. Veamos si tiene máximo. El máximo se define como el \(a_1\times a_2 \in \mathbb{Z}_{+}\times \mathbb{Z}_{+}\) tal que para todo \(x_1\times x_2\in \mathbb{Z}_{+}\times \mathbb{Z}_{+}\) se tiene \(x_1\times x_2 \leq a_1\times a_2\). Por tanto para todo \(x_1, x_2 \in \mathbb{Z}_{+}\) se tiene \(x_1\leq a_1\) y \(x_2\leq a_2\). Supongamos que \(a_1\times a_2 \in \mathbb{Z}_{+} \times \mathbb{Z}_{+} \) es el máximo. Sin embargo, \((a_1+1)\times (a_2 +1)\in \mathbb{Z}_{+} \times \mathbb{Z}_{+} \) cumple que para todo \(y_1, y_2 \in \mathbb{Z}_{+}\) se tiene \(y_1=x_1+1\leq a_1+1\) y \(y_2=x_2+1\leq a_2+1\). Por tanto, dado que no puede haber varios máximos, se concluye que no hay máximo con orden de diccionario.Veamos si tiene mínimo. El mínimo se define como el \(a_1\times a_2 \in \mathbb{Z}_{+}\times \mathbb{Z}_{+}\) tal que para todo \(x_1\times x_2\in \mathbb{Z}_{+}\times \mathbb{Z}_{+}\) se tiene \(a_1\times a_2 \leq x_1\times x_2\). Por tanto para todo \(x_1, x_2 \in \mathbb{Z}_{+}\) se tiene \(a_1\leq x_1\) y \(a_2\leq x_2\). Supongamos que \(a_1\times a_2 \in \mathbb{Z}_{+} \times \mathbb{Z}_{+} \) es el mínimo. Sin embargo, \((a_1-1)\times (a_2 -1)\in \mathbb{Z}_{+} \times \mathbb{Z}_{+} \) cumple que para todo \(y_1, y_2 \in \mathbb{Z}_{+}\) se tiene \(y_1=x_1-1\leq a_1-1\) y \(y_2=x_2-1\leq a_2-1\) excepto cuando \(a_1=1\) y \(a_2=1\). Por tanto, \(1\times 1\) es el mínimo  de \(\mathbb{Z}_{+} \times \mathbb{Z}_{+}\) con orden de diccionario.
\newline
\bf (ii) \rm  Sea \(\mathbb{Z}_{+}\times \mathbb{Z}_{+}\) con el orden definido por \((x_0,y_0)<(x_1,y_1) \text{ ya } x_0-y_0<x_1-y_1\text{, ya } x_0-y_0=x_1-y_1 \text{ y } y_0<y_1\). Veamos si tiene máximo. Supongamos que \(a_1\times b_1 \in \mathbb{Z}_{+} \times \mathbb{Z}_{+} \) es el máximo. Entonces, para todo \(x\times y\in \mathbb{Z}_{+}\times \mathbb{Z}_{+}\) se tiene que \((x,y)<(a_1,b_1) \text{ o } (x,y)=(a_1,b_1)\), es decir  \([\text{ si ya } x-y<a_1-b_1\text{, ya } x-y=a_1-b_1 \text{ y } y<a_1] \text{ o } [x=a_1 \text{ y } y=b_1]\). Pero entonces \([\text{ si ya } (x+1)-(y+1)<(a_1+1)-(b_1+1)\text{, ya } (x+1)-(y+1)=(a_1+1)-(b_1+1) \text{ y } (y+1)<(a_1+1)] \text{ o } [(x+1)=(a_1+1) \text{ y } (y+1)=(b_1+1)]\) para todo \((x+1,y+1)<(a_1+1,b_1+1) \text{ o } (x+1,y+1)=(a_1+1,b_1+1)\), para todo \((x+1,y+1)<(a_1+1,b_1+1)=(a_2,b_2) \text{ o } (x+1,y+1)=(a_2,b_2)\). Por tanto, \(a_1 \times b_1\), \(a_2\times b_2\) son dos máximos, lo cual no puede ser. Luego, no tiene máximo. Veamos si tiene mínimo. Supongamos que \(a_1\times b_1 \in \mathbb{Z}_{+} \times \mathbb{Z}_{+} \) es el mínimo. Entonces, para todo \(x\times y\in \mathbb{Z}_{+}\times \mathbb{Z}_{+}\) se tiene que \((a_1,b_1)<(x,y) \text{ o } (x,y)=(a_1,b_1)\), es decir  \([\text{ si ya } a_1-b_1<x-y\text{, ya } x-y=a_1-b_1 \text{ y } a_1<y] \text{ o } [x=a_1 \text{ y } y=b_1]\). Pero entonces \([\text{ si ya } (a_1-1)-(b_1-1)<(x-1)-(y-1)\text{, ya } (x-1)-(y-1)=(a_1-1)-(b_1-1) \text{ y } (a_1-1)<(y-1)] \text{ o } [(x-1)=(a_1-1) \text{ y } (y-1)=(b_1-1)]\) para todo \((a_1-1,b_1-1)<(x-1,y-1) \text{ o } (x-1,y-1)=(a_1-1,b_1-1)\), para todo \((a_2,b_2)=(a_1-1,b_1-1)<(x-1,y-1)\text{ o } (x-1,y-1)=(a_2,b_2)\). Por tanto, \(a_1 \times b_1\), \(a_2\times b_2\) son el mismo mínimo, ya que \((x,y)=(1,1)\Rightarrow (a_1,b_1)=(1,1)\) y \((x-1,y-1)=(1,1)\Rightarrow (a_2,b_2)=(1,1)\). Luego, tiene mínimo.
\newline
\bf (iii) \rm  Sea \(\mathbb{Z}_{+}\times \mathbb{Z}_{+}\) con el orden definido por \((x_0,y_0)<(x_1,y_1) \text{ ya } x_0+y_0<x_1+y_1\text{, ya } x_0+y_0=x_1+y_1 \text{ y } y_0<y_1\). Veamos si tiene máximo. Supongamos que \(a_1\times b_1 \in \mathbb{Z}_{+} \times \mathbb{Z}_{+} \) es el máximo. Entonces, para todo \(x\times y\in \mathbb{Z}_{+}\times \mathbb{Z}_{+}\) se tiene que \((x,y)<(a_1,b_1) \text{ o } (x,y)=(a_1,b_1)\), es decir  \([\text{ si ya } x+y<a_1+b_1\text{, ya } x+y=a_1+b_1 \text{ y } y<a_1] \text{ o } [x=a_1 \text{ y } y=b_1]\). Pero entonces \([\text{ si ya } (x+1)+(y+1)<(a_1+1)+(b_1+1)\text{, ya } (x+1)+(y+1)=(a_1+1)+(b_1+1) \text{ y } (y+1)<(a_1+1)] \text{ o } [(x+1)=(a_1+1) \text{ y } (y+1)=(b_1+1)]\). Pero entonces \([\text{ si ya } x+y<(x+1)+(y+1)<(a_1+1)+(b_1+1)\text{, ya } x+y<(x+1)+(y+1)=(a_1+1)+(b_1+1) \text{ y } y<(y+1)<(a_1+1)] \text{ o } [x<(x+1)=(a_1+1) \text{ y } y<(y+1)=(b_1+1)]\) para todo \((x,y)<(a_1+1,b_1+1) \text{ o } (x+1,y+1)=(a_1+1,b_1+1)\), para todo \((x,y)<(a_1+1,b_1+1)=(a_2,b_2) \text{ o } (x,y)=(a_2,b_2)\). Por tanto, \(a_1 \times b_1\), \(a_2\times b_2\) son dos máximos, lo cual no puede ser. Luego, no tiene máximo. Veamos si tiene mínimo. Supongamos que \(a_1\times b_1 \in \mathbb{Z}_{+} \times \mathbb{Z}_{+} \) es el mínimo. Entonces, para todo \(x\times y\in \mathbb{Z}_{+}\times \mathbb{Z}_{+}\) se tiene que \((a_1,b_1)<(x,y) \text{ o } (x,y)=(a_1,b_1)\), es decir  \([\text{ si ya } a_1+b_1<x+y\text{, ya } x+y=a_1+b_1 \text{ y } a_1<y] \text{ o } [x=a_1 \text{ y } y=b_1]\). Pero entonces \([\text{ si ya } (a_1-1)+(b_1-1)<(x-1)+(y-1)<x+y\text{, ya } x+y >(x-1)+(y-1)=(a_1-1)+(b_1-1) \text{ y } (a_1-1)<(y-1)<y] \text{ o } [x>(x-1)=(a_1-1) \text{ y } y>(y-1)=(b_1-1)]\) para todo \((a_1-1,b_1-1)<(x-1,y-1) \text{ o } (x-1,y-1)=(a_1-1,b_1-1)\), para todo \((a_2,b_2)=(a_1-1,b_1-1)<(x-1,y-1)<(x,y)\text{ o } (x,y)> (x-1,y-1)=(a_2,b_2)\). Por tanto, \(a_1 \times b_1\), \(a_2\times b_2\) son el mismo mínimo, ya que \((x,y)>(1,1)\Rightarrow (a_1,b_1)=(1,1)\) y \((x,y)>(x-1,y-1)=(1,1)\Rightarrow (a_2,b_2)=(1,1)\). Luego, tiene mínimo.

\section{Tema 1 Sección 3 Ejercicio 13}
Veamos que un conjunto ordenado \(A\) tiene propiedad de supremo, si y solo si, también tiene propiedad de ínfimo. Sea \(A^*_ 0=\{a|\text{para todo }x\in A_0, x\leq a \in A\}\). El mínimo \(b\) de \(A^*_0\) es el supremo de \(A_0\). Es decir, si existe \(b\in A^*_0 \text{ tal que } b\leq x \text{ para todo } x\in A^*_0 \text{ y para todo } A_0 \neq \varnothing, A_0 \subset A\), se tiene la propiedad del supremo. Sea \(b\) el supremo, luego \(b\in A^*_0 \Rightarrow b\leq x \text{ para todo } x \in A^*_0 \text{ y } y\leq b\text{ para todo } y\in A_0 \). Por tanto, como \(A^*_0\subset A\), \(b\in A \Rightarrow b\leq x \text{ para todo } x \in A^*_0 \text{ y } y\leq b\text{ para todo } y\in A_0 \). Por tanto, \(b\in \{a|\text{para todo }x\in A^*_0, a\leq x, a \in A\}\text{ y } y\leq b\text{ para todo } y\in A_0 \). Si fuera \(A_0 \subset\{a|\text{para todo }x\in A^*_0, a\leq x, a \in A\}\), se tendría que b es el máximo de todas las cotas inferiores de \(A^*_0\), el ínfimo de \(A^*_0\). Queda demostrar que \(A_0 \subset B_0=\{a|\text{para todo }x\in A^*_0, a\leq x, a \in A\}\). Se tiene que \(B_0= \{a|\text{ para todo }x\in A^*_0, a\leq x, a \in A\}\), entonces \(B_0 = \{a|\text{ para todo }x\in \{c|\text{ para todo }y\in A_0, y\leq c \in A\}, a\leq x, a \in A\}\). Pero \(B_0= \{a|\text{ para todo }x \in A\text{ y todo } y\in A_0,  a \leq x \text{ y } y\leq x \text{ y } a\in A\}\supset A_0\). Por tanto \(b\in B_0 \text{ y } y\leq b\text{ para todo } y\in A_0\subset B_0 \Rightarrow b\in B_0 \text{ y } y\leq b\text{ para todo } y\in B_0\), siendo \(b\) el máximo de las cotas inferiores de \(A^*_0\).
% Uncomment the following two lines if you want to have a bibliography
%\bibliographystyle{alpha}
%\bibliography{document}

\end{document}
