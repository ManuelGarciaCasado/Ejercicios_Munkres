\documentclass{article}
% Uncomment the following line to allow the usage of graphics (.png, .jpg)
%\usepackage[pdftex]{graphicx9990}o0y
% Comment the following line to NOT allow the usage ofp0 umlauts



\newcommand{\vect}[1]{\boldsymbol{#1}}
% Start the document00
\begin{document}
\section{Tema 1 Sección 1 Ejercicio 4}
Sean $A,B$ conjuntos de numeros reales.
\begin{itemize}
\item \bf (a) \rm Veamos la negación del enunciado "Para todo $a\in  A$, se verifica que $a^2\in B$."
\end{itemize}
"Para almenos un $a\in A$, se verifica que $a^2\notin B$."
\begin{itemize}
\item \bf (b) \rm Veamos la negación del enunciado "Para almenos un $a\in  A$, se verifica que $a^2\in B$."
\end{itemize}
"Para todo $a\in A$, se verifica que $a^2\notin B$."
\begin{itemize}
\item \bf (c) \rm Veamos la negación del enunciado "Para todo $a\in  A$, se verifica que $a^2\notin B$."
\end{itemize}
"Para almenos un $a\in A$, se verifica que $a^2\in B$."
\begin{itemize}
\item \bf (d) \rm Veamos la negación del enunciado "Para almenos un $a\notin  A$, se verifica que $a^2\in B$."
\end{itemize}
"Para todo $a\notin A$, se verifica que $a^2\notin B$."
\section{Tema 1 Sección 1 Ejercicio 5}
Sea $\mathcal{A}\neq \varnothing$. 
\begin{itemize}
\item \bf (a) \rm Veamos si la afirmación "$x\in \bigcup_{A\in \mathcal{A}} A\Rightarrow x\in A$ para al menos un  $A\in \mathcal{A}$" es verdadera.
\end{itemize}
La afirmación es verdadera. Por definición, $\bigcup_{A\in \mathcal{A}}A=\{y|y\in A \text{ para algún }A\in \mathcal{A}\}$. Entonces $x\in \bigcup_{A\in \mathcal{A}}A\Leftrightarrow x\in \{y|y\in A \text{ para algún }A\in \mathcal{A}\}$. Como $x$ es uno de los elementos del conjunto, $x\in A$ para algún $A\in \mathcal{A}$.
\begin{itemize}
\item \bf (b) \rm Veamos si la afirmación "$x\in \bigcup_{A\in \mathcal{A}} A\Rightarrow x\in A$ para todo  $A\in \mathcal{A}$" es verdadera.
\end{itemize}
La afirmación es falsa. Por definición, $\bigcup_{A\in \mathcal{A}}A=\{x|x\in A \text{ para algún }A\in \mathcal{A}\}$. Entonces $x\in \bigcup_{A\in \mathcal{A}}A\Leftrightarrow x\in \{y|y\in A \text{ para algún }A\in \mathcal{A}\}$. Como $x$ es uno de los elementos del conjunto, $x\in A$ para algún $A\in \mathcal{A}$. Es decir, no se da que $x\in A$ para todo $A\in \mathcal{A}$
\begin{itemize}
\item \bf (c) \rm Veamos si la afirmación "$x\in \bigcap_{A\in \mathcal{A}} A\Leftrightarrow x\in A$ para al menos un $A\in \mathcal{A}$" es verdadera. 
\end{itemize}
La afirmación es verdadera. Por definición, $\bigcap_{A\in \mathcal{A}}A=\{x|x\in A \text{ para todo }A\in \mathcal{A}\}$. Entonces $x\in \bigcap_{A\in \mathcal{A}}A\Rightarrow x\in \{y|y\in A \text{ para todo }A\in \mathcal{A}\}$. Como $x$ es uno de los elementos del conjunto, $x\in A$ para todo $A\in \mathcal{A}$. Es decir, se da que $x\in A$ para algún $A\in \mathcal{A}$. Otra forma de verlo es que, como $\bigcap_{A\in \mathcal{A}}A\subset x\in\bigcup_{A\in \mathcal{A}}A$, resulta que $x\in \bigcap_{A\in \mathcal{A}}A\Rightarrow x\in\bigcup_{A\in \mathcal{A}}A$, que es verdad según apartado (a).
\begin{itemize}
\item \bf (d) \rm Veamos si la afirmación "$x\in \bigcap_{A\in \mathcal{A}} A\Rightarrow x\in A$ para todo $A\in \mathcal{A}$" es verdadera.
\end{itemize}
La afirmación es verdadera. Por definición, $\bigcap_{A\in \mathcal{A}}A=\{y|y\in A \text{ para todo }A\in \mathcal{A}\}$. Entonces $x\in \bigcap_{A\in \mathcal{A}}A\Leftrightarrow x\in \{y|y\in A \text{ para todo }A\in \mathcal{A}\}$. Como $x$ es uno de los elementos del conjunto, $x\in A$ para todo $A\in \mathcal{A}$.
\section{Tema 1 Sección 1 Ejercicio 6}
Veamos el contrarrecíproco de las afirmaciones del ejercicio 5.
Sea $\mathcal{A}\neq \varnothing$. 
\begin{itemize}
\item \bf (a) \rm $x\notin A$ para todo  $A\in \mathcal{A} \Rightarrow x\notin \bigcup_{A\in \mathcal{A}}A$
\end{itemize}
\begin{itemize}
\item \bf (b) \rm $x\notin A$ para algún $A\in \mathcal{A} \Rightarrow x\notin \bigcup_{A\in \mathcal{A}}A$
\end{itemize}
\begin{itemize}
\item \bf (c) \rm $x\notin A$ para todo $A\in \mathcal{A} \Rightarrow x\notin \bigcap_{A\in \mathcal{A}}A$
\end{itemize}
\begin{itemize}
\item \bf (d) \rm $x\notin A$ para algún $A\in \mathcal{A} \Rightarrow x\notin \bigcap_{A\in \mathcal{A}}A$
\end{itemize}
\section{Tema 1 Sección 1 Ejercicio 7}
Sean $A,B,C$ conjuntos
\begin{itemize}
\item Veamos cómo se escribe $D=\{x| x\in A \text{ y }(x\in B \text{ o } x\in C)\}$
\end{itemize}
$D=\{x| x\in A \text{ y }x\in B \cup C\}= \{x|x\in A\cap (B\cup C)\}=A\cap (B\cup C)$
\begin{itemize}
\item Veamos cómo se escribe $E=\{x| (x\in A \text{ y }x\in B) \text{ o } x\in C)\}$
\end{itemize}
$E=\{x|( x\in A\cap B \text{ o }x\in C\}= \{x|x\in (A\cap B)\cup C)\}=(A\cap )B\cup C$
\begin{itemize}
\item Veamos cómo se escribe $F=\{x| x\in A \text{ y }(x\in B \Rightarrow x\in C)\}$
\end{itemize}
$F=\{x| x\in A \text{ y }(x\in B \subset C)\}= \{x|x\in A\cap (B\subset C)\}=(A\cap B)\subset (A\cap C)$
\section{Tema 1 Sección 1 Ejercicio 8}
Veamos que si $A$ tiene dos elementos, $\mathcal{P}(A)$ tiene cuatro elementos. Veamos qué pasa cuando $A$ tiene uno, tres o ningún elemento.
Si $A=\{a,b\}$, entonces $\mathcal{P}(A)=\{\varnothing, \{a\},\{b\},A\}$. Si $A=\{a\}$, entonces $\mathcal{P}(A)=\{\varnothing, A\}$. Si $A=\{a,b,c\}$, entonces $\mathcal{P}(A)=\{\varnothing, \{a\},\{b\}, \{c\},\{a,b\},\{a,c\},\{b,c\},A\}$. Si $A=\varnothing$, entonces $\mathcal{P}(A)=\{\varnothing\}$
\section{Tema 1 Sección 1 Ejercicio 9}
Veamos las leyes de Morgan para $\bigcap_{A\in \mathcal{A}}A$ y  $\bigcup_{A\in \mathcal{A}}A$.

Sea $x\in A-\bigcup_{B\in \mathcal{A}}B$ si, y sólo si, $x\in\{y|y\in A \text{ e }y\notin\bigcup_{B\in \mathcal{A}}B \}$ si, y sólo si, $x\in\{y|y\in A \text{ e }y\notin\{z|z\in B\text{ para algún }B\in \mathcal{A}\}\}$ si, y sólo si, $x\in\{y|y\in A \text{ e }y\notin B \text{ para todo }B\in \mathcal{A}\}$ si, y sólo si, $x\in\{y|y\in (A - B) \text{ para todo }B\in \mathcal{A}\}$ si, y sólo si, $x\in \bigcap_{B\in \mathcal{A}}(A-B)$.

Sea $x\in A-\bigcap_{B\in \mathcal{A}}B$ si, y sólo si, $x\in\{y|y\in A \text{ e }y\notin\bigcap_{B\in \mathcal{A}}B \}$ si, y sólo si, $x\in\{y|y\in A \text{ e }y\notin\{z|z\in B\text{ para todo }B\in \mathcal{A}\}\}$ si, y sólo si, $x\in\{y|y\in A \text{ e }y\notin B \text{ para algún }B\in \mathcal{A}\}$ si, y sólo si, $x\in\{y|y\in (A - B) \text{ para algún }B\in \mathcal{A}\}$ si, y sólo si, $x\in \bigcup_{B\in \mathcal{A}}(A-B)$.

\section{Tema 1 Sección 1 Ejercicio 10}
Veamos cuales de los siguentes subconjuntos de $\mathbb{R}\times \mathbb{R}$ es igual al procucto cartesiano de dos subconjuntos de $\mathbb{R}$
\begin{itemize}
\item \bf (a) \rm Sea $\{(x,y)| x\text{ es entero}\}$
\end{itemize}
$\{(x,y)| x\text{ es entero}\}=\bigcup_{x\in \mathbb{Z}}(\{x\}\times \mathbb{R})=(\bigcup_{x\in \mathbb{Z}}\{x\})\times \mathbb{R}=\mathbb{Z}\times \mathbb{R}$
\begin{itemize}
\item \bf (b) \rm Sea $\{(x,y)| 0<y\leq 1\}$
\end{itemize}
$\{(x,y)| 0<y\leq 1\}= \mathbb{R}\times \{x|0<x\leq 1\}$
\begin{itemize}
\item \bf (c) \rm Sea $\{(x,y)| 0<y\leq 1\}$
\end{itemize}
$\{(x,y)| y>x\}$ no se puede escribir como producto escalar porque puesto que los elementos de los pares ordenados $(x,y)=\{\{x\},\{x,y\}\}$ de aquel conjunto no son independientes.
\begin{itemize}
\item \bf (d) \rm Sea $\{(x,y)| x\notin \mathbb{Z} \text{ e } y\in \mathbb{Z}\}$
\end{itemize}
$\{(x,y)| x\notin \mathbb{Z} \text{ e } y\in \mathbb{Z}\}=\{(x,y)| x\in \mathbb{R}-\mathbb{Z} \text{ e } y\in \mathbb{Z}\}= (\mathbb{R}-\mathbb{Z})\times \mathbb{Z}$ aquel conjunto no son independientes.
\begin{itemize}
\item \bf (e) \rm Sea $\{(x,y)| x^2+y^2< 1\}$
\end{itemize}
$\{(x,y)| x^2+y^2< 1\}$ no se puede escribir como producto escalar porque puesto que los elementos de los pares ordenados $(x,y)=\{\{x\},\{x,y\}\}$ de aquel conjunto no son independientes.


\end{document}
