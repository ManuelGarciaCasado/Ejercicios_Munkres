\documentclass{article}
% Uncomment the following line to allow the usage of graphics (.png, .jpg)
%\usepackage[pdftex]{graphicx9990}o0y
% Comment the following line to NOT allow the usage ofp0 umlauts



\newcommand{\vect}[1]{\boldsymbol{#1}}
% Start the document00
\begin{document}
\section{Tema 2 Complementarios Ejercicio 4}
Sea $\alpha$ un elemento de $G$, veamos que las aplicaciones $f_\alpha: G\rightarrow G$ y $g_\alpha: G\rightarrow G$ definidas por $f_\alpha(x)=\alpha \cdot x$ y $g_\alpha (x)=x\cdot \alpha$ son homeomorfismos de $G$. Entonces hay que demostrar que $f_\alpha$ y $f^{-1}_\alpha$ son ambas continuas. Como $f_\alpha$ es la restricción de $h:G\times G\rightarrow G$ dada por $h(x\times y)= x\cdot y $ cuando $x=\alpha$ para todo $x\in G$, por teorema 18.3(d), $f_\alpha$ es continua. Además, $f^{-1}_\alpha$ es continua por ser la composisión $i\circ f_\alpha$ de dos funciones continuas, donde $i:G\rightarrow G$ está dada por $i(x)=x^{-1}$. Del mismo modo, $g_\alpha$ es la resticción de $h:G\times G\rightarrow G$ dada por $h(x\times y)= x\cdot y $ cuando $y=\alpha$ para todo $y\in G$, por teorema 18.3(d), $f_\alpha$ es continua. Además, $g^{-1}_\alpha$ es continua por ser la composisión $i\circ g_\alpha$ de dos funciones continuas, donde $i:G\rightarrow G$ está dada por $i(x)=x^{-1}$. Por tanto, $f_\alpha$ y $g_\alpha$ son homeomorfismos. Veamos que $G$ es un espacio homogeneo (para cada para de puntos $x$ e $y$ de $G$ hay una aplicación homeomorfa que lleva el uno al otro. Se tiene que para todo par de puntos $x,y\in G$ existe un $\alpha \in G$ tal que $f_\alpha$ es homeomorfo. Sea $\alpha= x$ para todo $x\in G$ y $f(x\times y)=x\cdot y$
\section{Tema 2 Complementarios Ejercicio 5}
Sea $H$ un subgrupo de $G$. Y sea la clase por la izquierda de $H$ en $G$ definida por el conjunto $xH=\{x\cdot h| h\in H\}$ tal que $x\in G$. Sea $G/H$ la colección de clases por la izquierda de $H$ en $G$, que es una partición de $G$. Supongamos que $G/H$ es una topología cociente.
\begin{itemize}
\item \bf (a) \rm Veamos que dado $\alpha \in G$, la función $f_\alpha :G\rightarrow G$ tal que $f_\alpha(x)=\alpha \cdot x$ induce un homeomorfismo en $G/H$ que lleva $xH$ a $(\alpha \cdot x) H$. Veamos también que $G/H$ es un espacio homogéneo.
\end{itemize}
Como 
\begin{eqnarray}
f_\alpha(xH)=\{f_{\alpha}(x\cdot h)|h\in H\}=\{\alpha \cdot (x\cdot h)|h\in H\},\nonumber
\end{eqnarray}
usando la propiedad distributiva $\alpha \cdot (x\cdot h)=(\alpha \cdot x)\cdot h$, se tiene
\begin{eqnarray}
f_\alpha(xH)=\{(\alpha \cdot x)\cdot h|h\in H\}=(\alpha \cdot x)H.\nonumber
\end{eqnarray}
Por tanto, $f_\alpha$ lleva los conjuntos $xH$ de $G$ a los conujntos $(\alpha \cdot x)H$ de $G$. ea $p:G\rightarrow G/H$ la aplicación cociente dada por $p(x)=xH$. Por tanto $h_\alpha=p\circ (f_\alpha\circ p^{-1})$, es una aplicación $h_\alpha:G/H\rightarrow G/H$ tal que $h_\alpha(xH)=(\alpha \cdot x)H$, por ejercicio 4, es homeomorfismo de $G/H$. Puesto que para todo par de puntos $xH$ e $yH$ hay un $\alpha\in G$ tal que $y=\alpha\cdot x$, también existe un $h_\alpha$ que es homoemorfismo de $G/H$ y, por tanto, $G/H$ es espacio homogéneo.
\begin{itemize}
\item \bf (b) \rm Veamos que si $H$ es cerrado en la topología de $G$, entonces los conjuntos unipuntuales son cerrados en $G/H$.
\end{itemize}
Lo que hay que probar es que si es $H$ un conjunto cerrado de $G$ entonces el conjunto unipuntual $\{xH\}$, tal que $xH\in G/H$ y tal que $x\in G$, es cerrado en $G/H$. Como $G$ es un grupo topologico, satisface el axioma $T_1$, por tanto $\{x\}$ es cerrado. Como $G/H$ está dotada de la topología cociente, existe una aplicación cociente $p:G\rightarrow G/H$ definida por $p(x)=xH$, donde $xH=\{x\cdot h|h\in H\}\in G/H$. Esto es, $p$ es sobreyectiva y $U$ es abierto en $G/H$ si, y solo si, $p^{-1}(U)$ es abierto de $G$. Por otro lado, $xH=\{f_x(h)|h\in H\}=f_x(H)$, ya que $f_x:G\rightarrow G$ definido por $f_x(h)=x\cdot h$. Como $f_x$ es homoemorfismo por ejercicio 4, $f_x(H)=xH$ es cerrado en $G$, por serlo $H$. Pero por apartado (a), las funciomes $f_x$  inducen homeomorfismos de $G/H$ en $G/H$ definido por $h_x(yH)= (x\cdot y)H$ para cada $x\in G$. Pero, si $\{xH\}$ es abierto, entonces, $ p^{-1}(\{xH\})$ es abierto. Pero por ser $G/H$ una partición de $G$, $ p^{-1}(\{xH\})=\{x\}$. Pero $\{x\}$ es cerrado. Por tanto, $\{xH\}$ tiene que ser cerrado.
\begin{itemize}
\item \bf (c) \rm Veamos que la aplicación cociente $p:G\rightarrow G/H$ es abierta.
\end{itemize}
Por apartado (b), $\{x\}$ cerrado implica $p(\{x\})$ cerrado. Se tiene que si $U$ es cerrado de $G$, entonces $U=\bigcup^n_{i=1 }\{ x_i\}$ con $x_i\in G$ para algún $n\in \mathbb{Z}_+$. Por tanto $p(U)=p(\bigcup^n_{i=1}\{x_i\})=\bigcup^n_{i=1}p(\{x_i\})=\bigcup^n_{i=1}\{x_iH\}=V$ es cerrado. Además, por ser una partición $p(x)\cap p(y)= xH\cap yH=\varnothing$ en $G$. Entonces, $x\neq y\Rightarrow p(x)\neq p(y)$ en $G/H$. Por tanto, $p$ es inyectiva. Por tanto, $p(A-B)=p(A)-P(B)$ para ciertos conjuntos $A,B\subset G$ Como $G-U$ es abierto, $p(G-U)= p(G)-p(U)=G/H-V$ es abierto. Luego $p$ es una aplicación cerrada.
\begin{itemize}
\item \bf (d) \rm Veamos que si $H$ es cerrado en la topología $G$ y es un grupo normal de $G$, entonces $G/H$ es un grupo topológico.
\end{itemize}
Un subgrupo $N$ es normal si $n\in N$ y $g^{-1}ng\in N$ para todo $g\in G$. Esto es $g^{-1}Ng\subset N$ para todo $g\in G$. Veamos que las funciones de $G/H\times G/H$ en $G/H$, enviando $xH \times yH$ a $xH \cdot yH$, y la aplicacion de $G/H$ en $G/H$, enviando $xH$ a $(xH)^{-1}$, son continuas. Sea $U$ un cerrado de $G/H$, entonces $U=\cup_{i=1}^n \{x_iH\}$, pero por ser $H$ normal , $x_iH=\{x_i\cdot h|h\in G\}=\{x_i\cdot [x_j^{-1}\cdot (h\cdot x_j)]|h\in H\}$ y, por ser grupo $H$ un grupo, $h=h_1\cdot h_2$, luego $x_iH=\{x_i\cdot h|h\in G\}=\{[(x_i\cdot x_j^{-1})\cdot h_1]\cdot (h_2\cdot x_j)|h_1,h_2\in H\}$. Sea $y_i=x_i \cdot x_j^{-1}$,  estonces  $x_iH=\{x_i\cdot h|h\in G\}=\{(y_i\cdot h_1)\cdot (h_2\cdot x_j)|h_1,h_2\in H\}=y_iH \cdot x_j^{-1}H$. Por tanto, si $x_iH\in G/H$, entonces $(y_i\cdot H)\times (x_j^{-1}\cdot H)\in G/H\times G/H$. Esto es, si $\{x_iH\}\subset G/H$, entonces $\{y_iH\times x_j^{-1}H\}\subset G/H\times G/H$. Por tanto, si $U=U_1\cdot U_2$ es cerrado en $G/H$, existe un $U_1\times U_2$ que es cerrado en $G/H\times G/H$, ya que $U=\bigcup_{i=1}^{n}x_iH,U_1=\bigcup_{i=1}^{n}x_iH\text{ y }U_2= \bigcup_{j=1}^{n}x_j^{-1}H$. Por tanto, el producto es una función continua. Igualmente se demuestra que la aplicación de invertir elementos de $G/H$ en $G/H$ es continua. Si
$(x_iH)^{-1}=\{(x_i\cdot h)^{-1}|h\in H\}=\{x^{-1}_i\cdot h|h^{-1}\in H\}=\{x^{-1}_i\cdot h|h\in H\}$ porque $h\in H\Rightarrow h^{-1}\in H$. Luego si $\{(xH)^{-1}\}\subset G/H$ cerrado entonces $\{x^{-1}H\}\subset G/H$ cerrado y si $V=\bigcup_{i=1}^n\{(x_iH)^{-1}\}$ cerrado en $G/H$ entonces $U=\bigcup_{i=1}^n\{x_i^{-1}H\}$ cerrado en $G/H$. Luego $G/H$ es un grupo.
\section{Tema 2 Complementarios Ejercicio 6}
Los enteros $\mathbb{Z}$ son un grupo normal de $(\mathbb{R},+)$. Veamos qué grupo topológico es $\mathbb{R}/\mathbb{Z}$.
Se tiene que $x\mathbb{Z}=\{x+n|n\in\mathbb{Z}\}\in \mathbb{R}/\mathbb{Z}$. Por tanto, $p(x)=x\mathbb{Z}$ induce la topología cociente. Dado el abierto $(a,b)$ de $\mathbb{R}$, se tiene que $p((a,b))=\{(a+n,b+n)|n\in\mathbb{Z}\}\subset\mathbb{R}/\mathbb{Z}$. Se tiene que $f(t)=(\sin (t),\cos (t))=\sin(t)+i\cos(t)=e^{it}$ es una función de $\mathbb{R}$ en $\mathbb{S}^1$ que es sobreyectiva y continua, pero $e^{i2\pi n}=1$. Por tanto existe una funcion $g: \mathbb{R}/\mathbb{Z}\rightarrow \mathbb{S}^1$ definida por $g((2\pi t)\mathbb{Z})=g(\{2\pi t+2\pi n|n \in \mathbb{Z}\})=e^{i2\pi t}$ con $n\in \mathbb{Z}$. Como por corolario 22.3, $g$ es un homeomorfismo, el grupo topológico $\mathbb{R}/\mathbb{Z}$ es equivalente a $\mathbb{S}^1$.
\section{Tema 2 Complementarios Ejercicio 7}
Sean $A$ y $B$ subconjuntos de $G$ y sea $A\cdot B$ en conjunto de los puntos $a\cdot b$ ta que $a\in A$ y $b\in B$; y sea $A^{-1}$ el subconjunto de $G$ tal que $a^{-1}$, para $a\in A$.
\begin{itemize}
\item \bf (a) \rm Un entorno $V$ del elemento identidad $e$ se dice simétrico si $V=V^{-1}$. Si $U$ es entorno $e$, veamos que existe un entorno simétrico $V$ tal que $V\cdot V\subset U$.
\end{itemize}
Veamos que si $W$ es entorno de $e$, entonces $W\cdot W^{-1}=\{a\cdot b|a\in W, b\in W^{-1}\}$. Como $e\in W$, para cada $c\in W$ existe un $c^{-1}\in W$ y $a,b\in W$ tales que $a\cdot b=c$. Por tanto $W\cdot W^{-1}=\{a^{-1}\cdot b^{-1}|a^{-1}\in W^{-1}, b^{-1}\in W\}=W^{-1}\cdot W$. Luego, si $W$ es entorno de $e$, $W\cdot W^{-1}$ es simétrico y además $e\in W\cdot W^{-1}=V=V^{-1}$. Además, si $e\in V$ y $V=V^{-1}$, entonces $e\in V\cdot V^{-1}=V\cdot V$. Por tanto, si $e\in U$, existe un $W\subset U$ y un $V=W\cdot W^{-1}$ tal que $V=V^{-1}$ y tal que $e\in V\cdot V$. Por teorema 18.1 (4) como $f(A,B)=A\cdot B$ es continua, si $A\cdot B$ es entorno de $a\cdot b$, existe un entorno $C$ de $a\times b$ tal que $f(C)\subset A\cdot B$. Además, la alplicacion $h:G\rightarrow G\times G$ definida por $h(x)= x\times x$ es continua. Luego $f\circ h$ es continua. Por tanto, si $V\cdot V$ es abierto, $V$ es abierto. Por tanto, si $U$ es entorno de $e$, existe un $V\cdot V\subset U$ tal que $V\cdot V$ es entorno de $e$ y un $V$ que tambien es entorno de $e$. Por tanto $V\cdot V$ es simétrico y $V$ también es simétrico. 
\begin{itemize}
\item \bf (b) \rm Veamos que $G$ es Hausforff, veamos que existe un entorno $V$ de $e$ y unos elementos $x,y\in G$ tales que $x\neq y$ implica $V\cdot x\cap V\cdot y=\varnothing $.
\end{itemize}
Supongamos $x,y\in G$ tal que $x\neq y$. Sea $V$ el entorno simétrico de $e$ tal que $V\cdot y=V\cdot x^{-1}$ y tal que $x\notin V$. Entonces $V\cdot y=V^{-1}\cdot x^{-1}=(V\cdot x)^{-1}$. Si fuera $V\cdot y\cap V\cdot x\neq \varnothing$, se tendría que $(V\cdot x)^{-1}\cap (V\cdot x)\neq \varnothing$. Entonces $ (V\cdot x)^{-1}\cap (V\cdot x)= \{e\}$. Pero si $e\in V\cdot x$ entonces $e=x^{-1}\cdot x $ y si $e\in (V\cdot x)^{-1}= V\cdot x^{-1}$ entonces $e=x\cdot x^{-1}$. Pero esto contradice la suposición inicial de que $x\notin V$. Entonces, para los elementos $x\neq y$ de $G$, existe un entorno $V$ de $e$ tal que $V\cdot x$ es entorno de $x$ (ya que $e\cdot x\in V\cdot x$), tal que $V\cdot y$ es entorno de $y$ (ya que $e\cdot y\in V\cdot y$),  tal que $x\notin V$ y tal que $V\cdot y = V\cdot x^{-1}$. Entonces $(V\cdot y)\cap (V\cdot x)=\varnothing$ y $G$ es Hausdorff.
\begin{itemize}
\item \bf (c) \rm Veamos que $G$ cumple el axioma de regularidad. Veamos que dado un conjunto cerrado $A$ de $G$ y un $x\notin A$, existen conjuntos disjuntos abiertos que contienen a $A$ y a $x$ respectivamente.
\end{itemize}
Puesto que $A$ es cerrado y $G$ es Hausdorff, $G$ cumple el axioma $T_1$. Por tanto, $A$ tiene un número finito de elementos. Luego $A=\bigcup_{i=1}^n\{y_i\}$. Por apartado (b), para cada $y_i\neq x$ existen entornos $V_i$ de $e$ tales que el entorno $V_i\cdot y_i$ de $y_i$ y el entorno $V_i\cdot x$ de $x$ son disjuntos. Por tanto, $V_i\cdot y_i\cap V_i\cdot x=\varnothing$ y $\bigcup_{i=1}^n\left(V_i\cdot y_i\cap V_i\cdot x\right)= (\bigcup_{i=1}^n(V_i\cdot y_i))\cap(\bigcup_{i=1}^n (V_i\cdot x))$
Pero $\bigcup_{i=1}^n (V_i\cdot x)=(\bigcup_{i=1}^n V_i)\cdot x$ y $\bigcup_{i=1}^n (V_i\cdot y_i)=(\bigcup_{i=1}^n V_i)\cdot \bigcup_{i=1}^n \{y_i\}$ (la imagen, por una función, de una unión es la unión de las imagenes, por esa función). Sea $V=\bigcup_{i=1}^n V_i$, entonces $\varnothing = V\cdot A\cap V\cdot x$
\begin{itemize}
\item \bf (d) \rm Sea $H$ un subgrupo cerrado en la topología de $G$ y la aplicación cociente $p:G\rightarrow G/H$. Veamos que $G/H$ cumple el axioma de regularidad. Veamos que dado un conjunto cerrado $p(A)$ de $G/H$ y un $xH\notin p(A)$, existen conjuntos disjuntos abiertos que contienen a $p(A)$ y a $xH$ respectivamente.
\end{itemize}
Por tanto, como $p(A)$ es cerrado, $A$ es cerrado por ser $p$ continua, y además $A$ es saturado en $G$. Por ejercicio 7 (c), existen abiertos $V\cdot A$ que contienen a $A$ y entornos $V\cdot x$ de $x$ tales que $V\cdot A\cap V\cdot x$. $G$, donde $V$ es entorno de $e$. Como $p$ es un aplicación abierta por ejercicio 5 (b), $p(V\cdot A) $ es abierto que contiene a $p(A)$. Esto se debe a que $p(A)$ es unión finita de cerrados (conjuntos unipuntuales). Del mismo modo, $p(V\cdot x) $ es abierto y contiene a $p(x)$. Se tiene que $p(V\cdot A)=\{zH |z\in (V\cdot A)\}=\{z\cdot h|h\in H,z\in (V\cdot A)\}=\{(v\cdot y )\cdot h|h\in H,v\in V,y\in A)\}$, como $(v\cdot y )\cdot h = v\cdot (y \cdot h)$ se tiene que $p(V\cdot A)=(V\cdot A)H=V\cdot AH= V\cdot p(A)$, por tanto, dado que $A$ es saturado, $V\cdot A=p^{-1}(V\cdot p(A))$ es saturado. Entonces $\varnothing =V\cdot A\cap V\cdot x=p(V\cdot A)\cap p(V\cdot x)$ por ser $V\cdot A$ saturado (vease demostración teorema 22.1).
\end{document}
