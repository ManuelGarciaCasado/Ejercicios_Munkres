\documentclass{article}
% Uncomment the following line to allow the usage of graphics (.png, .jpg)
%\usepackage[pdftex]{graphicx9990}o0y
% Comment the following line to NOT allow the usage ofp0 umlauts

\newcommand{\vect}[1]{\boldsymbol{#1}}
% Start the document00
\begin{document}
\section{Tema 2 Sección 16 Ejercicio 5}
Sean $X$ y $X'$ conjuntos de las topologías $\mathcal{T}$ y $\mathcal{T}'$, respectivamente; y  sean $Y$ e $Y'$ conjuntos de las topologías $\mathcal{U}$ y $\mathcal{U}'$, respectivamente. Asumimos que ninguno de los conjuntos es vacío.

\bf(a)\rm Veamos que si $\mathcal{T}'\supset\mathcal{T}$ y $\mathcal{U}'\supset\mathcal{U}$ entonces la topología producto sobre $X'\times Y'$ es mas fina que la topología producto sobre $X\times Y$. Dado que $X\in \mathcal{T}$ y $Y\in \mathcal{U}$, se tiene que la topología producto de $\mathcal{T}\times\mathcal{U}$ contiene al elemento $X\times Y$ y dado que $X'\in \mathcal{T}'$ y $Y'\in \mathcal{U}'$, se tiene que la topología producto de $\mathcal{T}'\times\mathcal{U}'$ contiene al elemento $X'\times Y'$. Pero por lema 13.3 si $\mathcal{T}'\supset\mathcal{T}$ entonces existe un elemento $B\in \mathcal{B}$ y otro $B'\in \mathcal{B}'$, bases de $\mathcal{T}$ y $\mathcal{T}'$, respectivamente, tales que $x\in B' \Rightarrow x\in B$ (esto es $B'\subset B$). Y por lema 13.3 si $\mathcal{U}'\supset\mathcal{U}$ entonces existe un elemento $C\in \mathcal{C}$ y otro $C'\in \mathcal{C}'$, bases de $\mathcal{U}$ y $\mathcal{U}'$, respectivamente, tales que $y\in C' \Rightarrow y\in C$ (esto es $C'\subset C$). Por tanto, se tiene que $(x,y)\in B'\times C' \Rightarrow (x,y)\in B\times C$. Por lema 13.3,la topología $\mathcal{T}'\times \mathcal{U}'$ es una topología mas fina que $\mathcal{T}\times \mathcal{U}$ ya que, por teorema 15.1, $B'\times C'$ pertenece a la base de la topología $\mathcal{T}'\times \mathcal{U}'$ y $B\times C$ pertenece a la base de la topología $\mathcal{T}\times \mathcal{U}$.

\bf(b)\rm Si $B\times C$ pertenece a la base de la topología $\mathcal{T}\times \mathcal{U}$ y $B'\times C'$ pertenece a la base de la topología $\mathcal{T}'\times \mathcal{U}'$. Entonces $B=\pi_1(B\times C)$,  $C=\pi_2(B\times C)$, $B'=\pi_1(B'\times C')$ y $C'=\pi_2(B'\times C')$ son abiertos de $X,Y,X'\text { e }Y'$, respectivamente. Suponiendo que $\mathcal{T}'\times \mathcal{U}'\supset\mathcal{T}\times \mathcal{U}$ entonces se tiene $X\times Y\subset X'\times Y'$ y que $B'\times C'\subset B\times C$. Pero por una de las propiedades de las funciones, $B'\times C'\subset B\times C\Rightarrow \pi_1(B'\times C')\subset \pi_1(B\times C)\text{ y } \pi_2(B'\times C')\subset \pi_2(B\times C)$, luego implica que $B'\subset B$ y que $C'\subset C$ simultáneamente. Se demuestra que $B'$, $B$, $C'$ y $C$ pertenecen a las bases de $\mathcal{T}'$,$\mathcal{T}$,$\mathcal{U}'$ y $\mathcal{U}$, respectivamente. Esto es debido a que para todo $(x,y)\in X\times Y$ se tiene que hay un $B$ tal que $\pi_1(x,y)\in B=\pi_1(B\times C)\subset  \pi_1(X\times Y)$; a que para todo $(x,y)\in X'\times Y'$ se tiene que hay un $B'$ tal que $\pi_1(x,y)\in B'=\pi_1(B'\times C')\subset  \pi_1(X'\times Y')$; a que para todo $(x,y)\in X\times Y$ se tiene que hay un $C$ tal que $\pi_2(x,y)\in C=\pi_2(B\times C)\subset  \pi_2(X\times Y)$; y a que para todo $(x,y)\in X'\times Y'$ se tiene que hay un $C'$ tal que $\pi_2(x,y)\in C'=\pi_2(B'\times C')\subset  \pi_2(X'\times Y')$.  Aplicando otra vez el lema 3.3 se tiene $\mathcal{T}'\supset\mathcal{T}$ y $\mathcal{U}'\supset\mathcal{U}$.
\section{Tema 2 Sección 16 Ejercicio 6}
Veamos que la colección numerable
\begin{equation}
\begin{aligned}
\{(a,b)\times (c,d)|a<b\text{ y }c<d, \text{ y }a,b,c,d\in \mathbb{Q}\}
\end{aligned}
\end{equation}
es una base para $\mathbb{R}^2$. Se tiene que para cada elemento de intervalo abierto $x\in(x_1,x_2)\subset \mathbb{R}$  existe un $(a,b)$ con $a,b\in \mathbb{Q}$ y $a<b$ tal que $x\in(a,b)\subset(x_1,x_2)$. Por tanto, los $(a,b)$ con $a,b\in \mathbb{Q}$ y $a<b$ forman una base en $\mathbb{R}$, por ejercicio 13.8(a). Por teorema 15.1, los $(a,b)\times (c,d)$ con $a,b,c,d\in \mathbb{Q}$ y tales que $a<b$ y $c<d$ forman una base para $\mathbb{R}^2$.
\section{Tema 2 Sección 16 Ejercicio 7}
Sea $X$ un conjunto ordenado y sea $Y$ subconjunto propio de $X$ y convexo en $X$. Veamos si $Y$ es un intervalo o un rayo de $X$.
Por ser $Y$ convexo en $X$ se tiene que $Y\subset X$ y que para cada par de puntos $a,b\in Y$ donde $ a<b$ el intervalo completo $(a,b)\subset X$ está en $Y$. Se supone que $Y\subsetneq X$. Sea $Y=(b,c)\cup\{d\}$ (donde $b<c<d$ son puntos de $X$) y $X=(a,c)\cup[d,e)$. Entonces $X$ tiene la topología del orden por ser unión de conjuntos $(a,c)$ y $[d,e)$ que pertenecen a la base de la topología del orden. Entonces $Y$ es convexo en $X$, pero no es ni un rayo ni un intervalo de $X$.
\section{Tema 2 Sección 16 Ejercicio 8}
Sea $L$ una recta de $\mathbb{R}\times \mathbb{R}$ Veamos que topología hereda como subespacio de $\mathbb{R}_l\times \mathbb{R}$ y como subespacio de $\mathbb{R}_l\times \mathbb{R}_l$. 
Por teorema 15.1, la topología de $\mathbb{R}_l\times \mathbb{R}$ tiene como elementos base $[a,b)\times (c,d)$ para $a<b$ y $c<d$ ya que $[a,b)$ y $(c,d)$ son elementos de bases de las topoogías $\mathbb{R}_l$ y $\mathbb{R}$, respectivamente. Sea $\mathcal{T}$ la topología de $\mathbb{R}_l\times \mathbb{R}$ Por tanto, la topología de subespacio $\mathcal{T}_L=\{L\cap U| U\in \mathcal{T}\}$ tiene como elementos base $L\cap([a,b)\times(c,d))$. La recta en el plano se define como $L=\{(x_1,x_2)| \alpha (x_1+a_1)= x_2+a_2\text{ y }x_1,x_2,a_1,a_2,\alpha\in \mathbb{R}\}$. Por tanto, para $\alpha>0$, $L\cap([a,b)\times(c,d))=\{(x_1,x_2)|  \alpha (x_1+a_1)= x_2+a_2\text{ y }a_1,a_2,\alpha\in \mathbb{R},x_1\in [a,b)\text{ y }x_2\in (c,d)\}$. Es decir, $a\leq (x_2+a_2)/\alpha-a_1<b$. Por tanto,  $a\leq (x_2+a_2)/\alpha-a_1<b$ y $c<x_2<d$ por tanto  $(a+a_1)\alpha -a_2\leq x_2<(b+a_1)\alpha-a_2$ y $c<x_2<d$; del mismo modo, $a\leq x_1<b$ y $(c+a_2)/\alpha-a_1<x_1<(d+a_2)/\alpha-a_1$. Luego, para $\alpha>0$, si $(a+a_1)\alpha -a_2< c$ o $a<(c+a_2)/\alpha-a_1$,  $\mathcal{T}_L$ hereda la topología de $\mathbb{R}$ y si $c\leq (a+a_1)\alpha -a_2$ o $(c+a_2)/\alpha-a_1\leq a$,  $\mathcal{T}_L$ entonces hereda la topología de $\mathbb{R}_l$. Para $\alpha<0$, Luego, si $(a+a_1)\alpha -a_2< c$ o $a<(c+a_2)/\alpha-a_1$,  $\mathcal{T}_L$ hereda la topología de $\mathbb{R}_l$ y si $c\leq (a+a_1)\alpha -a_2$ o $(c+a_2)/\alpha-a_1\leq a$,  $\mathcal{T}_L$ entonces hereda la topología de $\mathbb{R}$

Por teorema 15.1, la topología de $\mathbb{R}_l\times \mathbb{R}_l$ tiene como elementos base $[a,b)\times [c,d)$ para $a<b$ y $c<d$ ya que $[a,b)$ y $[c,d)$ son elementos de bases de las topoogías $\mathbb{R}_l$. Sea $\mathcal{T}$ la topología de $\mathbb{R}_l\times \mathbb{R}_l$ Por tanto, la topología de subespacio $\mathcal{T}_L=\{L\cap U| U\in \mathcal{T}\}$ tiene como elementos base $L\cap([a,b)\times[c,d))$. La recta en el plano se define como $L=\{(x_1,x_2)| \alpha (x_1+a_1)= x_2+a_2\text{ y }x_1,x_2,a_1,a_2,\alpha\in \mathbb{R}\}$. Por tanto, para $\alpha>0$, $L\cap([a,b)\times[c,d))=\{(x_1,x_2)|  \alpha (x_1+a_1)= x_2+a_2\text{ y }a_1,a_2,\alpha\in \mathbb{R},x_1\in [a,b)\text{ y }x_2\in [c,d)\}$. Es decir, $a\leq (x_2+a_2)/\alpha-a_1<b$. Por tanto,  $a\leq (x_2+a_2)/\alpha-a_1<b$ y $c\leq x_2<d$ por tanto  $(a+a_1)\alpha -a_2\leq x_2<(b+a_1)\alpha-a_2$ y $c\leq x_2<d$; del mismo modo, $a\leq x_1<b$ y $(c+a_2)/\alpha-a_1\leq x_1<(d+a_2)/\alpha-a_1$. Luego,  para $\alpha>0$, si $(a+a_1)\alpha -a_2\leq c$ o $a\leq(c+a_2)/\alpha-a_1$,  $\mathcal{T}_L$ hereda la topología de $\mathbb{R}_l$ y si $c\leq (a+a_1)\alpha -a_2$ o $(c+a_2)/\alpha-a_1\leq a$,  $\mathcal{T}_L$ entonces hereda la topología de $\mathbb{R}_l$ también. Para $\alpha<0$,  si $(a+a_1)\alpha -a_2\leq c$ o $a\leq(c+a_2)/\alpha-a_1$,  $\mathcal{T}_L$ hereda la topología de $\mathbb{R}_u$ y si $c\leq (a+a_1)\alpha -a_2$ o $(c+a_2)/\alpha-a_1\leq a$,  $\mathcal{T}_L$ entonces hereda la topología de $\mathbb{R}_u$

\section{Tema 2 Sección 16 Ejercicio 9}
Veamos que la topología del orden del diccionario sobre $\mathbb{R}\times\mathbb{R}$ es la misma que la topología $\mathbb{R}_d\times\mathbb{R}$ donde $\mathbb{R}_d$ denota a $\mathbb{R}$ con la topología discreta. Según ejemplo 2 de sección 14, la base de la topología de orden de diccionario sobre $\mathbb{R}\times\mathbb{R}$ es los intervalos abiertos $(a\times b,c\times d)$ con $a<c$ y si $a=c$ con $b<d$. La topología discreta de $\mathbb{R}_d\times\mathbb{R}$ tiene por base a los elementos del tipo $(a,b)\times(c,d)$,  $[a,b)\times(c,d)$, $(a,b]\times(c,d)$,  $[a,b]\times(c,d)$ y $\{a\}\times (c,d)$. Por tanto, como $\{a\}\times (c,d)=(a\times c, a\times d)$, se tiene que éste es un elemento de la base de la topología de orden de diccionario sobre $\mathbb{R}\times\mathbb{R}$. Si $a<b$, entonces $(a,b)\times(c,d)=\{\{x\}\times (c,d)|a<x<b\}$. Pero el intervalo $(a\times c,b\times d)= \{\{x\}\times (c,d)|a<x<b\}$. Por tanto, $(a,b)\times (c,d)=(a\times c,b\times d)$. Para $[a,b)\times(c,d)$ se costruye la unión  $(a\times c, a\times d)\cup (a\times c, b\times d)$; para $[a,b]\times(c,d)$ se costruye la unión  $(a\times c, a\times d)\cup (a\times c, b\times d)\cup (b\times c, b\times d)$ . Luego la base $\mathbb{R}_d\times\mathbb{R}$ y la base de la topología de orden de diccionario sobre $\mathbb{R}\times\mathbb{R}$ tienen los mismos elementos de la base. Se tiene que la topología del orden del diccionario sobre $\mathbb{R}\times\mathbb{R}$ es mas fina que la topología usual sobre $\mathbb{R}\times\mathbb{R}$, puesto que el intervalo $(a\times c, b\times d)$ está contenido en $(a,b)\times (c,d)$ pero lo contrario no se da.

\section{Tema 2 Sección 16 Ejercicio 10}
Sea $I=[1,0]$. Llamemos $\mathcal{T}_P$ a la topología producto sobre $I\times I$, que tiene por elementos base los conjuntos $A\times B$ donde $A$ y $B$ son intervalos abiertos, semiabiertos o cerrados que son subconjuntos de $I$. Llamemos $\mathcal{T}_O$ a la topología del orden sobre $I\times I$, que tiene por elementos base los intervalos $(a\times b, c\times d)$ para $0\leq a<c\leq 1$ o para $0\leq a=c\leq 1$, $0\leq b<d\leq 1$; los elementos del tipo $\{0\}\times [0,a)$ o del tipo $\{1\}\times (b,1]$. Y llamemos $\mathcal{T}_S$ a la topología de $I\times I$ como subespacio de $\mathbb{R}\times \mathbb{R}$ con la topología del orden, que tiene como elementos base los intervalos  $\{(x_1\times y_1, x_2\times y_2)| [[x_1<x_2] \text{ o } [x_1=x_2\text{ e } y_1<y_2]] \text{ y }[x_1,x_2,y_1,y_2\in I]\}$. Dado que los elementos base de $\mathcal{T}_S$ pertenecen a la bese de $\mathcal{T}_O$ pero $\{0\}\times [0,a)$ no pertenece  la base de $\mathcal{T}_S$, se tiene que $\mathcal{T}_O\subset \mathcal{T}_S$. Del mismo modo, los elementos $A\times B$ de la base de $\mathcal{T}_P$ contienen a los elementos de $\mathcal{T}_O$. Por tanto $\mathcal{T}_S\subset\mathcal{T}_O\subset\mathcal{T}_P$




\end{document}
