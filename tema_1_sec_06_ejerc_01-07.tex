\documentclass{article}
% Uncomment the following line to allow the usage of graphics (.png, .jpg)
%\usepackage[pdftex]{graphicx}
% Comment the following line to NOT allow the usage of umlauts

\newcommand{\vect}[1]{\boldsymbol{#1}}
% Start the document
\begin{document}

% Create a new 1st level heading
\section{Tema 1 Sección 6 Ejercicio 1}
% Create a new 1st level heading
\begin{itemize}
\item \bf (a) \rm
\end{itemize}
Sea la función $f:\{1,2,3\}\rightarrow \{1,2,3,4\}$
Veamos todas la funciones inyectivas:
\begin{eqnarray}
f(x)=\begin{cases}
1,& x=1\nonumber\\
2,& x=2\nonumber\\
3,& x=3\nonumber\\
\end{cases}
\end{eqnarray}
\begin{eqnarray}
f(x)=\begin{cases}
1,& x=2\nonumber\\
2,& x=3\nonumber\\
3,& x=1\nonumber\\
\end{cases}
\end{eqnarray}
\begin{eqnarray}
f(x)=\begin{cases}
1,& x=3\nonumber\\
2,& x=1\nonumber\\
3,& x=2\nonumber\\
\end{cases}
\end{eqnarray}
\begin{eqnarray}
f(x)=\begin{cases}
1,& x=2\nonumber\\
2,& x=1\nonumber\\
3,& x=3\nonumber\\
\end{cases}
\end{eqnarray}
\begin{eqnarray}
f(x)=\begin{cases}
1,& x=3\nonumber\\
2,& x=2\nonumber\\
3,& x=1\nonumber\\
\end{cases}
\end{eqnarray}
\begin{eqnarray}
f(x)=\begin{cases}
1,& x=1\nonumber\\
2,& x=3\nonumber\\
3,& x=2\nonumber\\
\end{cases}
\end{eqnarray}
\begin{eqnarray}
f(x)=\begin{cases}
1,& x=2\nonumber\\
2,& x=3\nonumber\\
3,& x=4\nonumber\\
\end{cases}
\end{eqnarray}
\begin{eqnarray}
f(x)=\begin{cases}
1,& x=3\nonumber\\
2,& x=4\nonumber\\
3,& x=2\nonumber\\
\end{cases}
\end{eqnarray}
\begin{eqnarray}
f(x)=\begin{cases}
1,& x=4\nonumber\\
2,& x=2\nonumber\\
3,& x=3\nonumber\\
\end{cases}
\end{eqnarray}
\begin{eqnarray}
f(x)=\begin{cases}
1,& x=2\nonumber\\
2,& x=4\nonumber\\
3,& x=3\nonumber\\
\end{cases}
\end{eqnarray}
\begin{eqnarray}
f(x)=\begin{cases}
1,& x=3\nonumber\\
2,& x=2\nonumber\\
3,& x=4\nonumber\\
\end{cases}
\end{eqnarray}
\begin{eqnarray}
f(x)=\begin{cases}
1,& x=4\nonumber\\
2,& x=3\nonumber\\
3,& x=2\nonumber\\
\end{cases}
\end{eqnarray}
\begin{eqnarray}
f(x)=\begin{cases}
1,& x=1\nonumber\\
2,& x=4\nonumber\\
3,& x=3\nonumber\\
\end{cases}
\end{eqnarray}
\begin{eqnarray}
f(x)=\begin{cases}
1,& x=4\nonumber\\
2,& x=3\nonumber\\
3,& x=1\nonumber\\
\end{cases}
\end{eqnarray}
\begin{eqnarray}
f(x)=\begin{cases}
1,& x=3\nonumber\\
2,& x=1\nonumber\\
3,& x=4\nonumber\\
\end{cases}
\end{eqnarray}
\begin{eqnarray}
f(x)=\begin{cases}
1,& x=4\nonumber\\
2,& x=1\nonumber\\
3,& x=3\nonumber\\
\end{cases}
\end{eqnarray}
\begin{eqnarray}
f(x)=\begin{cases}
1,& x=3\nonumber\\
2,& x=4\nonumber\\
3,& x=1\nonumber\\
\end{cases}
\end{eqnarray}
\begin{eqnarray}
f(x)=\begin{cases}
1,& x=1\nonumber\\
2,& x=3\nonumber\\
3,& x=4\nonumber\\
\end{cases}
\end{eqnarray}
\begin{eqnarray}
f(x)=\begin{cases}
1,& x=1\nonumber\\
2,& x=2\nonumber\\
3,& x=4\nonumber\\
\end{cases}
\end{eqnarray}
\begin{eqnarray}
f(x)=\begin{cases}
1,& x=2\nonumber\\
2,& x=4\nonumber\\
3,& x=1\nonumber\\
\end{cases}
\end{eqnarray}
\begin{eqnarray}
f(x)=\begin{cases}
1,& x=4\nonumber\\
2,& x=1\nonumber\\
3,& x=2\nonumber\\
\end{cases}
\end{eqnarray}
\begin{eqnarray}
f(x)=\begin{cases}
1,& x=2\nonumber\\
2,& x=1\nonumber\\
3,& x=4\nonumber\\
\end{cases}
\end{eqnarray}
\begin{eqnarray}
f(x)=\begin{cases}
1,& x=4\nonumber\\
2,& x=2\nonumber\\
3,& x=1\nonumber\\
\end{cases}
\end{eqnarray}
\begin{eqnarray}
f(x)=\begin{cases}
1,& x=1\nonumber\\
2,& x=4\nonumber\\
3,& x=2\nonumber\\
\end{cases}
\end{eqnarray}
Dado que en la imagen de cada uno de las funciones falta un elemento del conjunto $\{1,2,3,4\}$, estas funciones no son sobreyectivas, por tanto, no son biyectivas.
\begin{itemize}
\item \bf (b) \rm
\end{itemize}
Vemos las qué aplicaciones $f:\{1,2,...,8\}\rightarrow \{1,2,...,10\}$ hay. Si hay $n$ elementos para elegir de los cuales se seleccionan $r\leq n$ sin repetición y donde importa el orden, entonces hay $n\times (n-1)\times (n-2)\times ... \times (n-r+1)$ aplicaciones  inyectivas diferentes que se pueden construir. En este caso hay $n=10$ y $r=8$, por tanto son $10\times 9\times 8\times ... \times 3=1.814.400$ aplicaciones  inyectivas diferentes que se pueden construir.

\section{Tema 1 Sección 6 Ejercicio 2}
Veamos que si $B$ es un conjunto no finito y $B\subset A$ entonces $A$ es no finito. suponiendo $B\subset A$, se tiene que  $[A\text{ finito  }\Rightarrow B\text{ finito}] \Leftrightarrow [B\text{ no finito }\Rightarrow A\text{ no finito}]$. Por tanto, esto es lo mismo que decir que si $A$ es finito y $B\subset A$ entonces $B$ es finito. Eso es el corolario 6.6.

\section{Tema 1 Sección 6 Ejercicio 3}
Sea $X=\{0,1\}$. Veamos que hay una función biyectiva de  $X^{\omega}$ a un subconjunto propio de $X^{\omega}$. Sea $\vect{x}=(x_1,x_2,...)\in X^{\omega}$ con $x_i\in\{0,1\}$ e $i\in \mathbb{Z}_{+}$. Sea   la función  $f:X^{\omega}\rightarrow A$ donde $A=\{\vect{x}|\vect{x}\in X^{\omega}\text{ y }x_1=1\}$ definida como $f(\vect{x})=(1,x_1,x_2,...)$. Se puede ver que si $\vect{x}\neq \vect{y}$ entonces $f(\vect{x})=(1,x_1,x_2,...)\neq (1,y_1,y_2,...)\Rightarrow f(\vect{x})\neq f(\vect{y})$. Y para todo $\vect{x}\in A$ existe un $\vect{x}\in X^{\omega} $ Por tanto es sobreyectiva. Por tanto $f$ es biyectiva.
\section{Tema 1 Sección 6 Ejercicio 4}

Sea $A$ un conjunto finito simplemente ordenado. Por definición la relación de orden simple $R$ es tal que, i) para todo $x,y\in A$ con $x\neq y$ se tiene que $xRy$ o $yRx$. ii) ningún $x\in A$ cumple $xRx$. iii) si $x,y,z\in A$ cumple $xRy$ e $yRz$ entonces $xRz$.
\begin{itemize}
\item \bf (a) \rm
\end{itemize}

Veamos que $A$ tiene un elemento mayor. Como $A$ es finito, entonces exite una función biyectiva de $A$ en $\{1,2,...,n\}$ para algún $n\in \mathbb{Z}_{+}$. Si $A={a_1}$, entonces $n=1$ y $a_1$ es el elemento mayor. Si $A=\{a_1,a_2\}$ y $a_1Ra_2$ entonces $a_2$ es el elemento mayor en caso contrario $a_1$ es el elmento mayor; y $n=2$. Sea $C$ en subconjunto de los $\mathbb{Z}_{+}$ para los cuales se cumple $A$ es finito de orden $n-1$ y tiene un elemento mayor $a_{n-1}$. Veamos que si $n-1\in C$ entonces $n\in C$. Sea $B-\{a_n\}=A$ donde $A$ es  un subconjunto propio de $B$, entonces, por lema 6.1 hay una biyección $g:B\rightarrow \{1,2,...,n\}$, lo cual significa que $B$ es también finito. Por tanto, si $a_{n-1}Ra_n$, entonces $a_n$ es el elemento mayor de $B$; en caso contrario $a_{n-1}$ es el elemento mayor de $B$. En ambos casos $B$ tiene elemento mayor. Por tanto $n\in C$, y por inducción, $C=\mathbb{Z}_{+}$. Lo cual demuestra que si $A$ es un conjunto finito ordenado simple, entonces tiene elemento mayor.
\begin{itemize}
\item \bf (b) \rm
\end{itemize}
Se dice que $R$ en $A$ y $<$ en $\{1,2,...,n\}$ tienen el mismo tipos de orden si existe una función biyectiva $f:A\rightarrow \{1,2,...,n\}$ tal que $a_1Ra_2 \Rightarrow f(a_1)<f(a_2)$ Dado que $A$ tiene un elemento mayor $a$ , se puede definir $f(a)=n$. Por tanto, para todo $b\in A-\{a\}$ existe un $i\in \{1,2,...,n\}$ tal que $bRa\Rightarrow i<n$. Del mismo modo hay una biyección entre $A-\{a\}$ y $\{1,2,...,n-1\}$ por el lema 6.1. Por tanto $A-\{a\}$ tiene un elemento mayor $c$ tal que se puede definir $f(c)=n-1$. Por tanto, para todo $d\in A-\{a\}-\{c\}$ existe un $i\in \{1,2,...,n-1\}$ tal que $dRc\Rightarrow i<n-1$. Siguiendo esta construcción hasta el menor elemento de $A$, se tiene una biyección $f:A\rightarrow \{1,2,...,n\}$ tal que $a_1Ra_2\Rightarrow f(a_1)<f(a_2)$
\section{Tema 1 Sección 6 Ejercicio 5}
Sea $A\times B$ finito. Veamos si $A$ y $B$ son ambos finitos o no. Si $A\times B$ es finito entonces hay una aplicación biyectiva entre $A\times B$ y $\{1,2,...,n\}$. Por el corolario 6.6,  $A\times \{b\}$ es finito para todo $b\in B$ por ser $A\times \{b\}\subset A\times B$. Dado que para cada $b\in B$ se puede construir una aplicación biyectiva $g:A\times \{b\}\rightarrow A$  como $g(a,b)=a$ se tiene la aplicación biyectiva $f\circ g^{-1}: A\rightarrow \{1,2,...,m\}$ donde $m\leq n$. Por tanto $A$ es finito. Igualmente se demuestra que $B$ es finito.
\section{Tema 1 Sección 6 Ejercicio 6}
\begin{itemize}
\item \bf a \rm
\end{itemize}
Sea $A=\{1,2,...,n\}$. Veamos que hay una biyección entre $\mathcal{P} (A)$ y el producto cartesiano $X^n$ donde $X=\{0,1\}$. El cardinal de $X^n$ es $2^n$ ya que hay una función biyectiva entre $f:X^n\rightarrow \{1,2,3,...,2^n\}$ con $f(\vect{x})=1 + x_12^0+x_22^1+ x_32^2+ ...+x_{2^n}2^{2^n-1}$ Dado que $\mathcal{P} (A)$ es el conjunto de todos los subconjuntos de $A$, se tiene que cada elemento de $\mathcal{P} (A)$ es un conjunto finito o es el conjunto vacío. Despues vamos a ver que el cardinal de $\mathcal{P} (A)$ es $2^n$. Entonces hay una aplicación biyectiva $g:\mathcal{P} (A)\rightarrow \{1,2,3,...,2^n\}$ y por tanto hay una aplicación biyectiva $g\circ f^{-1}:\mathcal{P} (A)\rightarrow X^n$. Veamos que el cardinal de $\mathcal{P} (A)$ es $2^{n}$. Sean los elementos de  $\vect{x}\in X^n$ tales que sus $i$-esimas coordenadas tiene valor $x_i=1$ si el elemento $i$ pertenece al subconjunto de $A$ y $x_i=0$ en caso contrario. Así se puede ver que se pueden formar $2^n$ subconjuntos de $A$, los cuales son elementos de   $\mathcal{P} (A)$.
\begin{itemize}
\item \bf (b) \rm
\end{itemize}
Por ejercicio (a), que hay una biyeccion $g:\mathcal{P} (A)\rightarrow \{1,2,3,...,2^n\}$, por corolario 6.7, $\mathcal{P} (A)$ es finito.
\section{Tema 1 Sección 6 Ejercicio 7}
Si $A$ y $B$ son finitos, veamos que el conjunto de todas  aplicaciones $f:A\rightarrow B$ es finito. El número de funciones $f:A\rightarrow B$ viene determinado por el número de subconjuntos de $A\times B$, ya que una función es una regla de asignación que es un subconjunto de $A\times B$. Si el cardinal de $A$ es $n$ y el cardinal de $B$ es $m$, el número de elementos de $A\times B$ es $nm$. El número de subconjuntos de $A\times B$ viene dado por el cardinal de $\mathcal{P}(A\times B)=2^{nm}$. Luego el número de funciones es menor que $2^{nm}$.
% Uncom\Rightarrow  a^n \cdot a^{0} =a^{n}ment the following two lines if you want to have a bibliography
%\bibliographystyle{alpha}
%\bibliography{document}

\end{document}
