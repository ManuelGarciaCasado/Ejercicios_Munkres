\documentclass{article}
% Uncomment the following line to allow the usage of graphics (.png, .jpg)
%\usepackage[pdftex]{graphicx9990}o0y
% Comment the following line to NOT allow the usage ofp0 umlauts

%\usepackage[utf8]{inputenc}
%\usepackage{amsmath}
%\usepackage{amssymb}

\newcommand{\vect}[1]{\boldsymbol{#1}}
% Start the document00
\begin{document}
\section{Tema 3 Sección 24 Ejercicio 6}
Veamos que si $X$ es un conjunto bien ordenado, entonces $X\times [0,1)$ con la topología del orden es continuo lineal.
Por definición de conjunto bien ordenado, todo subconjunto no vacío de $X$ tiene un mínimo. Si tiene mínimo, entonces el mínimo es también la cota inferior máxima. Por tanto, $X$ tiene la propiedad del ínfimo, ya que todo conjunto que tiene cota inferior, tiene ínfimo. Por ejercicio 14(c) de la sección 3, todo conjunto ordenado que tiene la propiedad del ínfimo, tiene la propiedad del supremo.
Veamos si se cumple que existe un $x\times y$ tal que $x_1\times y_1<x\times y< x_2\times y_2$ para cada par de puntos  $x_1\times y_1, x_2\times y_2\in X\times [0,1)$. Por las propiedades del subconjunto $[0,1)$ de $\mathbb{R}$ se tiene que  $y_1<y< y_2$. Por tanto, si $x_1<x_2$ entonces $x_1\times y_1<x_2 \times y_1<x_2\times y_2$ y, si $x_1=x_2$ se tiene que $x_1\times y_1<x_1\times y<x_1\times y_2=x_2\times y_2$ Por tanto, $X\times [0,1)$ también cumple la segunda propiedad. Por tanto, es continuo lineal.
\section{Tema 3 Sección 24 Ejercicio 7}
\begin{itemize}
\item \bf (a) \rm Sean $X$ e $Y$ conjuntos ordenados con la topología del orden. Veamos que si $f:X\rightarrow Y$ es sobreyectiva y preserva el orden, entonces $f$ es un homeomorfismo.
\end{itemize}
Supongamos que $X$ tiene mínimo y máximo. Sean $a_0$ y $b_0$ el mínimo y el máximo de $X$, respectivamente. Entonces $[a_0, a)$, $(a,b)$ y $(b,b_0]$ son abiertos de $X$ en la topología del orden. Por conservar el orden se tiene que $a<b$ implica $f(a)<f(b)$ en $Y$. Por ser sobreyectiva, $f(a_0)$ y $f(b_0)$  son el mínimo y el máximo de $Y$, respectivamente. Además no hay elementos $a\neq b$ tales que $f(a)=f(b)$ ya que bien $a<b$, bien $b>a$ y en tales casos, bien  $f(a)<f(b)$, bien $f(b)<f(a)$. Por tanto $f$ es inyectiva. Por tanto, $f$ es biyectiva. Si no existe un $f(c)\in Y$ tal que $f(a)<f(c)<f(b)$, entonces $Y=[f(a_0),f(b))\cup (f(a),f(b_0)]$ y $[a_0,b)\cap (a,b_0]=\varnothing$, entonces $Y$ es conexo y el abierto $(f(a),f(b))=\varnothing$ en $Y$ y se lleva al abierto $f^{-1}((f(a),f(b))=\varnothing$ en $X$. Si existe un $f(c)\in Y$ tal que $f(a)<f(c)<f(b)$, entonces se tiene el abierto $(f(a),f(b))=\{f(c)| f(a)<f(c)<f(b)\}$ en $Y$. Como $f$ es sobreyectiva, $f^{-1}((f(a),f(b))=(a,b)$ es abierto de $X$. Igualmente pasa para los intervalos $[f(a_0),f(b))$ y $(f(a_0),f(b_0)]$. Por tanto $f$ es continua. Por la propiedad de conservar el orden, si existe un $c\in X$ tal que $a<c<b$, entonces se tiene el abierto $(a,b)=\{c| a<c<b\}$ en $Y$. Por tanto, $f((a,b))=(f(a),f(b))$ es abierto de $X$. Si no existe tal $c$, entonces $(a,b)=\varnothing$ y $f((a,b))=\varnothing$. Del mismo modo, $f([a_0,b))=[f(a_0),f(b))$ y $f((a,b_0])=(f(a),f(b_0)]$. Por tanto $f^{-1}$ es continua. Por tanto, $f$ es homeomorfismo.
\begin{itemize}
\item \bf (b) \rm Veamos que si dados $X=Y=\overline{\mathbb{R}_+}$ y dado $n\in \mathbb{Z}_+$, la función $f(x)=x^n$ preserva el orden y es sobreyectiva.
\end{itemize}
Si $a,b \in \overline{\mathbb{R}_+}$ y $a\neq 0$, se tiene que si $a<b$, entonces $a\cdot a < a \cdot b < b\cdot b$. Supongamos que $a^{n-1}<b^{n-1}$ y que $a<b$, entonces $a^n= a^{n-1}\cdot a < b^{n-1} \cdot a < b^{n-1}\cdot b=b^n$. Por inducción, se tiene que $a<b$ implica $a^n<b^n$, para todo $n\in \mathbb{Z}_+$. Si $a=0<b$ entonces $ 0<b$ implica $0=0\cdot b< b\cdot b$, y si $ 0<b^{n-1}$ implica $0=0\cdot b^{n-1} < b\cdot b^{n-1}$. Por inducción, se tiene que $0<b$ implica $0^n<b^n$, para todo $n\in \mathbb{Z}_+$. Como $y\in\overline{\mathbb{R}_+}$, se tiene que $y\geq 0$. Por tanto, existe un $x\geq 0$ tal que $x^n=y$ y que $x= y^{1/n}$. Por tanto, para todo $y\in\overline{\mathbb{R}_+}$ existe un $x\in\overline{\mathbb{R}_+}$ tal que $x=f^{-1}(y)$. Por tanto, $f$ es sobreyectiva.
\begin{itemize}
\item \bf (c) \rm Sea $X$ el subespacio $(-\infty,-1)\cup [0,\infty)$ de $\mathbb{R}$. Veamos que la función $f:X\rightarrow \mathbb{R}$ definida por
\begin{eqnarray}
f(x)=
\begin{cases}
x+1 & x<-1\nonumber\\
x & x\geq 0 \nonumber
\end{cases}
\end{eqnarray} 
es una función que preserva el orden y que es sobreyectiva.
\end{itemize}
Si $a,b\in \mathbb{R}$ y $a< b$ entonces $a+1<b+1$  por la propiedades de los números reales. Por tanto, si $a,b\in (-\infty,-1)$ y $a<b$ entonces $a+1<b+1$; y si $a,b\in [0,\infty)$ y $a<b$ entonces $a<b$. Además , si $a\in (-\infty,-1)$ y $b\in [0,\infty)$ entonces $a<-1<0\leq b$ y $ a+1<0\leq b$.  Por tanto, dados  $a,b\in (-\infty,-1)\cup [0,\infty)$ resulta $a<b\Rightarrow f(a)<f(b)$. Entonces $f$ preserva el orden. Por otro lado, si $y\geq 0$ entonces $y\in [0,\infty)$ y $f^{-1}(y)\in [0,\infty)$; y si $y\in (-\infty,0)$ y $f^{-1}(y)=y-1\in (-\infty,-1)$. Por tanto $f^{-1}(\mathbb{R})=f^{-1}([0,\infty))\cup f^{-1}((-\infty, 0)=(-\infty,-1)\cup [0,\infty)$. Entonces $f$ es sobreyectiva. Resulta que $(-\infty,-1)=(-\infty,-1)\cap X$ y $[0,\infty)=(-1/2,\infty)\cap X$ son abiertos de $X$ en la topología de subespacio sobre $\mathbb{R}$ y $(-\infty,-1)\cap [0,\infty)=\varnothing$, por tanto, $X$ no es conexo en la topología de subespacio. Pero $\mathbb{R}$ sí es conexo. Luego no hay aplicación homeomorfa entre $X$ y $\mathbb{R}$ con la topología de subespacio.
\section{Tema 3 Sección 24 Ejercicio 8}
\begin{itemize}
\item \bf (a) \rm Veamos si el producto de espacios conexos por caminos es también conexo por caminos o no.
\end{itemize}
Sean $f:[a,b]\rightarrow X$ y $g:[c,d]\rightarrow Y$ aplicaciones continuas y $X$ e $Y$ espacios topológicos conexos. Entonces $X$ e $Y$ son conexos por caminos. Veamos si $X\times Y$ es conexo por caminos. Veamos si existe una aplicación $h:[a,b]\rightarrow X\times Y$ que es continua. Sea $g':[a,b]\rightarrow [c,d]$ la aplicación definida por $g'(t)= \frac{(t-b)c+(a-t)d}{a-b}$. Además $g'^{-1}(t)=\frac{(a-b)t+bc-ad}{c-d}$ Entonces $g'$ es homeomorfismo entre $[a,b]$ y $[c,d]$ (ver ejercicio 5 de la sección 18). Entonces $(g\circ g'):[a,b]\rightarrow Y$ es continua. Por teorema 18.4, la función $h:[a,b]\rightarrow X\times Y$ definida por $h(x)=(f(x), g(g'(x))$ es continua si, y solo si, $f$ y $g\circ g'$ son continuas. Solo queda ver si el producto de dos espacios conexos es conexo. Por teorema 23.6, sabemos que el producto cartesiano finito de espacios conexos es conexo. Entonces $h:[a,b]\rightarrow X\times Y$ es continua y $X\times Y$ es conexo. Por tanto $X\times Y$ es conexo por caminos.
\begin{itemize}
\item \bf (b) \rm Veamos que si dado $A\subset X$ tal que $A$ es conexo por caminos, entonces $\overline{A}$ es conexo por caminos.
\end{itemize}
Del ejemplo 7, se deduce que $S$ es conexo por caminos, pero la curva del seno topológo $\overline{S}$ no es conexa por caminos.
\begin{itemize}
\item \bf (c) \rm Si $f:X\rightarrow Y$ es continua y $X$ es conexo por caminos. Veamos si $f(X)$ es conexo por caminos.
\end{itemize}
Como $X$ es conexo por caminos, existe una $g:[a,b]\rightarrow X$ continua para cada par de puntos  $x,y\in X$ tales que $g(a)=x$ y $g(b)=y$. Por tanto $(f\circ g):[a,b]\rightarrow Y$ también es continua. Por teorema 18.2(e), la retricción de recorrido $(f\circ g):[a,b]\rightarrow f(X)$ también es continua. Por tanto $(f\circ g)(a)= f(x)$ y $(f\circ g)(b)= f(y)$. Esto es, para cada par de puntos $w,z\in f(X)$ se tiene que $(f\circ g)(a)= f(x)=w$ y $(f\circ g)(a)= f(y)=z$
\begin{itemize}
\item \bf (d) \rm Sea $\left\{A_\alpha\right\}$ una colección de subespacios conexos por caminos de $X$ tales que $\bigcap_\alpha A_\alpha\neq \varnothing$. Veamos si $\bigcup_\alpha A_\alpha$ es conexo por caminos.
\end{itemize}
Sean cualesquiera $\alpha\neq \beta$. Entonces existen funciones continuas $f_\alpha:[a,b]\rightarrow A_\alpha$  y $f_\beta:[b,c]\rightarrow A_\beta$ tales que $f_\beta(b)=f_\alpha(b)\in A_\alpha\cap A_\beta$, para cuales quiera pares de puntos $f_\alpha (a),f_\alpha (b)\in A_\alpha$ y $f_\beta (b),f_\beta (c)\in A_\beta$. Por teorema 18.3, existe una función $g:[a,c]\rightarrow A_\alpha\cup A_\beta$ continua que une cualesquiera pares de puntos $g(a),g(c)\in A_\alpha\cup A_\beta$. Además, como $A_\alpha$, $A_\beta$ son conexos y como $A_\alpha\cap A_\beta=\varnothing$, por teorema 23.3 , $A_\alpha\cup A_\beta$ es conexo. Por tanto,  $A_\alpha\cup A_\beta$ es conexo por caminos. Por el mismo argumento, $\bigcup_\alpha A_\alpha$ es conexo por caminos.
\section{Tema 3 Sección 24 Ejercicio 9}
Sea $\mathbb{R}$ no numerable. Veamos que si $A$ es numerable en $\mathbb{R}^2$, entonces $\mathbb{R}^2-A$ es conexo por caminos.
Veamos cuantas líneas pasan por un punto de $\mathbb{R}^2$. Sean $a_n\times b_n\in A$ con $n\in \mathbb{Z}_+$. Las funciones continuas $f:[t_0,t_1]\rightarrow \mathbb{R}^2$ definidas por $f(t)=(x\times y)\frac{t_1-t}{t_1-t_0}+(w\times z)\frac{t-t_0}{t_1-t_0}$ unen los puntos $x\times y$ y $w\times z$ por una recta. Supongamos que existe un $a_n\times b_n$ tal que $f(t_n)=a_n\times b_n$. Por ejercicio 8 (d), podemos encontrar una función continua que une los puntos $x\times y$ y $w\times z$ a partir de dos rectas $f_n:[t_0,t_n]\rightarrow \mathbb{R}^2$ y $f_{n+1}:[t_n,t_1]\rightarrow \mathbb{R}^2$ dadas por $f_n(t)=(x\times y)\frac{t_n-t}{t_n-t_0}+((a_n+\epsilon)\times (b_n+\epsilon))\frac{t-t_0}{t_n-t_0}$ y $f_{n+1}(t)=((a_n+\epsilon)\times (b_n+\epsilon))\frac{t_1-t}{t_1-t_n}+(w\times z)\frac{t-t_n}{t_1-t_n}$, sin que pasen por $a_n\times b_n$. Puesto que hay un número incontable de $\epsilon\in \mathbb{R}$, el proceso se puede repetir para que el camino que une los puntos $x\times y$ y $w\times z$ sea una función continua que no pasa por ningún elemento de $A$. 
\section{Tema 3 Sección 24 Ejercicio 10}
Veamos que si $U$ es un subespacio abierto y conexo de $\mathbb{R}^2$, entonces $U$ es conexo por caminos.
Primero veamos que dado $x_0\in U$, el conjunto de los puntos que se unen a $x_0$ por un camino en $U$, es abierto y cerrado a la vez en $U$. Los caminos se definen como funciones continuas $f_x:[a,b]\rightarrow U$ tales que $f_x(a)=x$ y $f_x(b)=x_0$ con $x_0\in U$. Pero como $\mathbb{R}^2= \bigcup_{x\in \mathbb{R}^2}f_x([a,b])$ se tiene que $\mathbb{R}^2\cap U=\bigcup_{x\in \mathbb{R}^2}\left(f_x([a,b])\cap U\right)=\bigcup_{x\in U}f_x([a,b])$. Por tanto  $\bigcup_{x\in U}f_x([a,b])$ es abierto  y cerrado de $U$ a la vez. Por tanto, para todo $x\in U$ existe un camino $f_x:[a,b]\rightarrow U$ con $f_x(a)=x$ y $f_x(b)=x_0$
\end{document}
