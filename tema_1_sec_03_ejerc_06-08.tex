\documentclass{article}
% Uncomment the following line to allow the usage of graphics (.png, .jpg)
%\usepackage[pdftex]{graphicx}
% Comment the following line to NOT allow the usage of umlauts


% Start the document
\begin{document}

% Create a new 1st level heading
\section{Tema 1 Sección 3 Ejercicio 6}
Si \([y_0-x_{0}^{2}<y_1-x_{1}^{2}]\) o \([y_0-x_{0}^{2}=y_1-x_{1}^{2}\text{ y } x_0<x_1]\) entonces la relación en \(\mathbb{R}\times\mathbb{R}\) es \((x_0,y_0) <(x_1,y_1)\). Veamos que es una relación de orden. (Comparabilidad) Sea \((x_0, y_0)\neq(x_1,y_1)\), entonces \((x_0,y_0) <(x_1,y_1)\) o \((x_1,y_1) <(x_0,y_0)\). En efecto, porque de no cumplirse \((x_0,y_0) <(x_1,y_1)\), se tendría que \([y_0-x_{0}^{2}\geq y_1-x_{1}^{2}]\) y \([y_0-x_{0}^{2}\neq y_1-x_{1}^{2}\text{ o } x_0\geq x_1]\), o lo que es lo mismo, \([y_0-x_{0}^{2}\geq y_1-x_{1}^{2}\) y \(y_0-x_{0}^{2}\neq y_1-x_{1}^{2}]\text{ o } [y_0-x_{0}^{2}\geq y_1-x_{1}^{2}\) y \(x_0\geq x_1]\), por lo tanto \([y_0-x_{0}^{2} > y_1-x_{1}^{2}]\text{ o } [y_0-x_{0}^{2}\geq y_1-x_{1}^{2}\) y \(x_0\geq x_1] \Rightarrow [y_0-x_{0}^{2} > y_1-x_{1}^{2}]\text{ o } [[y_0-x_{0}^{2}= y_1-x_{1}^{2}\) y \(x_0 > x_1]\text{ o } [y_0-x_{0}^{2}= y_1-x_{1}^{2}\) y \(x_0 = x_1]]\Rightarrow  [y_0-x_{0}^{2} > y_1-x_{1}^{2}]\text{ o } [y_0-x_{0}^{2}= y_1-x_{1}^{2}\) y \(x_0 > x_1] \), ya que  \((x_0, y_0)\neq (x_1,y_1)\Rightarrow x_0\neq x_1 \text{ y } y_0\neq y_1\). Esto es, de no cumplirse \((x_0,y_0) <(x_1,y_1)\) , se cumple \((x_0,y_0) >(x_1,y_1)\). (No reflexividad) Veamos que ningún \((x,y)\) cumple que \((x,y)>(x,y)\). De lo contrario se tendría que \([y-x^{2}<y-x^{2}]\) o \([y-x^{2}=y-x^{2}\text{ y } x<x]\), es decir \(0<0\), lo cual es imposible. (Transitividad) Si \((x_0,y_0) <(x_1,y_1)\) y \((x_1,y_1) <(x_2,y_2)\) entonces \((x_0,y_0) <(x_2,y_2)\). De la definición se tiene que \([[y_0-x_{0}^{2}<y_1-x_{1}^{2}]\) o \([y_0-x_{0}^{2}=y_1-x_{1}^{2}\text{ y } x_0<x_1]] \text{ y }[[y_1-x_{1}^{2}<y_2-x_{2}^{2}]\) o \([y_1-x_{1}^{2}=y_2-x_{2}^{2}\text{ y } x_1<x_2]]\), por tanto \([y_0-x_{0}^{2}<y_2-x_{2}^{2}]\) o \([y_0-x_{0}^{2}=y_2-x_{2}^{2}\text{ y } x_0<x_2]]\). Luego \((x_0,y_0)<(x_2,y_2)\).

\section{Tema 1 Sección 3 Ejercicio 7}
Sea \(C\) relación sobre \(A\). La restricción de \(C\) sobre \(A_0\) se define como \(C\cap A_0\times A_0\). Como \(C=\{(x,y)\text{ tal que } x\in A \text{, } y \in A \text{ y } x \text{ relacionado con }y\} \text{ y } A_0 \subset A\), entonces \(C\cap A_0 \times A_0 =\{(x,y) \text{ es tal que } x\in A_0 \text{, } y \in A_0 \text{ y } x \text{ relacionado con } y\}\) ya que \((A\times B)\cap(C\times D)\Leftrightarrow (A \cap C)\times (B \cap D)\). Es decir, \(C\cap A_0 \times A_0\)  es una relación sobre \(A_0\). Supongamos que \(C\) es relación de orden. Veamos que \(C\cap A_0 \times A_0\) tambien es relación de orden. (Comparabilidad) Si para cualesquiera \(x\in A\) y \(y\in A\) con \(x\neq y\), entonces \(xCy\) o \(yCx\). Entonces, si \(x \in A_0 \subset A\) y \(y \in A_0 \subset A\) con \(x\neq y\), se tiene que \((x,y)\in \{(a,b)| a\in A_0, b\in A_0 \text{ y } aCb\} = C\cap A_0 \times A_0 \) o que \((y,x)\in \{(a,b)| a\in A_0, b\in A_0 \text { y } aCb\} = C\cap A_0 \times A_0 \) (No reflexividad). Supongamos que \((x,x)\in \{(a,b)| a\in A_0, b\in A_0 \text { y } aCb\} = C\cap A_0 \times A_0 \), pero \((x,x)\neq C\), por tanto  \((x,x)\in C\cap A_0 \times A_0 \) es falso. (Transitividad) Como \(x, y, z \in A_0\text{ y } [(x,y),(y,z)\in C \Rightarrow (x,z)\in C]\) se tiene que \([(x,y)\in C\cap A_0 \times A_0,(y,z)\in C\cap A_0 \times A_0 \Rightarrow (x,z)\in C\cap A_0 \times A_0]\) ,ya que \(C\cap A_0 \times A_0\subset C\).
\section{Tema 1 Sección 3 Ejercicio 8}
Decimos que \(xCy\) si \(x^2<y^2\), o \(x^2 = y^2\) y \(x<y\). (Comparabilidad) Veamos que si \(x\neq y\)  entonces \(xCy\) o \(yCx\). Como \(x\neq y\) entonces \( x^2 \neq y^2\), o \(x=-y\) y \(y^2=x^2\). Es decir, si \(x\neq y\) entonces \( x^2 < y^2\), o \( x^2 > y^2\), o \(x=-y>y\) y \(y^2=x^2\), o \(x=-y<y\) y \(y^2=x^2\). Es decir, si \(x\neq y\) entonces \( x^2 < y^2\), o \(x=-y<y\) y \(y^2=x^2\); o  \( x^2 > y^2\), o \(x=-y>y\) y \(y^2=x^2\). Por tanto, si \(x \neq y\) entonces \(xCy\) o \(yCx\). (No reflexivilidad) Veamos que no hay \(x\) tal que \(xCx\). Si fuera \(xCx\) entonces se tendría que \(x^2<x^2\), o \(x^2 = x^2\) y \(x<x\), pero ambas afirmaciones son falsas, luego no existe \(x\) tal que \(xCx\). (Transitividad). Veamos que si \(xCy\) y \(yCz\) entonces \(xCz\). Si \(xCy\) y \(yCz\) entonces [\( x^2 < y^2\), o \(x<y\) y \(y^2=x^2\)] y [\( y^2 < z^2\), o \(y<z\) y \(y^2=z^2\)]. Por tanto, si \(xCy\) y \(yCz\), entonces \( x^2 < y^2<z^2\), o \(x<y\) y \(z^2>y^2=x^2\), o \(y<z\) y \(x^2<y^2=z^2\), o \(x<y<z\) y \(x^2=y^2=z^2\). Es decir, si \(xCy\) y \(yCz\) entonces \(xCz\) o \(x<y \) y \(z^2>y^2=x^2\), o \(y<z\) y \(x^2<y^2=z^2\). Es decir, si \(xCy\) y \(yCz\) entonces  \(xCz\) o [\(x<y \text{ o } y<z \)] y \(z^2>x^2\), entonces  \(xCz\). Luego, es \(C\) es transitiva.
% Uncomment the following two lines if you want to have a bibliography
%\bibliographystyle{alpha}
%\bibliography{document}

\end{document}
