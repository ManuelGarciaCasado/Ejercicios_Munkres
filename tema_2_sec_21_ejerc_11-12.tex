\documentclass{article}
% Uncomment the following line to allow the usage of graphics (.png, .jpg)
%\usepackage[pdftex]{graphicx9990}o0y
% Comment the following line to NOT allow the usage ofp0 umlauts


\newcommand{\vect}[1]{\boldsymbol{#1}}
% Start the document00
\begin{document}
\section{Tema 2 Sección 21 Ejercicio 11}
\begin{itemize}
\item \bf (a) \rm Sea $(s_n)$ una sucesión acotada de números reales y $s_n\leq s_{n+1}$ para todo $n\in \mathbb{Z}_+$. Veamos que $(s_n)$ converge.
\end{itemize}
Si la sucesión $(s_n)$ está acotada, tendrá cotas superiores. Sea $m$ el mínimo de las cotas superiores. Entonces $m$ es el supremo de las $(s_n)$. Esto es $s_n\leq m$ para todo $n\in \mathbb{Z}_+$. Por tanto, $m-s_1\geq m- s_2\geq m- s_{n}\geq m- s_{n+1}$. Entonces, exite un $N$ tal que dado un $\epsilon>0$ se tiene que $d(s_{n+1},m)\leq d(s_{n},m)\leq d(s_{N},m)\leq \epsilon\leq...\leq d(s_{1},m)$ para todo $n\geq N$. Por tanto, dado un entorno $U=(m-\epsilon,m+\epsilon)$ de $m$, se tiene que todos los $s_n\in U$ para $n>N$. Por tanto, $(s_n)$ converge a $m$.
\begin{itemize}
\item \bf (b) \rm Sea $(a_n)$ una sucesión de números reales y sea $s_n=\sum_{i=1}^{n}a_n$ tal que si $s_n\rightarrow s$, la serie infinita $\sum_{i=1}^{\infty}a_n$ converge a $s$. veamos que si $\sum a_i$ converge a $s$ y $\sum b_i$ converge a $t$ entonces $\sum (a_i+c b_i)$ converge a $t+cb$.
\end{itemize}
Como
\begin{eqnarray}
\sum_{i=1}^n\left( a_i+c b_i\right)=\sum_{i=1}^na_i+ c \sum_{i=1}^nb_i,\nonumber
\end{eqnarray}
resulta
\begin{eqnarray}
\sum_{i=1}^\infty\left( a_i+c b_i\right)=\sum_{i=1}^\infty a_i+ c \sum_{i=1}^\infty b_i.\nonumber
\end{eqnarray}
Por tanto, $\sum_{i=1}^\infty\left( a_i+c b_i\right)$ converge a $s+ct$.
\begin{itemize}
\item \bf (c) \rm Veamos que si $|a_i|\leq b_i$ para cada $i$ y  $\sum b_i$ converge, entonces la serie $\sum a_i$ converge.
\end{itemize}

Supongamos que $\sum b_i$ converge a $t$. Entonces, se tiene que existe un $N$ tal que
\begin{eqnarray}
t\geq\sum^{n+1}_{i=1}b_i\geq \sum^{n+1}_{i=1}|a|_i\geq \sum^n_{i=1}|a|_i\nonumber
\end{eqnarray}
para todo $n\geq N$. Por apartado (a), $\sum |a_i|$ converge. Por otro lado, como $|a_i|+a_i\geq 0$ para todo $i$, se tiene que existe un $N$ tal que
\begin{eqnarray}
\sum^{n+1}_{i=1}2|a_i|\geq \sum^{n+1}_{i=1}(|a|_i+a_i)=|a_{n+1}|+a_{n+1}+\sum^{n}_{i=1}(|a|_i+a_i)\geq \sum^n_{i=1}(|a|_i+a_i)\nonumber
\end{eqnarray}
para todo $n\geq N$. Por apartado (a), $\sum (|a_i|+a_i)$ converge. Por tanto, como $\sum a_i=\sum (|a_i|+a_i-|a_i|)$. Aplicando apartado (b), se tiene que $\sum a_i=\sum (|a_i|+a_i)-\sum |a_i|$ converge a la suma de dos numeros reales.
\begin{itemize}
\item \bf (d) \rm Dada la sucesión de funciones $f_:X\rightarrow \mathbb{R}$, sea la sucesión
\begin{eqnarray}
s_n(x)=\sum^{n}_{i=1}f_i(x)\nonumber
\end{eqnarray}
Veamos que $(s_n)$ converge uniformemente a $s$ si $|f_i(x)|\leq M_i$ para todo $i$ y todo $x\in X$, y si la serie $\sum M_i$ converge.
\end{itemize}
Como  $|f_i(x)|\leq M_i$ y para todo $i$ y todo $x\in X$, y si la serie $\sum M_i$ converge, por lo visto en apartado (c), la serie $\sum f_i(x)$ converge. Sea $s(x)$ la función a la que $\sum f_i(x)$ converge. Veamos que dado un $\epsilon>0$, existe un $N$ tal que $|s_n(x)-s(x)|<\epsilon$   para todo $n\geq N$ y todo $x\in X$. Sea $r_n=\sum_{i=n+1}^{\infty}M_i$. Se tiene que 
\begin{eqnarray}
|s_n(x)-s(x)|=|\sum_{i=1}^{n}f_i(x)-\sum_{i=1}^{\infty}f_i(x)|=|\sum_{i=n+1}^{\infty}f_i(x)|\nonumber
\end{eqnarray}
y, por la desigualdad triangular, $|\sum_{i=n+1}^{\infty}f_i(x)|\leq \sum_{i=n+1}^{\infty}|f_i(x)|$. Como de las condiciones del problema se obtiene $\sum_{i=n+1}^{\infty}|f_i(x)|\leq \sum_{i=n+1}^{\infty}M_i$ y $r_{n+1}\leq r_n$. Entonces resulta que $|s_n(x)-s(x)|\leq r_n$. Entonces, cosiderando $r_N<\epsilon$ se tiene que $|s_n(x)-s(x)|<\epsilon$ para todo $n\geq N$ y todo $x\in X$.
\section{Tema 2 Sección 21 Ejercicio 12}
Veamos la continuidad de las propiedades algebraicas sobre $\mathbb{R}$. Sea la distancia $d(x,y)=|x-y|$ sobre $\mathbb{R}$ y la distancia $\rho((x,y),(x_0,y_0))=\max\{|x-y|,|x_0-y_0|\}$ sobre $\mathbb{R}^2$
\begin{itemize}
\item \bf (a) \rm Veamos que la adición es continua.
\end{itemize}
De la definicion y desigualdad triangular de distancia $d$ se tiene que $d(x+y,x_0+ y_0)=|x+y-x_0+y-y_0|\leq |x-x_0|+|y-y_0|$. Pero $|x-x_0|+|y-y_0|\leq 2 |x-x_0|$ si $|x-x_0|\geq |y-y_0|$; o $|x-x_0|+|y-y_0|\leq 2 |y-y_0|$ si $|x-x_0|\leq |y-y_0|$. Por tanto $d(x+y,x_0+ y_0)\leq |x-x_0|+|y-y_0|\leq 2\rho((x,y),(x_0,y_0))$. Luego, dado un $\epsilon>0$ y un $x_0\times y_0$ resulta
\begin{eqnarray}
d(x+y,x_0+ y_0)\leq 2\rho((x,y),(x_0,y_0))<\epsilon\nonumber\\
\rho((x,y),(x_0,y_0))<\frac{\epsilon}{2}\Rightarrow d(x+y,x_0+ y_0)<\epsilon\nonumber
\end{eqnarray}
para todo $x\times y$.
Ahora consideremos la adición como una función $+:\mathbb{R}\times \mathbb{R}\rightarrow \mathbb{R}$. Entonces aplicando el teorema 21.1, la adición es continua.
\begin{itemize}
\item \bf (b) \rm Veamos que la multiplicación es continua.
\end{itemize}
De la definicion y desigualdad triangular de distancia $d$ se tiene que
\begin{eqnarray}
 d(xy,x_0y_0)=|xy-x_0y_0|\nonumber
=|xy-x_0y_0-xy_0+xy_0|\nonumber\\
=|x_0(y-y_0)+y_0(x-x_0)+(x-x_0)(y-y_0)|\nonumber\\
\leq |x_0||y-y_0|+|y_0||x-x_0|+|x-x_0||y-y_0|\nonumber\\
\leq |x_0|\max\{|y-y_0|,|x-x_0|\}+|y_0|\max\{|y-y_0|,|x-x_0|\}\nonumber\\
+(\max\{|y-y_0|,|x-x_0|\})^2\nonumber\\
\leq \rho((x,y),(x_0,y_0))\left(|x_0|+|y_0|+\rho((x,y),(x_0,y_0))\right)\nonumber
\end{eqnarray}
Por tanto, si $\rho((x,y),(x_0,y_0))<\epsilon$ y $0< \epsilon<1$ entonces
\begin{eqnarray}
d(xy,x_0y_0)\leq \rho((x,y),(x_0,y_0))\left(|x_0|+|y_0|+\rho((x,y),(x_0,y_0))\right)<\epsilon\left(|x_0|+|y_0|+1\right);\nonumber
\end{eqnarray}
y si $\rho((x,y),(x_0,y_0))<\epsilon$ y $1\leq \epsilon$ entonces
\begin{eqnarray}
d(xy,x_0y_0)\leq \rho((x,y),(x_0,y_0))\left(|x_0|+|y_0|+\rho((x,y),(x_0,y_0))\right)<\epsilon^2\left(|x_0|+|y_0|+1\right)\nonumber
\end{eqnarray}

Luego, dado un $\epsilon>0$ y un $x_0\times y_0$ resulta que $\rho((x,y),(x_0,y_0))<\epsilon$ implica
\begin{eqnarray}
d(x+y,x_0+ y_0)<\begin{cases}\epsilon\left(|x_0|+|y_0|+1\right)  \text{ si }0<\epsilon<1\nonumber\\
\epsilon^2\left(|x_0|+|y_0|+1\right) \text{ si }\epsilon\geq 1\nonumber
\end{cases}
\end{eqnarray}
para todo $x\times y$.
Ahora consideremos la multiplicación como una función de $\mathbb{R}\times \mathbb{R}$ en $ \mathbb{R}$. Entonces aplicando el teorema 21.1, la multiplicación es continua.
\begin{itemize}
\item \bf (c) \rm Veamos que la operación de tomar inversos es una aplicación es continua de $\mathbb{R}-\{0\}$ en $\mathbb{R}$.
\end{itemize}
Sea $a,b\in\mathbb{R}-\{0\}$. Resulta que $b\geq a>0$ si, y solo si, $\frac{b}{a}\geq 1$ si, y solo si, $\frac{1}{a}\geq \frac{1}{b}$. De este modo, si $x\in \mathbb{R}-\{0\}$ y $\frac{1}{x}\in (\frac{1}{b},\frac{1}{a})$ entonces $x\in (a,b)$.

Sea $a,b\in\mathbb{R}-\{0\}$. Resulta que $0>b\geq a$ si, y solo si, $\frac{b}{a}\leq 1$ si, y solo si,$\frac{1}{a}\geq \frac{1}{b}$. De este modo, si $x\in \mathbb{R}-\{0\}$ y $\frac{1}{x}\in (\frac{1}{b},\frac{1}{a})$ entonces $x\in (a,b)$.

Sea $a,b\in\mathbb{R}-\{0\}$. Resulta que $a>0> b$ si, y solo si, $\frac{a}{b}\leq 1$ si, y solo si,$\frac{1}{b}\leq \frac{1}{a}$. De este modo, si $x\in \mathbb{R}-\{0\}$ y $\frac{1}{x}\in (-\infty,\frac{1}{a})\cup(\frac{1}{b},\infty)$ entonces $x\in (a,b)-\{0\}$. 

Sea $a\in\mathbb{R}-\{0\}$. Resulta que $a>0$ si, y solo si, $\frac{1}{a}> 0$. De este modo, si $x\in \mathbb{R}-\{0\}$ y $\frac{1}{x}\in (\frac{1}{a},\infty,)$ entonces $x\in (0,a)$.

Sea $a\in\mathbb{R}-\{0\}$. Resulta que $0> a$ si, y solo si, $\frac{1}{a}< 0$. De este modo, si $x\in \mathbb{R}-\{0\}$ y $\frac{1}{x}\in (-\infty,\frac{1}{a})$ entonces $x\in (a,0)$.

Por tanto, como para todo $x\in\mathbb{R}-\{0\}$ existe un entorno $V$ de $\frac{1}{x}$ tal que $\frac{1}{U}\subset V$ es un entorno de $x$. Por teorema 18.1, la inversa es una función continua.
\begin{itemize}
\item \bf (d) \rm Veamos que la operación de sustracción y cociente son funciones continuas.
\end{itemize}
La operación de sustracción es la composición de la operación de suma con la de cambio de signo. Dado que $-x\in (-b,-a)\subset \mathbb{R}$ si, y solo si, $x\in (a,b)$, la operación de cambio de signo es continua. Como la composición de dos funciones continuas es continua (vease teorema 18.2 (c)), la sustracción es continua.

La operación de cociente es la composición de la operación de producto con la de inversa de uno de los factores. Dado que $ x\in \mathbb{R}$ e $y\in \mathbb{R}-\{0\}$ se tiene $x\times y \in \mathbb{R}\times (\mathbb{R}-\{0\})$ y $x\cdot \frac{1}{y}\in \mathbb{R}$. Como la composición de dos funciones continuas es continua (vease teorema 18.2 (c)), la operación de cociente es continua.



\end{document}
