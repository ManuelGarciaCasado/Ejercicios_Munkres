\documentclass{article}
% Uncomment the following line to allow the usage of graphics (.png, .jpg)
%\usepackage[pdftex]{graphicx9990}o0y
% Comment the following line to NOT allow the usage ofp0 umlauts


\newcommand{\vect}[1]{\boldsymbol{#1}}
% Start the document00
\begin{document}
\section{Tema 1 Sección 1 Ejercicio 1}
Probemos las leyes de morgan
\begin{itemize}
\item $A\cap(B\cup C)=(A\cap B)\cup (A\cap C)$
\end{itemize}
Se tiene que $A\cap(B\cup C)=\{x|x\in A \text{ y } x\in B\cup C\}$. Por tanto, $A\cap(B\cup C)=\{x|x\in A \text{ y además }  x\in B \text{ o }x\in C\}$. Es decir $A\cap(B\cup C)=\{x|x\in A \text{ y }  x\in B \text{, o }x\in C\text{ y }  x\in B\}=(A\cap B)\cup (A\cap C)$
\begin{itemize}
\item $A\cap(B\cup C)=(A\cap B)\cup (A\cap C)$
\end{itemize}
Se tiene que $A\cup(B\cap C)=\{x|x\in A \text{ o } x\in B\cap C\}$. Por tanto, $A\cup(B\cap C)=\{x|x\in A \text{ o }  x\in B \text{ y }x\in C\}$. Es decir $A\cap(B\cup C)=\{x|x\in A \text{ o }  x\in B \text{, y }x\in C\text{ o }  x\in B\}=(A\cup B)\cap (A\cup C)$
\begin{itemize}
\item $A-(B\cup C)=(A-B)\cap (A-C)$
\end{itemize}
Se tiene que $A-(B\cup C)=\{x|x\in A \text{ y } x\notin B\cup C\}$. Por tanto, $A-(B\cup C)=\{x|x\in A \text{ y, }  x\notin B \text{ y }x\notin C\}$. Es decir $A-(B\cup C)=\{x|x\in A \text{ y }  x\notin B \text{, y }x\in A\text{ y }  x\notin C\}=(A-B)\cap (A-C)$
\begin{itemize}
\item $A-(B\cap C)=(A-B)\cup (A-C)$
\end{itemize}
Se tiene que $A-(B\cap C)=\{x|x\in A \text{ y } x\notin B\cap C\}$. Por tanto, $A-(B\cap C)=\{x|x\in A \text{ y, }  x\notin B \text{ o }x\notin C\}$. Es decir $A-(B\cap C)=\{x|x\in A \text{ y }  x\notin B \text{, o }x\in A\text{ y }  x\notin C\}=(A-B)\cup (A-C)$
\section{Tema 1 Sección 1 Ejercicio 2}
\begin{itemize}
\item \bf (a) \rm Veamos si  $A\subset B\text{ y }A\subset C \iff A\subset(B\cup C)$ es verdad.
\end{itemize}
Si $A\subset B$ y $A\subset C$ es lo mismo que $ x\in A\Rightarrow x\in B \text{ y } x\in C$. Luego $A\subset B\cap C$. Como $B\cap C \subset B\cup C$, se tiene que $A\subset( B\cap C) \subset (B\cup C)$. Se concluye que si $A\subset B$ y $A\subset C$ entonces $A\subset (B\cup C)$.

Supongamos que $A\subset (B\cup C)$. Esto es $x\in A\Rightarrow x\in B \text{ o } x\in C$. Por tanto, $x\in A\Rightarrow x\in B$ o $x\in A\Rightarrow x\in C$. Por tanto, $A\subset B$ o $A\subset C$. Se concluye que $A\subset (B\cup C)\Rightarrow A\subset B\text{ o }A\subset C$. Pero $A\subset (B\cup C)\nRightarrow A\subset B\text{ y }A\subset C$ 
\begin{itemize}
\item \bf (b) \rm Veamos si  $A\subset B\text{ o }A\subset C \iff A\subset(B\cup C)$ es verdad.
\end{itemize}
En apartado (a) se ha visto que $A\subset (B\cup C)\Rightarrow A\subset B\text{ o }A\subset C$. Supongamos ahora que $A\subset B\text{ o }A\subset C$. Por tanto, $x\in A\Rightarrow x\in B\text{ o }x\in A\Rightarrow x\in C$. Es decir $x\in A\Rightarrow x\in B\text{ o } x\in C$. Entonces $x\in A\Rightarrow x\in (B\cup C)$, es decir $A\subset (B\cup C)$
\begin{itemize}
\item \bf (c) \rm Veamos si  $A\subset B\text{ y }A\subset C \iff A\subset(B\cap C)$ es verdad.
\end{itemize}
Si $A\subset B$ y $A\subset C$ es lo mismo que $ x\in A\Rightarrow x\in B \text{ y } x\in C$. Luego $A\subset B\cap C$. Se concluye que si $A\subset B$ y $A\subset C$ entonces $A\subset (B\cap C)$.

Sea $A\subset (B\cap C)$. Entonces $x\in A\Rightarrow [x\in B\text{ y } x\in C]$. Por tanto $[x\in A\Rightarrow x\in B]\text{ y } [x\in A\Rightarrow x\in C]$. Por tanto $A\subset (B\cap C)\Rightarrow A\subset B \text{ y } A\subset C$
\begin{itemize}
\item \bf (d) \rm Veamos si  $A\subset B\text{ o }A\subset C \iff A\subset(B\cap C)$ es verdad.
\end{itemize}
Si $A\subset B$ o $A\subset C$ es lo mismo que $ x\in A\Rightarrow x\in B \text{ o } x\in C$. Luego $A\subset B\cup C$.  Se concluye que si $A\subset B$ o $A\subset C$ entonces $A\subset (B\cup C)$. Luego $A\subset B\text{ o }A\subset C \nRightarrow  A\subset(B\cap C)$

Si $A\subset (B\cap C)$, $A\subset (B\cup C)$ y, por apartado (b), $A\subset (B\cup C)\Rightarrow A\subset B \text{ o } A\subset C$. Luego $A\subset (B\cap C)\Rightarrow A\subset B \text{ o } A\subset C$

\begin{itemize}
\item \bf (e) \rm Veamos si  $A-(A-B)=B$ es verdad.
\end{itemize}
Sea $x\in A-(A-B)$. Entonces $x\in A$ y $x\notin A-B$. Como $x\notin A-B\Rightarrow x\notin A\text{ o }x\in B$. Por tanto $x\in A-(A-B)\Rightarrow [x\in A]\text{ y } [x\notin A \text{ o } x\in B]$. Luego $x\in A-(A-B)\Rightarrow x\in A\cap B\subset B$. Esto es $A-(A-B)\subset B$

Sea $x\in A\cap B$. Entonces $x\in A\text{ o } x\notin A$, o $x\in A\text{ y } x\in B$. Luego $x\in A$ y $x\notin A-B$. Es decir $x\in A-(A-B)$. Luego $A\cap B\subset A-(A-B)$, pero $A\cap B\subset B$
\begin{itemize}
\item \bf (f) \rm Veamos si  $A-(B-A)=A-B$ es verdad.
\end{itemize}
Sea $x\in A-(B-A)$. Entonces $x\in A\text{ y, }x\notin B \text{ o } x \in A$. Luego $x\in  A\cup (A-B)$. Como $A-B\subset A$, resulta $A-(B-A)\subset A$

Sea $x\in A$. Entonces, $x\in A-B$ o $x\in B\cap A$. Entonces $x\in A$ y $x\notin B$ o, $x\in B$ y $x\in A$. Luego $x\in A$ y $x\notin B-A$. Luego $A\subset A-(B-A)$. Por tanto $A-B\subset A\subset A-(B-A)$.
\begin{itemize}
\item \bf (g) \rm Veamos si  $A\cap(B-C)=(A\cap B)-(A\cap C)$ es verdad.
\end{itemize}
Sea $x\in A\cap (B-C)$. Entonces $x\in A$ y $x\in B$ y $x\notin C$. Por tanto, $x\in A\cap B$ y $x\notin C$. Luego, $x\in (A\cap B)-C$. Como $x\notin C\Rightarrow x\notin C\cap A$, se tiene $A\cap (B-C)\subset (A\cap B)-(A\cap C)$.

Supongamos que $x\in (A\cap B)-(A\cap C)$. Entonces $x\in A$ y $x\in B$ y, $x\notin A$ o $x\notin C$. Por tanto, $x\in A$ y $x\in B$ y $x\notin C$. Luego $x\in A\cap (B-C)$. Entonces $(A\cap B)-(A\cap C)\subset A\cap (B-C)$.
\begin{itemize}
\item \bf (h) \rm Veamos si  $A\cup(B-C)=(A\cup B)-(A\cup C)$ es verdad.
\end{itemize}
Sea $x\in A\cup (B-C)$. Entonces $x\in A$ o $x\in B$ y $x\notin C$. Por tanto, $x\in A\cup B$ y, $x\notin C$ o $x\in A$. Luego, $x\in (A\cup B)-(C-A)$. Como $C-A\subset C\cup A$, se tiene que $x\notin C\cup A\Rightarrow x\notin C-A$. Luego $(A\cup B)-(C\cup A)\subset (A\cup B)-(C-A)$.

Supongamos que $x\in (A\cup B)-(A\cup C)$. Entonces $x\in (A\cup B)-(C-A)$. Por tanto, $x\in A$ o $x\in B$ y, $x\notin C$ o $x\in A$. Luego $x\in A\cup (B-C)$. Entonces $(A\cup B)-(A\cup C)\subset A\cup (B-C)$.
\begin{itemize}
\item \bf (i) \rm Veamos si  $(A\cap B)\cup(A-B)=A$ es verdad.
\end{itemize}
Como $x\in (A\cup B)\cup (A-B)$ si, y solo si, [$x\in A$ y $x\in B$] o [$x\in A $ y $x\notin B$] si, y solo si, $x\in A$ y [$x\in B$ o $x\notin B$] si, y solo si, $x\in A$. Luego $(A\cap B)\cup(A-B)=A$
\begin{itemize}
\item \bf (j) \rm Veamos si  $A\subset C\text{ y }B\subset D \Rightarrow( A\times B)\subset(C\times  D)$ es verdad.
\end{itemize}
Como $A\times B=\{(a,b)|a\in A\text{ y } b\in B\}$ y $a\in A\subset C$ y $b\in B\subset D$, se tiene $x\times y\in A\times B$ implica que $x\times y\in\{(a,b)|a\in C\text{ y } b\in D\}=C\times D$. Luego $A\subset C$ y $B\subset D$ implica $( A\times B)\subset(C\times  D)$
\begin{itemize}
\item \bf (k) \rm Veamos si  $A\subset C\text{ y }B\subset D \Leftarrow( A\times B)\subset(C\times  D)$ cuando $A,B\neq \varnothing$ es verdad.
\end{itemize}
Sea $( A\times B)\subset(C\times  D)$. Entonces $(x,y)\in A\times B\Rightarrow (x,y)\in C\times D$. Luego $(x,y)\in\{(a,b)|a\in A\text{ y } b\in B\}$ implica $(x,y)\in\{(a,b)|a\in C\text{ y } b\in D\}$. Por tanto, $x\in A$ e $y\in B$ implica $x\in C$ e $y\in D$ cuando $A\neq \varnothing $ y $B\neq \varnothing$, ya que $A= \varnothing $ o $B= \varnothing$ implica $A\times B=\varnothing$. Por tanto $( A\times B)\subset(C\times  D)\Rightarrow A\subset C \text{ y }B\subset D$ cuando $A\neq \varnothing $ y $B\neq \varnothing$.

\begin{itemize}
\item \bf (m) \rm Veamos si  $(A\times B )\cup (C\times D)=(A\cup C)\times ( B\cup D)$ es verdad.
\end{itemize}
Sea $(x,y)\in (A\times B )\cup (C\times D)$. Esto es si, y solo si, $(x,y)\in (A\times B)$ o $ (x,y)\in(C\times D)$ si, y solo si, $(x,y)\in \{(a,b)|a\in A\text{ y } b\in B\}$ o $(x,y)\in \{(a,b)|a\in C\text{ y } b\in D\}$ si, y solo si, $(x,y)\in \{(a,b)|[a\in A\text{ o } a\in C]\text{ y } [b\in B\text{ o } b\in D]\}$ si,y solo si, $(x,y)\in \{(a,b)|[a\in A\cup C\text{ y } b\in B\cup D\}=(A\cup C)\times ( B\cup D)$
\begin{itemize}
\item \bf (n) \rm Veamos si  $(A\times B )\cap (C\times D)=(A\cap C)\times ( B\cap D)$ es verdad.
\end{itemize}
Sea $(x,y)\in (A\times B )\cap (C\times D)$. Esto es si, y solo si, $(x,y)\in (A\times B)$ y $ (x,y)\in(C\times D)$ si, y solo si, $(x,y)\in \{(a,b)|a\in A\text{ y } b\in B\}$ y $(x,y)\in \{(a,b)|a\in C\text{ y } b\in D\}$ si, y solo si, $(x,y)\in \{(a,b)|[a\in A\text{ y } a\in C]\text{ y } [b\in B\text{ y } b\in D]\}$ si,y solo si, $(x,y)\in \{(a,b)|[a\in A\cap C\text{ y } b\in B\cap D\}=(A\cap C)\times ( B\cap D)$
\begin{itemize}
\item \bf (o) \rm Veamos si  $A\times ( B-C )= (A \times B) -( A\times C)$ es verdad.
\end{itemize}
Sea $(x,y)\in A\times (B - C)$. Esto es si, y solo si, $(x,y)\in \{(a,b)|a\in A\text{ y } b\in B\text{ y } b\notin C\}$ si, y solo si, $(x,y)\in \{(a,b)|a\in A\text{ y } b\in B\text{ y } b\notin C\}\cup \varnothing$ si, y solo si, $(x,y)\in \{(a,b)|[a\in A\text{ y } b\in B\text{ y } b\notin C] \text{ o }[a\in A\text{ y }a\notin A\text{ y } b\in B]\}$ si,y solo si, $(x,y)\in \{(a,b)|[a\in A\text{ y } b\in B]\text{ y } [a\notin A \text{ o }b\notin C]\}$ si, y solo si, $(x,y)\in \{(a,b)|(a,b)\in A\times B\text{ y } (a,b)\notin A\times C\}$ si y solo si, $(x,y)\in A\times B-A\times C$
\begin{itemize}
\item \bf (p) \rm Veamos si  $(A-  B)\times (C-D )= (A \times C- B\times C)-A\times D$ es verdad.
\end{itemize}
Se tiene que $(x,y)\in(A \times C- B\times C)-A\times D$ si, y solo si, $(x,y)\in\{(a,b)| [a\in A \text{ y } b\in C]\text{ y }[a\notin B \text{ o }b\notin C]\text{ y }[a\notin A\text{ o }b\notin D]\}$ si, y solo si, $(x,y)\in\{(a,b)| a\in A \text{ y } b\in C \text{ y }a\notin B \text{ y }b\notin D\}$ si, y solo si, $(x,y)\in (A-B)\times (C-D)$.
\begin{itemize}
\item \bf (q) \rm Veamos si $(A\times B )- (C\times D)=(A-C)\times ( B- D)$ es verdad.
\end{itemize}
Por apartado (p), $(A-C)\times ( B- D)=(A \times B- C\times B)-C\times D$ Pero $(A \times B- C\times B)\subset A\times B$. Por tanto, $(A\times B )- (C\times D)\subset (A-C)\times ( B- D)$
\section{Tema 1 Sección 1 Ejercicio 3}
\begin{itemize}
\item \bf (a) \rm Veamos el recíproco y el contrarrecíproco del enunciado: "si $x<0$ entonces $x^2-x>0$."
\end{itemize}
El enunciado es verdadero, puesto que $x<0\Rightarrow -x>0 \Rightarrow x^2-x>x^2>0$

Recíproco: "Si $x^2-x>0$ entonces $x<0$."; esto es falso, puesto que $2^2-2=2>0$ verdadero pero $2<0$ es falso.

Contrarecíproco: "Si $x^2-x<0$ entonces $x>0$."; esto es verdadero, puesto que que el contrarrecíproco de una verdad es verdad.
\begin{itemize}
\item \bf (b) \rm Veamos el recíproco y el contrarrecíproco del enunciado: "si $x>0$ entonces $x^2-x>0$."
\end{itemize}
El enunciado es falso, porque $1>x>0$ implica $x>x^2$ implica $x^2-x<0$.

Recíproco: "Si $x^2-x>0$ entonces $x>0$."; esto es falso, puesto que $1^2-1=0$.

Contrarecíproco: "Si $x^2-x>0$ entonces $x<0$."; esto es falso, puesto que que el contrarrecíproco de una falsedad es falso.

\end{document}
