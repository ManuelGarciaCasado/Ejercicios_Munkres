\documentclass{article}
% Uncomment the following line to allow the usage of graphics (.png, .jpg)
%\usepackage[pdftex]{graphicx}
% Comment the following line to NOT allow the usage of umlauts



% Start the document
\begin{document}

% Create a new 1st level heading
\section{Tema 1 Sección 4 Ejercicio 02}
% Create a new 1st level heading
\begin{itemize}
\item \bf (a) \rm 
\end{itemize}
Veamos que \(x>y\text{ y }w>z\Longrightarrow x+w>y+z\). Por (6) se tiene que \(x>y\text{ y }w>z\Longrightarrow x+w>y+w\text{ y }w+y>z+y\). Por (2), \(x>y\text{ y }w>z\Longrightarrow x+w>y+w\text{ y }y+w>z+y\). Por tanto \(x>y\text{ y }w>z\Longrightarrow x+w>z+y\)
\begin{itemize}
\item \bf (b) \rm 
\end{itemize}
Veamos que \(x>0\text{ y }y>0\Longrightarrow x+y>0 \text{ y }x\cdot y>0\). Por (6) se tiene que \(x>0\Longrightarrow x+y>y\). Por tanto, \(x+y>y\text{ y }y>0\Longrightarrow x+y>0\). Por (6), \(x>0\text{ y }y>0\Longrightarrow x\cdot y >0\cdot y\). Por ejercicio 1(b),  \(x>0\text{ y }y>0\Longrightarrow x\cdot y >0\cdot y=0\). Luego \(x>0\text{ y } y>0\Longrightarrow x\cdot y >0\).
\begin{itemize}
\item \bf (c) \rm 
\end{itemize}
Veamos que \(x>0 \Longleftrightarrow -x<0 \). Por (6), \(x>0 \Longleftrightarrow   x+\left(-1\right)\cdot x>\left(-1\right)\cdot x\). Luego por ejercicio 1(f), \(x>0 \Longleftrightarrow x-x>-x\)
 Luego por definicion y por (4), \(x>0 \Longleftrightarrow 0>-x\). Por definicion, \(x>0 \Longleftrightarrow -x<0\).
\begin{itemize}
\item \bf (d) \rm 
\end{itemize}
Veamos que \(x>y \Longleftrightarrow -x<-y\). Por (6), \(x>y\Longleftrightarrow   x+\left(-1\right)\cdot x>y+\left(-1\right)\cdot x\). Luego por ejercicio 1(f), \(x>y\Longleftrightarrow x-x>y-x\)
 Luego por definicion y por (4), \(x>y\Longleftrightarrow 0>y-x\). Por (6), \(x>y\Longleftrightarrow 0+(-1)y>(y-x)+(-1)y\).Por (3) y por definicion, \(x>y\Longleftrightarrow -y>(y-x)-y\). Por (2), \(x>y\Longleftrightarrow -y>-y+(y-x)\). Por (1), \(x>y\Longleftrightarrow -y>(-y+y)-x\). Por (4), \(x>y\Longleftrightarrow -y>-x\). Por definicion \(x>y\Longleftrightarrow -x<-y\)
\begin{itemize}
\item \bf (e) \rm 
\end{itemize}
Veamos que \(x>y\text{ y }z<0 \Longrightarrow zx<yz\). Por ejercicio (d) y (c), \(x>y \text{ y }z<0\Longrightarrow -y>-x\text{ y }-z>0\). Por propiedad (6), \(x>y \text{ y }z<0 \Longrightarrow \left(-z\right)\cdot\left(-y\right)>\left(-z\right)\cdot\left(-x\right)\). Por ejercicio 1(e) y 1(d), se tiene \(x>y \text{ y }z<0 \Longrightarrow zy>zx\). Por definicion \(zy>zx\Longleftrightarrow zx<zy\). Luego \(x>y \text{ y }z<0 \Longrightarrow zx<zy\).
\begin{itemize}
\item \bf (f) \rm 
\end{itemize}
Veamos que \(x\neq 0 \Rightarrow x^2>0\) donde \(x^2=x\cdot x\). Si \(x>0\), entonces por ejercicio (e) \(-x \cdot x<-x\cdot 0\). Por ejercicio 1(b),  \(-x \cdot x<0\). Luego por (c), \(x\cdot x >0\). Si \(x>0\), entonces por (6), \(x\cdot x>x \cdot 0\). Por ejercicio 1(b), \(x\cdot x>0\) . Luego si \(x>0 \text{ o } x<0\Rightarrow x^2>0\)
\begin{itemize}
\item \bf (g) \rm 
\end{itemize}
Veamos que \(-1<0<1\). Supongamos que \(x>0 \text{ y }1<0\) entonces, por (6) \(x\cdot 1 <0\cdot 1\). Por (3) y ejercicio 1(b), \(x=x\cdot 1 <0\cdot 1=0\). Por tanto \(x<0\), que contradice la afirmación inicial. Por tanto, \(1>0\) y por ejercicio (c), \(-1<0\).
\begin{itemize}
\item \bf (h) \rm 
\end{itemize}
Veamos que si \(x\cdot y>0\) entonces o \(x>0 \text{ e } y>0\), o \(x<0 \text{ e } y<0\).  
Por ejercicio (e), cuando \(x=0\), se tiene que \(0>y\text{ y } y z<0\Rightarrow 0\cdot z<yz\). Por ejercicio 1(b) \(0>y\text{ y } y z<0\Rightarrow 0<yz\). Por tanto \(x<0\text{ y } e y<0\Rightarrow 0<yx\)  Por ejercicio (6), cuando \(z=0\),\(x>z\text{ e }y>0\Rightarrow x\cdot y > x\cdot z\) entonces \(x>0\text{ e }y>0\Rightarrow x\cdot y > x\cdot 0\). ejercicio 1(b) \(x>0\text{ e }y>0\Rightarrow x\cdot y > 0\). Por otro lado, para demostrar \(xy>0\Rightarrow \text{ o } x>0 \text{ e } y>0, \text{ o } x<0 \text{ e } y<0 \) es lo mismo que demostrar que \( \text{ o } x>0 \text{ e } y<0, \text{ o } x<0 \text{ e } y>0 \Rightarrow xy<0\). Por ejercicio (e),  \(x<z\text{ e } y>0 \Rightarrow xy<zy\). Cuando \(z=0\), se tiene  \(x<0\text{ e } y>0 \Rightarrow xy<0\) Lo mismo ocurre intercambiando \(x\) por \(y\), y usando (2).
\begin{itemize}
\item \bf (i) \rm 
\end{itemize}
Veamos que \(x>0\Rightarrow 1/x>0\). Por ejercicio (g), \(1>0\). Por definición, \(x\cdot 1/x=1\) Luego  \(x\cdot 1/x>0\) Por tanto, como por ejercicio (h), o \(x>0 \text{ y } 1/x>0\) o \(x<0 \text{ y } 1/x<0\), se tiene que \(1/x>0\).
\begin{itemize}
\item \bf (j) \rm 
\end{itemize}
Veamos que \(x>y>0 \Rightarrow 1/y>1/x>0\). Como \(x>0\) e \(y>0\), se tiene por (6), que \(x>y>0 \Rightarrow x\cdot 1/y>y\cdot1/y\). Por tanto \(x>y>0 \Rightarrow x\cdot 1/y>1\). Entonces \(x>y>0 \Rightarrow 1/x\cdot\left(x\cdot 1/y\right)>\left(1/x\right)\cdot 1\). Por (1) y (3), \(x>y>0 \Rightarrow \left(1/x\cdot x\right)\cdot 1/y>\left(1/x\right)\cdot 1=1/x\). Por tanto \(x>y>0 \Rightarrow 1/y>1/x\).
\begin{itemize}
\item \bf (k) \rm 
\end{itemize}
Veamos que \(x<y\Rightarrow x<\left(x+y\right)/2<y\). Por un lado \(x<y\Rightarrow y\cdot1/2<x\cdot1/2\Rightarrow x/2<y/2\), por otro lado \(x<y\Rightarrow x<\left(x+x\right)/2\), luego \(x<y\Rightarrow x<\left(x+x\right)/2<\left(x+y\right)/2\). Luego \(x<y\Rightarrow x<\left(x+y\right)/2<\left(y+y\right)/2\). Por tanto \(x<y\Rightarrow x<\left(x+y\right)/2<y\).  
% Uncomm\(x>0\text{ e }y>0\Rightarrow x\cdot y > x\cdot 0\)ent the following two lines if you want to have a bibliography
%\bibliographystyle{alpha}
%\bibliography{document}
\end{document}
