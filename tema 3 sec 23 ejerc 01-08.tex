\documentclass{article}
% Uncomment the following line to allow the usage of graphics (.png, .jpg)
%\usepackage[pdftex]{graphicx9990}o0y
% Comment the following line to NOT allow the usage ofp0 umlauts

\usepackage[utf8]{inputenc}
\usepackage{amsmath}
\usepackage{amssymb}

\newcommand{\vect}[1]{\boldsymbol{#1}}
% Start the document00
\begin{document}
\section{Tema 2 Sección 23 Ejercicio 1}
Sean $\mathcal{T}$ y $\mathcal{T}'$ dos topologías en $X$. Si $\mathcal{T}'\supset \mathcal{T}$, veamos qué se puede decir de la conexión de $X$ respecto de una topología y respecto de la otra. Supongamos que $X$ es separable en $A$ y $B$ en la topología de $\mathcal{T}$. Entonces, como $\mathcal{T}'\supset \mathcal{T}$, resulta que $A\in \mathcal{T}'$ y $B\in \mathcal{T}'$. Como $A\cap B=\varnothing $, $A\cup B=X$, y $X\in \mathcal{T}'$, entonces $X$ es separable en $\mathcal{T}'$. Del mismo modo, si $X$ es conexo en la topología $\mathcal{T}'$ entonces es conexo en la topología $\mathcal{T}$.
\section{Tema 2 Sección 23 Ejercicio 2}
Dada la sucesion $\{A_n\}$ de subespacios conexos de $X$ tales que $A_n\cap A_{n+1}\neq \varnothing$ para cada $n$, veamos que $\bigcup_{n\in \mathbb{Z}} A_n$ es conexo. Por teorema 23.3, dado que $A_1\cap A_2\neq \varnothing$ y que $A_1$ y $A_2$ son conexos, $A_1\cup A_2$ es conexo. Ahora supongamos que $\bigcup_{i=1}^nA_i$ es conexo, que $A_{n+1}$ también es conexo y que $(\bigcup_{i=1}^nA_i)\cap A_{n+1}\neq \varnothing$. Entonces $\bigcup_{i=1}^{n+1}A_i$ es conexo. Por el teorema de inducción fuerte 4.2, $\bigcup_{n\in \mathbb{Z}}A_n$ es conexo.
\section{Tema 2 Sección 23 Ejercicio 3}
Dados la sucesion $\{A_\alpha\}$ de subespacios conexos de $X$ y el subespacio conexo $A$ de $X$, veamos que si $A_\alpha\cap A\neq \varnothing$ para cada $\alpha$, entonces $A\cup(\bigcup_\alpha A_\alpha)$ es conexo. Por teorema 23.3, $A_\alpha\cup A$ es conexo para cada $\alpha$. Supongamos que $Y=A\cup(\bigcup_\alpha A_\alpha)$ no es conexo. Sea la separación de $Y=B\cup C$. Entonces  $A$ está contenido, bien en $B$, bien en $C$. Supongamos que $A\subset B$. Como $A\cup A_\alpha$ es conexo, ya $A\cup A_\alpha\subset B$, ya $A\cup A_\alpha\subset C$. Esta última posibilidad se descarta porque $A\subset A\cup A_\alpha$ y $A\subset B$. Luego para $A\cup A_\alpha\subset B$ todo $\alpha$. Luego $A\cup (\bigcup_\alpha A_\alpha)=\bigcup_\alpha (A\cup A_\alpha)\subset B$ y $C=\varnothing$, contradiciendo la hipótesis de que $Y$ es separable.
\section{Tema 2 Sección 23 Ejercicio 4}
Veamos que si $X$ es un conjunto infinito, entonces $X$ es conexo con la topología de los complementos finitos. Por definición, $U$ es abierto si $X-U$ es finito o es todo $X$. Supongamos que $X=Y\cup Z$ es una separación. Entonces $Y$ es abierto y $Z$ es abierto. Por tanto $X-Z=Y$ es finito y $X-Y=Z$ es finito. Pero la unión de finitos $Y\cup Z$ es finita, contradiciendo el hecho de que $X$ es finito.
\section{Tema 2 Sección 23 Ejercicio 5}
Un espacio es totalmente disconexo si los únicos subespacios conexos son unipuntuales. Veamos que si $X$ tiene la topología discreta, entonces es totalmente disconexo. Supongamos que $X$ tiene la topologia discreta e $Y$ es un subespacio conexo con varios elementos. Sea $a\in Y$. Entonces $\{a\}\cap Y=\{a\}$ es abierto en $X$ y $(Y-\{a\})\cap Y=Y-\{a\}$ tambien es abierto en $X$, por ser un subconjunto perteneciente a la topología discreta. Pero $Y-\{a\}$ y $\{a\}$ son disjuntos tales que $Y=(Y-\{a\})\cup\{a\}$. Por tanto, $Y$ no es conexo. Por tanto, si $Y$ es conexo entonces no puede tener varios elementos.

Contrariamente, supongamos que los únicos subespacios conexos son unipuntuales en cierta topología. Veamos que la topología es la topología discreta. Si $a,b\in X$ y $\{a\}$ y $\{b\}$ son conexos, $Y=\{a\}\cup\{b\}$ es una separación, entonces $\{a\}$ es abierto en la topología del subespacio $Y$ y $\{a\}=\{a\}\cap Y$ es abierto en la topología de $X$. Por tanto $\{b\}$ y $\{a,b\}=\{a\}\cup\{b\}$, también son abiertos. Asi se tiene que cualquier abierto se construye con $U=\cup_{x\in U}\{x\}$ donde $U$ es cualquier subconjunto de $X$ y la topología resultante es la topología discreta.

\section{Tema 2 Sección 23 Ejercicio 6}
Sea $A\subset X$. Veamos que si $C$ es un subespacio conexo en $X$ que interseca tanto a $A$ como a $X-A$, entonces $C$ interseca a $\text{Fr} A$. De ejercicio 17.19 se define $\text{Fr} A=\overline{A}\cap \overline{(X-A)}$. Supongamos que $C$ es separable en $B$ y $D$, de tal manera que $C\cap \overline{A}= B$ y $C\cap \overline{(X-A)}=D$ son subespacios conexos, y supongamos que $C\cap \text{Fr}A\neq \varnothing$. Esto es $(C\cap \overline{(X-A)})\cap (C\cap \overline{A})\neq \varnothing$. Entonces $B\cap D\neq \varnothing$. Como $B$ y $D$ son subespacios conexos con elementos comunes, por lema 23.3, $B\cup D=C$ es conexo, lo que contradice la suposición de que $C$ es separable.
\section{Tema 2 Sección 23 Ejercicio 7}
Veamos si la topología del límite inferior $\mathbb{R}_\ell$ es conexa. Supongamos que $\mathbb{R}$ es una separación en los conjuntos $U=[a,b)$ con $a<b$ y $V=\mathbb{R}-U$. Entonces $b$ no es punto límite de $U$ porque exiten un abiertos $[b,x)$  con $b<x$ que contienen a $b$ y no intersecan a $U$. Por tanto, $U$ es cerrado puesto que contiene atodos sus puntos límite. Por tanto, $V$ es abierto. Luego $U$ y $V$ son una separación de $X$ puesto que son abiertos y cerrados a la vez tales que $U\cup V=X$. 
\section{Tema 2 Sección 23 Ejercicio 8}
Veamos si $\mathbb{R}^{\omega}$ con la topología uniforme es conexo. Por el teorema 20.4, la topología uniforme es mas fina que la topología producto, y mas gruesa que la topología por cajas sobre $\mathbb{R}^{\omega}$. Del ejemplo 7, $\mathbb{R}^{\omega}$ es conexo en la topología producto y no es conexo en la topología por cajas. Entonces por definición de $\overline{\rho}(\vect{x},\vect{y})$ se tiene que $\overline{\rho}(\vect{x},\vect{y})\leq 1$ para todo par de puntos $\vect{x},\vect{y}\in \mathbb{R}^{\omega}$. Por tanto $\overline{B_{\overline{\rho}}(\vect{x},1)}=B_{\overline{\rho}}(\vect{x},\epsilon)$ para todo $\epsilon> 1$. Supongamos que $\mathbb{R}$ es conexo en la topología uniforme. Sea $\widetilde{\mathbb{R}}^n$ el subespacio de $\mathbb{R}^{\omega}$ formado por el cunjunto de todas las sucesiones $\vect{x}$ tales que $x_i=0$ para $i>n$. Entonces $\widetilde{\mathbb{R}}^n$ es homeomorfo a $\mathbb{R}^n$. Por teoremas 23.5 y 23.6, $\widetilde{\mathbb{R}}^n$ es conexo en la topología de subespacio de la topología  uniforme. Además, el conjunto $\mathbb{R}^{\infty}=\bigcup_{n\in \mathbb{Z}} \widetilde{\mathbb{R}}^n$, formado por todas las sucesiones que tienen $x_i\neq 0$ sólo para un número finito de valores de $i$, también es conexo, puesto que tiene el elemento $\vect{0}$ en común con $\widetilde{\mathbb{R}}^n$. Sea $B_{\overline{\rho}}(\vect{a},\epsilon)\subset\mathbb{R}^{\omega}$ con $\epsilon>0$, $\vect{a}=(a_1,a_2,...)$. Sea $\vect{x}=(a_1,...,a_n,0,0,...)$ tal que pertenece a $\mathbb{R}^{\infty}$. Entonces existen $B_{\overline{\rho}}(\vect{a},\epsilon)$ tales que  $B_{\overline{\rho}}(\vect{a},\epsilon)\cap\mathbb{R}^\infty= \varnothing$ (por ejemplo, si $a_i\geq 1$ para algún $i\leq n$ y $\epsilon \leq 1$), y $\mathbb{R}^\infty$ no es la adherencia de $\mathbb{R}^\omega$ en la topología uniforme. Por tanto, aunque $\mathbb{R}^\infty$ es conexo, $\mathbb{R}^\omega$ es separable.
\end{document}
