\documentclass{article}
% Uncomment the following line to allow the usage of graphics (.png, .jpg)
%\usepackage[pdftex]{graphicx9990}o0y
% Comment the following line to NOT allow the usage ofp0 umlauts

%\usepackage[utf8]{inputenc}
%\usepackage{amsmath}
%\usepackage{amssymb}

\newcommand{\vect}[1]{\boldsymbol{#1}}
% Start the document00
\begin{document}
\section{Tema 3 Sección 27 Ejercicio 4}
Veamos que cualquier espacio métrico conexo con más de un punto es no numerable. Para el espacio métrico $(X,d_X)$ se tiene que, $U$ es abierto en la topología métrica si, y sólo si, para cada $x\in U$ existe un $\delta>0$ tal que $B_{d_X}(x,\delta)\subset U$. Se tiene además que el axioma de Hausdorff se cumple para toda topología métrica. Como en el teorema 27.7, veamos que ninguna función $f:\mathbb{Z}_+\rightarrow X$ puede ser sobreyectiva. Como por ser $X$ de Hausdorff, dado un conjunto no vacío $U$ y un $x\in X$, existe un subconjunto no vacío $V$ de tal manera que $V\subset U$ pero $x\notin \overline{V}$. Entonces dado sea $U=X$ elijamos $V_1$, entonces $V_1\subset X$ con $x_1\notin \overline{V}_1$. Igualmente se puede elegir un $V_2$ no vacío de tal manera que $V_2\subset V_1$ y tal que $x_2\notin \overline{V}_2$. Continuando así, se puede elegir un $V_{n+1}$ de tal manera que $V_{n+1}\subset V_n$ y que $x_n\notin \overline{V}_n$.  Como $X$ es conexo por hipótesis, se tiene que existe un $x$ tal que $x\in\bigcap_{n\in \mathbb{Z}_+} \overline{V}_n$. Porque de lo contrario, se tendría que $X=X-\bigcap_{n\in \mathbb{Z}_+} \overline{V}_n=\bigcup_{n\in \mathbb{Z}_+} (X-\overline{V}_n)$, contradiciendo el hecho de que es conexo. Por tanto, $f$ no es sobreyectiva y $X$ es no numerable.
\section{Tema 3 Sección 27 Ejercicio 5}
Sea $X$ un espacio compacto y de Hausdorff y $\{A_n\}$ una colección numerable de conjuntos cerrados de $X$. Veamos que si cada conjunto $A_n$ tiene interior vacío en $X$, entonces la unión $\bigcup_{n\in\mathbb{Z}_+} A_n$ tiene interior vacío en $X$. Supongamos que el interior de $\bigcup_{n\in\mathbb{Z}_+} A_n$ es no vacío. Recordemos que el interior de un conjunto es la unión de todos los abiertos contenidos en él. Entonces, dado un abierto no vacío $U$ queremos encontrar un $x$ de $U$ que no pertenezca a ninguno de los $A_n$. De este modo, $U$ no está contenido en $A_1$ y por tanto, se puede elegir un $y$ tal que $y\in U$ y $y\notin A_1$. Como $A_1$ es cerrado, escojamos un abierto no vacío $U_1$ tal que $\overline{U}_1\cap A_1=\varnothing$ y tal que $\overline{U}_1\subset U$. Entonces dado un abierto no vacío $U_{n}$, éste no está contenido en $A_{n+1}$ y se puede escoger un $U_{n+1}$ tal que $\overline{U}_{n+1}\cap A_{n+1}=\varnothing$ y que $\overline{U}_{n+1}\subset U_n$. De este modo, tenemos una sucesión encajada $\overline{U}\supset \overline{U}_1\supset \overline{U}_2...$. Como por teorema 26.9, $\bigcap_{n\in{\mathbb{Z}_+}}\overline{U}_n$ es no vacío, existe un $y\in \bigcap_{n\in\mathbb{Z}_+}\overline{U}_n$ que no pertenece a ningún $A_n$. Por tanto para todo abierto no vicío $U$ existe un $x$ tal que $x\in U$ y $x\notin A_n$, para cada $n\in \mathbb{Z}_+$. Esto es lo mismo que, dado un abierto no vacío $U$ que no está contenido en $\bigcup_{n\in \mathbb{Z}_+} A_n$ y por tanto, se puede elegir un $x$ tal que $x\in U$ y $x\notin \bigcup_{n\in \mathbb{Z}_+} A_n$.
\section{Tema 3 Sección 27 Ejercicio 6}
Sea $A_0=[0,1]$ subconjunto de $\mathbb{R}$. Y sea $A_n=A_{n-1}-\bigcup_{k=0}^\infty \left(\frac{1+3k}{3^n},\frac{2+3k}{3^n}\right)$. Sea el conjunto de cantor $C=\bigcap_{n\in \mathbb{Z}_+}A_n$.
\begin{itemize}
\item \bf (a) \rm Veamos que $C$ es totalmente disconexo.
\end{itemize}
Recordemos que un conjunto es totalmente disconexo si los únicos subespacios conexos son los conjuntos unipuntuales. Como
\begin{eqnarray} A_n=A_{n-1}-\bigcup_{k=0}^\infty\left(\frac{1+3k}{3^n},\frac{2+3k}{3^n}\right)\nonumber\\
=A_{n-1}\cap\left(\mathbb{R}-\bigcup_{k=0}^\infty\left(\frac{1+3k}{3^n},\frac{2+3k}{3^n}\right)\right)\nonumber\\
=[0,1]\cap\left[\bigcap_{i=1}^n\left(\mathbb{R}-\bigcup_{k=0}^\infty\left(\frac{1+3k}{3^i},\frac{2+3k}{3^i}\right)\right)\right],\nonumber
\end{eqnarray}
se tiene que los elementos de $C$ son del tipo $a=\frac{1+3k_1}{3^{n_{1}}}$, es decir, son números racionales. Procediendo como en el Ejemplo 4 de la sección 23, si $Y$ es un subespacio de $C$, conteniendo los puntos $a$ y $b$, es posible encontrar un irracional $c\notin C$ entre $a$ y $b$ tal que $Y$ es la unión de $Y\cap (-\infty,c)$ y $Y \cap (c,\infty)$. 
\begin{itemize}
\item \bf (b) \rm Veamos que $C$ es compacto.
\end{itemize}
Como la unión infinita de abiertos es abierta, $\bigcup_{k=0}^\infty \left(\frac{1+3k}{3^k},\frac{2+3k}{3^k}\right)$ es abierta por ser $\left(\frac{1+3k}{3^n},\frac{2+3k}{3^n}\right)$ intervalos abiertos de la topología usual sobre $\mathbb{R}$. Como $[0,1]$ es cerrado, $A_1=[0,1]\cap \left(\mathbb{R}-\bigcup_{k=0}^\infty \left(\frac{1+3k}{3},\frac{2+3k}{3}\right)\right)$ es cerrado. Por el mismo motivo, si $A_{n-1}$ es cerrado, $A_n=A_{n-1}-\bigcup_{k=0}^\infty\left(\frac{1+3k}{3^n},\frac{2+3k}{3^n}\right)=A_{n-1}\cap\left(\mathbb{R}-\bigcup_{k=0}^\infty\left(\frac{1+3k}{3^n},\frac{2+3k}{3^n}\right)\right)$ también es cerrado para todo $n\in \mathbb{Z}_+$. Entonces $A_0\supset A_1\supset A_2...$ es una sucesión encajada de cerrados no vacíos. Entonces, automáticamente, la sucesión $\{A_n\}_{n\in \mathbb{Z}_+\cup 0}$ tiene la propiedad de la intersección finita. Entonces, como $C=\bigcap_{n\in\mathbb{Z}_+}A_n$ es no vacío, y como $\mathbb{R}-C=\mathbb{R}-\bigcap_{n\in\mathbb{Z}_+\cup \{0\}}A_n=\bigcup_{n\in\mathbb{Z}_+\cup \{0\}}\left(\mathbb{R}-A_n\right)$ es abierto por ser unión infinita de abiertos, $C$ es un subconjunto cerrado de $\mathbb{R}$. Como es subconjunto de $\mathbb{R}$, $C$ está simplemente ordenado. Además tiene la propiedad del supremo porque cada subconjunto suyo que está acotado superiormente tiene supremo. Por teorema 27.1, $C$ es compacto.
\begin{itemize}
\item \bf (c) \rm Veamos que cada conjunto $A_n$ es la unión de intervalos cerrados disjuntos con longitud $1/3^n$ y que los extremos de dichos intervalos están en $C$.
\end{itemize}
Como se ha visto en apartado (a), $A_n=[0,1]\cap\left[\bigcap_{i=1}^n\left(\mathbb{R}-\bigcup_{k=0}^\infty\left(\frac{1+3k}{3^i},\frac{2+3k}{3^i}\right)\right)\right]$. Por tanto,
\begin{eqnarray}
A_0=[0,1]\nonumber\\
A_1=[0,\frac{1+3\cdot 0}{3^1}]\cup [\frac{2+3\cdot 0}{3^1},1]\nonumber\\
A_2=[0,\frac{1+3\cdot 0}{3^2}]\cup[\frac{2+3\cdot 0}{3^2},\frac{1+3\cdot 0}{3^1}]\cup [\frac{2+3\cdot 0}{3^1},\frac{1+3\cdot 2}{3^2}]\cup[\frac{2+3\cdot 2}{3^2},1]&&\nonumber\\
\vdots&&\nonumber\\
A_n=[0,1]\cap\bigcap_{i=1}^n\bigcap_{k=0}^\infty\left(\mathbb{R}-\left(\frac{1+3k}{3^i},\frac{2+3k}{3^i}\right)\right)&&\nonumber
\end{eqnarray}
Como $0\leq\frac{1+3k}{3^i}\leq 1$ y como $0\leq\frac{2+3k}{3^i}\leq 1$, se tiene que $k \leq \frac{3^i-1}{3}$ y que $k \leq \frac{3^i-2}{3}$. Por tanto, $k \leq 3^{i-1}-1$. Entonces
$A_n=\bigcap_{i=0}^n\bigcap_{k=0}^{3^{i-1}-1}\left(\left[0,\frac{1+3k}{3^i}\right]\cup\left[\frac{2+3k}{3^i},1\right]\right)$.
Por tanto, $\frac{1+3k}{3^i},\frac{2+3k}{3^i}\in A_n$ para todo $k\in \{0,...,3^{i-1}-1\}$ y todo $i\in \{0,..,n\}$. Además $A_n=\bigcup_{i=0}^n\bigcup_{k=0}^{3^{i-1}-1}\left[a_{i,k},b_{i,k}\right]$
donde
\begin{eqnarray}
a_{i,k}=\min\{\frac{2+3k}{3^i},\frac{2+3(k+1)}{3^i},\frac{2+3k}{3^{(i+1)}},\frac{2+3(k+1)}{3^{(i+1)}}\}\nonumber\\
b_{i,k}=\max\{\frac{1+3k}{3^i},\frac{1+3(k+1)}{3^i},\frac{1+3k}{3^{(i+1)}},\frac{1+3(k+1)}{3^{(i+1)}}\}\nonumber
\end{eqnarray}
\begin{itemize}
\item \bf (d) \rm Veamos que $C$ no tiene puntos aislados.
\end{itemize}
Veamos que el conjunto $\{x\}\subset C$ no es abierto para ningún $x\in C$. Como $C$ es compacto por apartado (b), $\{x\}$ es compacto. Dado que $\{x\}$ es compacto y $C$ es de Hausdorff (por teorema 17.11), $\{x\}$ es cerrado. Aunque $C$ es totalmente disconexo, no puede ser $\{x\}$ abierto y cerrado a la vez, porque si no, $C-\{x\}$ sería abierto y cerrado a la vez, lo cual es absurdo.
\begin{itemize}
\item \bf (e) \rm Veamos que $C$ es no numerable.
\end{itemize}
Por teorema 27.7, $C$ es no numerable.
\end{document}
