\documentclass{article}
% Uncomment the following line to allow the usage of graphics (.png, .jpg)
%\usepackage[pdftex]{graphicx}
% Comment the following line to NOT allow the usage of umlauts

% Start the document
\begin{document}

% Create a new 1st level heading

\section{Sección 2 Ejercicio 4}
Sea \(f:A \longrightarrow B\) y \(g:B \longrightarrow C\) y \(C_0 \subset C\).
\newline
\bf (a) \rm 
\newline Demostración de que
\begin{equation}
\left( g \circ f\right)^{-1} \left(C_0\right) = f^{-1}\left(g^{-1} \left(C_0 \right) \right)
\end{equation}
Se tiene que
\begin{equation}
\begin{aligned}
\left( g \circ f \right)^{-1} \left( C_0 \right) = \left\{ a | \left(g \circ f \right)\left( a\right) \in C_0 \right\}
\\
= \left\{ a | \left(a,c\right) \in \left\{\left(a',c' \right)|\text{ para algún } b \in B, f\left( a'\right)= b \text{ y }  g \left( b \right)=c' \right\}, c \in C_0 \right\}
\\
= \left\{ a | \text{ para algún } b \in B, f\left( a\right)= b \text{ y }  g \left( b \right) \in C_0 \right\}
\\
= \left\{ a | \text{ para algún } b \in B, f\left( a\right)= b \text{ y }  b \in \left\{b'| g \left( b' \right) \in C_0 \right\} \right\}
\\
= \left\{ a | \text{ para algún } b \in B, f\left( a\right)= b \text{ y }  b \in g^{-1}\left( C_0\right) \right\}
\\
= \left\{ a | f \left(a\right) \in g^{-1}\left( C_0\right) \right\}
\\
=f^{-1} \left( g^{-1}\left( C_0\right) \right)
\end{aligned}
\end{equation}
\bf (b) \rm Si \(f\) inyectiva y \(g\) inyectiva, entonces \( g\circ f \) es inyectiva. 
\newline
Sea \(f\) inyectiva, entonces \( [ f(a)=f(a')]\Rightarrow [a=a']\)
\newline
Sea \(g\) inyectiva, entonces \( [ g(b)=g(b')]\Rightarrow [b=b']\)
\newline
Por tanto, \([(g \circ f)(a) = (g \circ f)(a')] \Leftrightarrow \)
\newline
\(\Leftrightarrow [g ( f(a)) = g ( f(a'))] \Rightarrow   [ f(a) = f(a')] \Rightarrow [a=a'] \)
\newline
\bf (c) \rm Si \([(g \circ f)(a) = (g \circ f)(a')] \Rightarrow [a=a'] \), se puede a firmar que \([f(a) = g (a')] \Rightarrow [a=a'] \). Veamos que si \([(g \circ f)(a) = (g \circ f)(a')] \Rightarrow [a=a'] \), se puede a firmar que \([g(f(a))=g(f(a'))] \Rightarrow [a=a'] \Rightarrow [f(a)=f(a')]\) por la definición de \(f\). Además, supongamos que \(f\) es no inyectiva, es decir, \(f(a)=f(a')=b \) y \(a \neq a' \) para algún \(b \in B\), entonces \([f(a)=f(a')=b\) y \(g(f(a)) \neq g(f(a'))] \), luego \( [ g(b) \neq g(b)\) para algún \(b]\). Lo cual es absurdo. Por tanto, si \(g \circ f\) es inyectiva, entonces \(f\) es inyectiva.
\newline
\bf (d) \rm Si \(f\) sobreyectiva y \(g\) sobreyectiva, entonces \( g\circ f \) es sobreyectiva. 
\begin{equation}
\begin{aligned}
\text{ Es decir, si}
\\
[b \in B] \Rightarrow [b=f\left( a\right) \text{ para almenos un } a \in A]
\\ \text{y}
\\
[c \in C] \Rightarrow [c=g\left( b\right) \text{ para almenos un } b \in B]
\\ \text{entonces}
\\
[c \in C] \Rightarrow [c=g\left( b\right) \text{ para almenos un } [b \in B]]
\\ \Rightarrow [c=g\left( b\right)
\text{ para almenos un } [b=f\left( a\right) \text{ para almenos un } a \in A]]
\\ \text{ por tanto } \\
[c \in C] \Rightarrow [c=g\left( f\left( a\right)\right) \text{ para almenos un } a \in A]
\end{aligned}
\end{equation}
Por tanto, \(g \circ f \) es sobreyectiva.
\newline
\bf (e) \rm Sea \(g \circ f\) sobreyectiva. Entonces
\begin{equation}
\begin{aligned}\\
[c \in C] \Rightarrow [c=g\left( f\left( a\right)\right) \text{ para almenos un } a \in A]
\end{aligned}
\end{equation}
Es decir
\begin{equation}
\begin{aligned}
\\
[c \in C] \Rightarrow [(a,c) \in \left\{ \left(a'',c'' \right)|\text{ para algún } b \in B, f\left( a''\right)= b \text{ y } \right. \\ \left.  g \left( b \right)=c'' \right\} \text{ para almenos un } a \in A]
\end{aligned}
\end{equation}
Por lo tanto,
\begin{equation}
\begin{aligned}
\\
[c \in C] \Rightarrow [g\left( b\right)= c\text{ para almenos un } b \in B]
\end{aligned}
\end{equation}
Entonces \(g\) es sobreyectiva.
\newline
\bf (f) \rm Si \(g\) y \(f\) son inyectivas, \( g \circ f\) es inyectiva.\newline Si \(g\) y \(f\) son sobreyectivas, \( g \circ f\) es sobreyectiva.\newline Si  \( g \circ f\) es inyectiva, \( f\) es inyectiva. \newline Si \( g \circ f\) es sobreyectiva,  \( g \) es sobreyectiva.
\end{document}
